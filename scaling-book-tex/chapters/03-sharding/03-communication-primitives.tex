\section*{A Deeper Dive into TPU Communication Primitives}
\addcontentsline{toc}{section}{A Deeper Dive into TPU Communication Primitives}

The previous 4 cases have introduced several ``core communication primitives'' used to perform sharded matrix multiplications:

\begin{enumerate}
    \item \textbf{AllGather:} removes a subscript from a sharding, gathering the shards.
    \item \textbf{ReduceScatter:} removes an ``un-reduced'' suffix from an array by summing shards over that axis, leaving the array sharded over a second axis.
    \item \textbf{AllReduce:} removes an ``un-reduced'' suffix, leaving the array unsharded along that axis.
\end{enumerate}

There's one more core communication primitive to mention that arises in the case of Mixture of Experts (MoE) models and other computations: the \textbf{AllToAll}.

\subsection*{Our final communication primitive: the AllToAll}
\addcontentsline{toc}{subsection}{Our final communication primitive: the AllToAll}

A final fundamental collective which does not occur naturally when considering sharded matrix multiplies, but which comes up constantly in practice, is the \textbf{AllToAll} collective, or more precisely the special case of a \emph{sharded transposition} or resharding operation. e.g.

$$\textbf{AllToAll}_{X, J} A[I_X, J] \rightarrow A[I, J_X]$$

AllToAlls are typically required to rearrange sharded layouts between different regions of a sharded computation that don't have compatible layout schemes. They arise naturally when considering sharded mixture-of-experts models. \emph{You can think of an AllToAll as moving a subscript from one axis to another}. Because an all to all doesn't need to replicate all of the data of each shard across the ring, it's actually \emph{cheaper} than an AllGather (by a factor of ¼)\footnote{For even-sized bidirectional rings, each device will send $(N/2 + (N/2-1) + \ldots + 1)$ chunks right and $((N/2-1) + \ldots + 1)$ chunks left $= 0.5 \cdot (N / 2) \cdot (N/2 + 1) + 0.5 \cdot (N / 2) \cdot (N/2 - 1) = N^2/4$. The size of each chunk (aka shard of a shard) is $\text{bytes} / N^2$ so the per-device cost is $(\text{bytes} / N^2) \cdot N^2 / 4 = \text{bytes} / 4$. This result scales across all devices as the total bandwidth scales with device number.}.

If we generalize to an ND AllToAll, the overall cost for an array of $V$ bytes on an AxBxC mesh is

$$T_\text{comms per AllToAll} = \frac{V \cdot \max(A, B, C, \ldots)}{4 \cdot N \cdot W_\text{ici}}$$

where as usual $W_\text{ici}$ is the bidirectional ICI bandwidth. For a 1D mesh, this reduces to $V / (4 \cdot W_\text{ici})$, which is 1 / 4 the cost of an AllReduce. In 2D, the cost actually scales down with the size of the smallest axis.

\emph{Aside: If you want a hand-wavy derivation of this fact, start with a 1D torus $\mathbb{Z} / N\mathbb{Z}$. If we pick a source and target node at random, they are on average N / 4 hops from each other, giving us a cost of $(V \cdot N) / (4 * N)$. Now if we consider an ND torus, each axis is basically independent. Each node has $1 / Z$ bytes and on average has to hop its data $\max(A, B, C, \ldots) / 4$ hops.}

\subsection*{More about the ReduceScatter}
\addcontentsline{toc}{subsection}{More about the ReduceScatter}

ReduceScatter is a more fundamental operation than it first appears, as it is actually the derivative of an AllGather, and vice versa. i.e. if in the forward pass we have:

$$\textbf{AllGather}_X A[I_X] \rightarrow A[I]$$

Then we ReduceScatter the reverse-mode derivatives \textbf{A'} (which will in general be different on each shard) to derive the sharded \textbf{A'}:

$$\textbf{ReduceScatter}_X A'[I] \{ U_X \} \rightarrow A'[I_X]$$

Likewise, $\text{ReduceScatter}_X(A[I] \{U_X\}) \to A[I_X]$ in the forward pass implies $\text{AllGather}_{X}(A'[I_X]) \to A'[I]$ in the backwards pass.

\paragraph{How AllGather and ReduceScatter are derivatives of eachother}

This stems from the fact that broadcasts and reductions are transposes of eachother as linear operators, and AllGather and ReduceScatter are outer products (also known as \href{https://en.wikipedia.org/wiki/Kronecker_product}{Kronecker products}) of broadcast and reduce, respectively. Concretely, if we have a vector $x \in \mathbb{R}^n$, any number of devices $p \in \mathbb{N}$, and we let $u = (1, \ldots, 1) \in \mathbb{R}^p$, we can define broadcast and reduce in the following way, which should match your intuitive understanding of them:

\begin{align*}
\text{broadcast} &: \mathbb{R}^n \rightarrow \mathbb{R}^{p n} \\
\text{broadcast} &= u \otimes \mathbf{I}_n \\
\text{reduce} &: \mathbb{R}^{p n} \rightarrow \mathbb{R}^n \\
\text{reduce} &= u^T \otimes \mathbf{I}_n
\end{align*}

Let's see how this looks in an example, where $n = 1$, $p = 2$. If $x = (7)$, we have $$\text{broadcast}(x) = \left(\begin{pmatrix} 1 \\ 1 \end{pmatrix} \otimes \begin{pmatrix} 1 \end{pmatrix}\right) x = \begin{pmatrix} 1 \\ 1 \end{pmatrix} x = \begin{pmatrix}  7\\  7  \end{pmatrix} \in \mathbb{R}^{p n}$$. This matches what we'd expect, broadcasting a vector in $\mathbb{R}^n$ to $\mathbb{R}^{pn}$. Now letting $y = (8, 9)$, we have $$\text{reduce}(y) = \left(\begin{pmatrix} 1 & 1 \end{pmatrix} \otimes \begin{pmatrix} 1\end{pmatrix}\right) y = \begin{pmatrix} 1 & 1  \end{pmatrix} \begin{pmatrix}  8 \\ 9  \end{pmatrix} = \begin{pmatrix}   17    \end{pmatrix}.$$ This again matches what we'd expect, reducing a vector in $\mathbb{R}^{p n}$ to a vector in $\mathbb{R}^{n}$. Since $(A \otimes B)^T = A^T \otimes B^T$ for any two matrices $A$ and $B$, we see that $\text{reduce} = \text{broadcast}^T$. We recover AllGather and ReduceScatter as the following outer products:

\begin{align*}
\text{AllGather} &: \mathbb{R}^{p n} \rightarrow \mathbb{R}^{p^2 n} \\
\text{AllGather} &= \text{broadcast} \otimes \mathbf{I}_p \\
\text{ReduceScatter} &= \mathbb{R}^{p^2 n} \rightarrow \mathbb{R}^{p n} \\
\text{ReduceScatter} &= \text{reduce} \otimes \mathbf{I}_p
\end{align*}

Here we think of $\mathbb{R}^{p^2 n}$ as $\mathbb{R}^{p \times p n}$, so one $\mathbb{R}^{p n}$ vector for each of our $p$ devices. We suggest playing around with small examples, say $n = 2$, $p = 3$, to see what these operators look like as matrices. Using the same transposition property, we once more obtain $\text{AllGather}^T = \text{ReduceScatter}$, and of course $\text{ReduceScatter}^T = \text{AllGather}$. This transposition will arise during backpropagation, since if we have $y = Ax$ for some linear operator $A$, such as AllGather or ReduceScatter, then during backpropagation we will have the derivative of the loss with respect to $y$, $\frac{\partial L}{\partial y}$, and we obtain $\frac{\partial L}{\partial x}$ as $\frac{\partial L}{\partial x} = A^T \frac{\partial L}{\partial y}$. This shows how the derivative of AllGather will be ReduceScatter, and viceversa.

Turning an AllReduce into an AllGather and ReduceScatter also has the convenient property that we can defer the final AllGather until some later moment. Very commonly we'd rather not pay the cost of reassembling the full matrix product replicated across the devices. Rather we'd like to preserve a sharded state even in this case of combining two multiplicands with sharded contracting dimensions:

$$A[I, J_X] \cdot B[J_X, K] \rightarrow C[I, K_X]$$

In this case, we can also perform a ReduceScatter instead of an AllReduce, and then optionally perform the AllGather at some later time, i.e.

\begin{align*}
A[I, J_X] \cdot_{LOCAL} B[J_X, K] \rightarrow &\ C[I, K] \{ U_X \} \\
\textbf{ReduceScatter}_{X,K} C[I, K] \{ U_X \} \rightarrow &\ C[I, K_X]
\end{align*}

Note that ReduceScatter \emph{introduces} a sharded dimension, and so has a natural freedom to shard along either the \textbf{I} or \textbf{K} named dimensions in this case. We generally need to choose \emph{which} named dimension to introduce a new sharding to when using a ReduceScatter (though the choice is usually forced by the larger modeling context). This is why we use the syntax \textbf{ReduceScatter\textsubscript{X,K}} to specify the axis to shard.
