\subsection{Trace Viewer}

\textbf{The Trace Viewer is probably the most useful part of the profiler.} The example below shows a simple Transformer with pieces annotated. Names come from labels provided in the code.

\begin{figure}[htb]
\centering
\includegraphics[width=\textwidth]{images/trace-viewer.png}
\caption{The Trace Viewer showing a chronological timeline of operations for a simple Transformer. The top row shows XLA operations (HLO names), while lower rows show approximate traces from JAX scope annotations.}
\label{fig:trace-viewer}
\end{figure}

The Trace Viewer shows a chronological timeline of all the actions on each TPU core. We're only looking at TPU:0 here, since typically all TPUs execute the same instructions. A few key notes:

\begin{enumerate}
\item The top row (XLA Ops) shows the actual TPU operations (the names are HLO names). Everything else is an approximate trace based on \texttt{jax.named\_scope}, \texttt{jax.named\_call}, and the Python stack trace.
\item Noting the repeated blocks, we can isolate a single layer here. We can also see (from looking at the code/understanding how a Transformer works) what parts are attention and what parts are MLPs.
\item By clicking on an XLA op, we can view where in the code it comes from (useful for understanding the trace) and see links to the Graph viewer.
\end{enumerate}

\begin{takeawaybox}
You can navigate the Trace Viewer using ``video game'' style controls, with A/D panning left and right, and W/S zooming in and out. These controls make navigating a lot easier.
\end{takeawaybox}
