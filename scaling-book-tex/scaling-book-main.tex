% Pandoc LaTeX template for B6 two-sided book
\documentclass[10pt,twoside]{book}

% Font setup - New Computer Modern
\usepackage{fontsetup}

% B6 paper size with two-sided margins
% Using tighter margins to accommodate wide tables and math
\usepackage[
    b6paper,
    inner=1.6cm,      % Binding side margin (more space for binding)
    outer=0.8cm,    % Outer edge margin (reduced for more text width)
    top=0cm,        % No top margin for maximum content space
    bottom=0.4cm,   % Minimal bottom margin for page numbers
    includeheadfoot
]{geometry}

% Allow some text to extend into margins if needed for math overflow
\setlength{\emergencystretch}{3em}
\setlength{\hfuzz}{0.5pt}

% Essential packages
\usepackage{amsmath,amssymb}

% Allow breaking of long equations across lines and pages
\allowdisplaybreaks

% Allow more flexible math spacing to prevent overflow
\binoppenalty=100
\relpenalty=100

% Make math environments more tolerant of tight spaces
\setlength{\mathsurround}{0pt}

% Use slightly smaller font for math if needed
\DeclareMathSizes{10}{10}{7}{5}

\usepackage{graphicx}

% Set maximum image dimensions to fit within text area
\setkeys{Gin}{width=\linewidth,height=0.8\textheight,keepaspectratio}

\usepackage{longtable,booktabs}
\usepackage{calc} % for calculating minipage widths
\usepackage{microtype}

% Better table handling for small pages
\usepackage{adjustbox}
\usepackage{etoolbox}
\usepackage{pdflscape}  % For landscape pages
\usepackage{rotating}   % For rotating tables without rotating pages

% Make tables use small font and tighter spacing to fit B6 pages
% Only apply to longtable (content tables), not tabular (which includes title page author list)
\AtBeginEnvironment{longtable}{\scriptsize\setlength{\tabcolsep}{2.5pt}}

% Code highlighting - Define pandoc syntax highlighting colors and commands
\usepackage{xcolor}
\usepackage{fancyvrb}

% Define colors for syntax highlighting
\definecolor{shadecolor}{RGB}{248,248,248}
\definecolor{keywordcolor}{RGB}{0,0,255}
\definecolor{stringcolor}{RGB}{163,21,21}
\definecolor{commentcolor}{RGB}{0,128,0}
\definecolor{numbercolor}{RGB}{9,134,88}
\definecolor{operatorcolor}{RGB}{102,102,102}

% Define Shaded environment for code blocks
\newenvironment{Shaded}{%
  \begin{snugshade}%
  \footnotesize%
}{%
  \end{snugshade}%
}

% Define snugshade environment
\makeatletter
\newenvironment{snugshade}{%
  \def\FrameCommand##1{\hskip\@totalleftmargin
    \colorbox{shadecolor}{##1}%
    \hskip-\linewidth \hskip-\@totalleftmargin \hskip\columnwidth}%
  \MakeFramed {\advance\hsize-\width
    \@totalleftmargin\z@ \linewidth\hsize
    \@setminipage}}%
{\par\unskip\endMakeFramed}
\makeatother

\usepackage{framed}

% Define Highlighting environment with line wrapping
% Using scriptsize for better fit on B6 pages
\DefineVerbatimEnvironment{Highlighting}{Verbatim}{commandchars=\\\{\},fontsize=\scriptsize}

% Define all syntax highlighting token commands
\newcommand{\AlertTok}[1]{\textcolor{red}{\textbf{#1}}}
\newcommand{\AnnotationTok}[1]{\textcolor{commentcolor}{\textit{#1}}}
\newcommand{\AttributeTok}[1]{\textcolor{numbercolor}{#1}}
\newcommand{\BaseNTok}[1]{\textcolor{numbercolor}{#1}}
\newcommand{\BuiltInTok}[1]{\textcolor{keywordcolor}{#1}}
\newcommand{\CharTok}[1]{\textcolor{stringcolor}{#1}}
\newcommand{\CommentTok}[1]{\textcolor{commentcolor}{\textit{#1}}}
\newcommand{\CommentVarTok}[1]{\textcolor{commentcolor}{\textit{#1}}}
\newcommand{\ConstantTok}[1]{\textcolor{numbercolor}{#1}}
\newcommand{\ControlFlowTok}[1]{\textcolor{keywordcolor}{\textbf{#1}}}
\newcommand{\DataTypeTok}[1]{\textcolor{keywordcolor}{#1}}
\newcommand{\DecValTok}[1]{\textcolor{numbercolor}{#1}}
\newcommand{\DocumentationTok}[1]{\textcolor{commentcolor}{\textit{#1}}}
\newcommand{\ErrorTok}[1]{\textcolor{red}{\textbf{#1}}}
\newcommand{\ExtensionTok}[1]{#1}
\newcommand{\FloatTok}[1]{\textcolor{numbercolor}{#1}}
\newcommand{\FunctionTok}[1]{\textcolor{blue}{#1}}
\newcommand{\ImportTok}[1]{\textcolor{keywordcolor}{\textbf{#1}}}
\newcommand{\InformationTok}[1]{\textcolor{commentcolor}{\textit{#1}}}
\newcommand{\KeywordTok}[1]{\textcolor{keywordcolor}{\textbf{#1}}}
\newcommand{\NormalTok}[1]{#1}
\newcommand{\OperatorTok}[1]{\textcolor{operatorcolor}{#1}}
\newcommand{\OtherTok}[1]{\textcolor{numbercolor}{#1}}
\newcommand{\PreprocessorTok}[1]{\textcolor{keywordcolor}{#1}}
\newcommand{\RegionMarkerTok}[1]{#1}
\newcommand{\SpecialCharTok}[1]{\textcolor{stringcolor}{#1}}
\newcommand{\SpecialStringTok}[1]{\textcolor{stringcolor}{#1}}
\newcommand{\StringTok}[1]{\textcolor{stringcolor}{#1}}
\newcommand{\VariableTok}[1]{\textcolor{blue}{#1}}
\newcommand{\VerbatimStringTok}[1]{\textcolor{stringcolor}{#1}}
\newcommand{\WarningTok}[1]{\textcolor{red}{\textit{#1}}}

% Hyperref for cross-references
\usepackage[
    colorlinks=true,
    linkcolor=blue!70!black,
    citecolor=green!60!black,
    urlcolor=blue!70!black,
    bookmarks=true
]{hyperref}

% Bibliography support
\usepackage[authoryear]{natbib}
\bibliographystyle{plainnat}

% Custom environment for takeaway boxes
\usepackage[most]{tcolorbox}
\newtcolorbox{takeawaybox}{
    colback=blue!5!white,
    colframe=blue!75!black,
    fonttitle=\bfseries,
    title=Key Takeaway,
    breakable,
    before skip=\baselineskip,
    after skip=\baselineskip
}

% Custom environment for algorithm boxes
\newtcolorbox{algorithmbox}{
    colback=gray!5!white,
    colframe=black!75,
    fonttitle=\bfseries,
    title=Algorithm,
    breakable,
    before skip=\baselineskip,
    after skip=\baselineskip
}

% Pandoc custom div support
\newenvironment{takeaway}{\begin{takeawaybox}}{\end{takeawaybox}}

% Chapter and section formatting
\usepackage{titlesec}
\titleformat{\chapter}[display]
  {\normalfont\huge\bfseries}{\chaptertitlename\ \thechapter}{20pt}{\Huge}
\titlespacing*{\chapter}{0pt}{-20pt}{40pt}

% Page headers and footers - simple page numbers in footer only
\usepackage{fancyhdr}
\pagestyle{fancy}
\fancyhf{} % Clear all headers and footers
\fancyfoot[C]{\thepage} % Center page number in footer
\renewcommand{\headrulewidth}{0pt} % Remove header line
\renewcommand{\footrulewidth}{0pt} % No footer line

% Make sure figures don't float too far
\usepackage{float}
\let\origfigure\figure
\let\endorigfigure\endfigure
\renewenvironment{figure}[1][]{%
  \origfigure[H]
}{%
  \endorigfigure
}

% Adjust figure and table captions
\usepackage[font=small,labelfont=bf]{caption}

% Handle pandoc's tight lists
\providecommand{\tightlist}{%
  \setlength{\itemsep}{0pt}\setlength{\parskip}{0pt}}

% Title and author information
\title{How to Scale Your Model\\[0.5em]
\large A Systems View of LLMs on TPUs}

\author{
Jacob Austin\\
Sholto Douglas\\
Roy Frostig\\
Anselm Levskaya\\
Charlie Chen\\
Sharad Vikram\\
Federico Lebron\\
Peter Choy\\
Vinay Ramasesh\\
Albert Webson\\
Reiner Pope\textsuperscript{*}\\[1em]
\textit{Google DeepMind}
}

\date{February 4, 2025}

\begin{document}

% Use sloppy formatting to prevent overflow (allows more spacing)
\sloppy

% Temporarily use centered geometry for title page
\newgeometry{
    b6paper,
    left=1.4cm,     % Equal margins on both sides for centered title
    right=1.4cm,
    top=0cm,
    bottom=0.3cm,
    includeheadfoot
}
\maketitle
% Restore two-sided margins for content
\thispagestyle{empty}
\restoregeometry

\newpage
\thispagestyle{empty}
\ % The empty page
\newpage
\thispagestyle{empty}
\ % The empty page
\thispagestyle{empty}
% Table of contents
% \tableofcontents
\thispagestyle{empty}

% Main content - chapters
\chapter*{Introduction}
\addcontentsline{toc}{chapter}{Introduction}

Much of deep learning still boils down to a kind of black magic, but optimizing the performance of your models doesn't have to—even at huge scale! Relatively simple principles apply everywhere—from dealing with a single accelerator to tens of thousands—and understanding them lets you do many useful things:

\begin{itemize}
\item Ballpark how close parts of your model are to their theoretical optimum.
\item Make informed choices about different parallelism schemes at different scales (how you split the computation across multiple devices).
\item Estimate the cost and time required to train and run large Transformer models.
\item Design algorithms that take advantage of specific hardware affordances.
\item Design hardware driven by an explicit understanding of what limits current algorithm performance.
\end{itemize}

\textbf{Expected background:} We're going to assume you have a basic understanding of LLMs and the Transformer architecture but not necessarily how they operate at scale. You should know the basics of LLM training and ideally have some basic familiarity with JAX. Some useful background reading might include this blog post\footnote{\url{https://jalammar.github.io/illustrated-transformer/}} on the Transformer architecture and the original Transformer paper\footnote{\url{https://arxiv.org/abs/1706.03762}}. Also check this list out for more useful concurrent and future reading.

\textbf{Goals \& Feedback:} By the end, you should feel comfortable estimating the best parallelism scheme for a Transformer model on a given hardware platform, and roughly how long training and inference should take. If you don't, email us or leave a comment! We'd love to know how we could make this clearer.

\section*{Why should you care?}
\addcontentsline{toc}{section}{Why should you care?}

Three or four years ago, I don't think most ML researchers would have needed to understand any of the content in this book. But today even ``small'' models run so close to hardware limits that doing novel research requires you to think about efficiency at scale.\footnote{Historically, ML research has followed something of a tick-tock cycle between systems innovations and software improvements. Alex Krizhevsky had to write unholy CUDA code to make CNNs fast but within a couple years, libraries like Theano and TensorFlow meant you didn't have to. Maybe that will happen here too and everything in this book will be abstracted away in a few years. But scaling laws have pushed our models perpetually to the very frontier of our hardware, and it seems likely that, for the forseeable future, doing cutting edge research will be inextricably tied to an understanding of how to efficiently scale models to large hardware topologies.} \textbf{A 20\% win on benchmarks is irrelevant if it comes at a 20\% cost to roofline efficiency.} Promising model architectures routinely fail either because they \emph{can't} run efficiently at scale or because no one puts in the work to make them do so.

\textbf{The goal of ``model scaling'' is to be able to increase the number of chips used for training or inference while achieving a proportional, linear increase in throughput.} This is known as ``\emph{strong scaling}''. Although adding additional chips (``parallelism'') usually decreases the computation time, it also comes at the cost of added communication between chips. When communication takes longer than computation we become ``communication bound'' and cannot scale strongly.\footnote{As your computation time decreases, you also typically face bottlenecks at the level of a single chip. Your shiny new TPU or GPU may be rated to perform 500 trillion operations-per-second, but if you aren't careful it can just as easily do a tenth of that if it's bogged down moving parameters around in memory. The interplay of per-chip computation, memory bandwidth, and total memory is critical to the scaling story.} If we understand our hardware well enough to anticipate where these bottlenecks will arise, we can design or reconfigure our models to avoid them.\footnote{Hardware designers face the inverse problem: building hardware that provides just enough compute, bandwidth, and memory for our algorithms while minimizing cost. You can imagine how stressful this ``co-design'' problem is: you have to bet on what algorithms will look like when the first chips actually become available, often 2 to 3 years down the road. The story of the TPU is a resounding success in this game. Matrix multiplication is a unique algorithm in the sense that it uses far more FLOPs per byte of memory than almost any other (N FLOPs per byte), and early TPUs and their systolic array architecture achieved far better perf / \$ than GPUs did at the time they were built. TPUs were designed for ML workloads, and GPUs with their TensorCores are rapidly changing to fill this niche as well. But you can imagine how costly it would have been if neural networks had not taken off, or had changed in some fundamental way that TPUs (which are inherently less flexible than GPUs) could not handle.}

\emph{Our goal in this book is to explain how TPU (and GPU) hardware works and how the Transformer architecture has evolved to perform well on current hardware. We hope this will be useful both for researchers designing new architectures and for engineers working to make the current generation of LLMs run fast.}

\section*{High-Level Outline}
\addcontentsline{toc}{section}{High-Level Outline}

\begin{figure}[htb]
\centering
\includegraphics[width=\textwidth]{images/transformer-diagram.png}
\caption{A standard Transformer layer with each matrix multiplication (matmul) shown as a dot inside a circle. All parameters (excluding norms) are shown in purple. Section 4 walks through this diagram in more detail.}
\end{figure}

The overall structure of this book is as follows:

Section 1 explains roofline analysis and what factors can limit our ability to scale (communication, computation, and memory). Section 2 and Section 3 talk in detail about how TPUs work, both as individual chips and—of critical importance—as an interconnected system with inter-chip links of limited bandwidth and latency. We'll answer questions like:

\begin{itemize}
\item How long should a matrix multiply of a certain size take? At what point is it bound by compute or by memory or communication bandwidth?
\item How are TPUs wired together to form training clusters? How much bandwidth does each part of the system have?
\item How long does it take to gather, scatter, or re-distribute arrays across multiple TPUs?
\item How do we efficiently multiply matrices that are distributed differently across devices?
\end{itemize}

Five years ago ML had a colorful landscape of architectures—ConvNets, LSTMs, MLPs, Transformers—but now we mostly just have the Transformer. We strongly believe it's worth understanding every piece of the Transformer architecture: the exact sizes of every matrix, where normalization occurs, how many parameters and FLOPs\footnote{FLoating point OPs, basically the total number of adds and multiplies required. While many sources take FLOPs to mean ``operations per second'', we use FLOPs/s to indicate that explicitly.} are in each part. Section 4 goes through this ``Transformer math'' carefully, showing how to count the parameters and FLOPs for both training and inference. This tells us how much memory our model will use, how much time we'll spend on compute or comms, and when attention will become important relative to the feed-forward blocks.

Section 5: Training and Section 7: Inference are the core of this essay, where we discuss the fundamental question: given a model of some size and some number of chips, how do I parallelize my model to stay in the ``strong scaling'' regime? This is a simple question with a surprisingly complicated answer. At a high level, there are 4 primary parallelism techniques used to split models over multiple chips (\textbf{data}, \textbf{tensor}, \textbf{pipeline} and \textbf{expert}), and a number of other techniques to reduce the memory requirements (\textbf{rematerialisation}, \textbf{optimizer/model sharding (aka ZeRO)}, \textbf{host offload}, \textbf{gradient accumulation}). We discuss many of these here.

We hope by the end of these sections you should be able to choose among them yourself for new architectures or settings. Section 6 and Section 8 are practical tutorials that apply these concepts to LLaMA-3, a popular open-source model.

Finally, Section 9 and Section 10 look at how to implement some of these ideas in JAX and how to profile and debug your code when things go wrong. Section 12 is a new section that dives into GPUs as well.

Throughout we try to give you problems to work for yourself. Please feel no pressure to read all the sections or read them in order. And please leave feedback. For the time being, this is a draft and will continue to be revised. Thank you!

\emph{We'd like to acknowledge James Bradbury and Blake Hechtman who derived many of the ideas in this doc.}
\chapter{All About Rooflines}

\section*{Where Does the Time Go?}
\addcontentsline{toc}{section}{Where Does the Time Go?}

Let's start with an extremely simple question: \emph{why does an algorithm take 50ms instead of 50s or 5ms}? What is actually happening within the model that takes substantial time and how long should we expect it to take?

\textbf{Computation:} A deep learning model is effectively a bunch of matrix multiplications, each composed of floating-point multiplication and addition `operations' (FLOPs). Our accelerator speed determines how long these take to compute:

\begin{equation}
T_\text{math} = \frac{\text{Computation FLOPs}}{\text{Accelerator FLOPs/s}}
\end{equation}

For instance, an NVIDIA H100 can perform about 9.89e14 bfloat16\footnote{bf16 is short for \href{https://en.wikipedia.org/wiki/Bfloat16_floating-point_format}{bfloat16}, a 16-bit floating point format often used in ML.} FLOPs/s while a TPU v6e can perform 9.1e14 FLOPs/s.\footnote{H100s and B200s can usually only achieve around 80-85\% of the claimed peak FLOPs, while TPUs can get closer to 95\% in normal use.} That means doing 1e12 FLOPs on an H100 will take (roughly) \texttt{1e12 / 9.89e14 = 1.01ms} and \texttt{1e12 / 9.1e14 = 1.1ms} on a TPU v6e.\footnote{Note that these chips are priced differently, and this comparison does not normalize to cost.}

\textbf{Communication within a chip:} \emph{Within an accelerator}, tensors need to be transferred between on-chip memory (HBM) and the compute cores. You'll see the bandwidth of this link referred to as ``HBM bandwidth''\footnote{NVIDIA also calls this ``memory bandwidth.''} On an H100, this is about 3.35TB/s and on TPU v6e this is about 1.6TB/s.

\textbf{Communication between chips:} When we distribute a model \emph{across multiple accelerators}, tensors frequently need to be transferred between them. There are often a few options for this on our hardware (ICI, DCN, and PCIe), each with different bandwidths.

Whether the communication is within a chip or between chips, we measure this in bytes/s and estimate the total communication time with:

\begin{equation}
T_\text{comms} = \frac{\text{Communication Bytes}}{\text{Network/Memory Bandwidth Bytes/s}}
\end{equation}

Typically (but not always), computation within a single chip can be overlapped with communication within a chip and between chips. This means \textbf{we can lower-bound training and inference time by using the maximum of computation and communication time}. We can also \textbf{upper-bound with their sum}. In practice, we optimize against the maximum as the algebra is simpler and we can usually come close to this bound by overlapping our communication and computation. If we optimize with the maximum in mind then the lower and upper bounds differ by at most a factor of 2 since $T_\text{math} + T_\text{comms} \leq 2 * \max(T_\text{math}, T_\text{comms})$. We then increase accuracy beyond this by modeling `overlap regions' and overheads, which can be informed by profiling your specific model and target system.

\begin{equation}
T_\text{lower}=\max(T_\text{math}, T_\text{comms})
\end{equation}

\begin{equation}
T_\text{upper} = T_\text{math} + T_\text{comms}
\end{equation}

If we assume we can perfectly overlap communication and computation, when $T_\text{math} > T_\text{comms}$, we see full utilization from our hardware. We call this being ``compute-bound''. When $T_\text{comms} > T_\text{math}$, we tend to be ``communication-bound'' and at least some fraction of our accelerator FLOPs/s is wasted waiting for data to be passed around. One way to tell if an operation will be compute or communication-bound is to look at its ``\emph{arithmetic intensity}'' or ``\emph{operational intensity}''.

\textbf{Definition:} the arithmetic intensity of an algorithm is given by the ratio of the total FLOPs it performs to the number of bytes it needs to communicate—either within a chip or between chips.

\begin{equation}
\text{Arithmetic Intensity} = \frac{\text{Computation FLOPs}}{\text{Communication Bytes}}
\end{equation}

Arithmetic intensity measures the ``FLOPs per byte'' of a given operation. To a first order, when our arithmetic intensity is high, $T_\text{math}$ is large compared to $T_\text{comms}$ and we typically use most of the available FLOPs. When the opposite is true, we spent more time on comms and waste FLOPs. The point where this crossover happens is the ``peak arithmetic intensity'' of our hardware, the ratio of peak accelerator FLOPs/s to accelerator bandwidth.

{\small
\begin{align*}
T_\text{math} > T_\text{comms} \Leftrightarrow \frac{\text{Computation FLOPs}} {\text{Accelerator FLOPs/s}} > \frac{\text{Communication Bytes}}{\text{Bandwidth Bytes/s}} & \\[0.5em]
\Leftrightarrow \frac{\text{Computation FLOPs}}{\text{Communication Bytes}} > \frac{\text{Accelerator FLOPs/s}}{\text{Bandwidth Bytes/s}} & \\[0.5em]
\Leftrightarrow \text{Intensity}(\text{Computation}) > \text{Intensity}(\text{Accelerator}) & \\
\end{align*}
}

The quantity $\text{Intensity}(\text{Accelerator})$ is the arithmetic intensity at which our accelerator achieves its peak FLOPs/s. \textbf{For the TPU v5e MXU, this is about 240 FLOPs/byte}, since the TPU can perform \texttt{1.97e14} FLOPs/s and load \texttt{8.2e11} bytes/s from HBM.\footnote{The MXU is the matrix multiply unit on the TPU. We specify this here because the TPU has other accelerators like the VPU that are responsible for elementwise operations that have a different peak FLOPs/s.} That means if an algorithm has a lower arithmetic intensity than 240 FLOPs/byte, it will be bound by byte loading and thus we won't make good use of our hardware.\footnote{This is only true if the algorithm loads its weights from HBM and runs in the MXU. As we'll discuss in the next section, we can sometimes store parameters in VMEM which has a much higher bandwidth. Many algorithms also run in the VPU, which has different performance characteristics.} Let's look at one such example:

\textbf{\textcolor{blue!60!black}{Example (dot product)}} to compute the dot product of two vectors in bfloat16 precision, \texttt{x • y: bf16[N], bf16[N] → bf16[1]}, we need to load $x$ and $y$ from memory, each of which has $2 * N = 2N$ bytes, perform $N$ multiplications and $N-1$ additions, and write $2$ bytes back into HBM

\begin{align}
\text{Intensity}(\text{dot product}) &= \frac{\text{Total FLOPs}}{\text{Total Bytes}} = \frac{N + N - 1}{2N + 2N + 2} \nonumber \\
&= \frac{2N - 1}{4N + 2} \rightarrow \frac{1}{2}
\end{align}

as $N\rightarrow\infty$. So the dot product has an arithmetic intensity of $\frac{1}{2}$ or, put another way, the dot product does 0.5 floating point operations per byte loaded. This means our arithmetic intensity is lower than that of our hardware and we will be communication-bound.\footnote{The 240 number above is not the correct comparison here since, as you will see in the next section, a dot-product is performed on the VPU and not the MXU. The TPU v5p VPU can do roughly 7e12 FLOPs / second, so its critical intensity is around 3, which means we are still somewhat comms-bound here. Either way, the fact that our intensity is low and constant means it is difficult to be compute-bound on most hardware.}

\subsection*{Visualizing rooflines}
\addcontentsline{toc}{subsection}{Visualizing rooflines}

We can visualize the tradeoff between memory and compute using a \textbf{roofline plot}, which plots the peak achievable FLOPs/s (throughput) of an algorithm on our hardware (the y-axis) against the arithmetic intensity of that algorithm (the x-axis). Here's an example log-log plot:

\begin{figure}[htb]
\centering
\includegraphics[width=\textwidth]{images/roofline-improved.png}
\caption{An example roofline plot showing two algorithms with different arithmetic intensities (Algo 1 and Algo 2) and their corresponding theoretical peak throughput under different bandwidths (BW1 and BW2). In the red area, an algorithm is bandwidth bound at both bandwidths and is wasting some fraction of the hardware's peak FLOPs/s. The yellow area is bandwidth-bound only at the lower bandwidth (BW1). The green area is compute-bound at all bandwidths. Here, we are using the peak FLOPs/s of the accelerator and increasing bandwidth or improving intensity yield no benefit.}
\end{figure}

Above, as the intensity increases (moving left to right), we initially see a linear increase in the performance of our algorithm (in FLOPs/s) until we hit the critical arithmetic intensity of the hardware, 240 in the case of the TPU v5e. Any algorithm with a lower intensity will be bandwidth (BW) bound and limited by the peak memory bandwidth (shown in red). Any algorithm to the right will fully utilize our FLOPs (shown in green). Here, Algo 1 is comms-bound and uses only a fraction of the total hardware FLOPs/s. Algo 2 is compute-bound. We can generally improve the performance of an algorithm either by increasing its arithmetic intensity or by increasing the memory bandwidth available (moving from BW1 to BW2).

\subsection*{Matrix multiplication}
\addcontentsline{toc}{subsection}{Matrix multiplication}

Let's look at our soon-to-be favorite algorithm: matrix multiplication (aka matmul). We write $X * Y \rightarrow Z$ where $X$ has shape $\text{bf16}[B, D]$, $Y$ has shape $\text{bf16}[D, F]$, and $Z$ has shape $\text{bf16}[B, F]$. To do the matmul we need to load $2DF + 2BD$ bytes, perform $2BDF$ FLOPs, and write $2BF$ bytes back.\footnote{Technically we perform $BF \times (2D - 1)$ FLOPs but this is close enough. This comes from $BDF$ multiplications and $BF * (D-1)$ additions. Section 4 has more details.}\footnote{Although the output of a matmul is technically float32 we usually cast down to bfloat16 before copying back to HBM.} Thus:

\begin{equation}
\text{Intensity}(\text{matmul}) = \frac{2BDF}{2BD + 2DF + 2BF} = \frac{BDF}{BD + DF + BF}
\end{equation}

We can get a nice simplification if we assume our ``batch size'' $B$ is small relative to $D$ and $F$. Then we get

\begin{equation}
\frac{BDF}{BD + DF + BF} \approxeq \frac{BDF}{DF} = B
\end{equation}

\begin{equation}
\text{Intensity}(\text{matmul}) > \text{Intensity}(\text{TPU}) \implies B > \frac{1.97e14}{8.20e11} = 240
\end{equation}

This is a reasonable assumption for Transformer matmuls since we typically have a local (per-replica) batch size $B < 1024$ tokens (\emph{not sequences}) but $D$ and $F > 8000$. Thus we generally become compute-bound when our per-replica\footnote{We say per-replica because, if we do some kind of model sharding to increase the number of chips used in the matmul, we scale both our available compute and memory bandwidth by the same amount. Thus the critical batch size is true per independent copy of the model weights.} batch size is greater than 240 tokens, a very simple rule!

\begin{takeawaybox}
For a bfloat16 matmul to be compute-bound on most TPUs, we need our per-replica token batch size to be greater than 240.\footnote{Note that this is \emph{not} the batch size in the usual sense, where it means the batch size in sequences. It turns out most rooflines depend purely on the number of tokens, whether they belong to the same or different sequences. For instance if you have a batch size of 512 sequences of 4096 tokens on 128 GPUs, you have a total batch size of \texttt{512 * 4096 = 2M} tokens, and a local batch size of 16k tokens.}
\end{takeawaybox}

This comes with a few notable caveats we'll explore in the problems below, particularly with respect to quantization (e.g., if we quantize our activations but still do full-precision FLOPs), but it's a good rule to remember. For GPUs, this number is slightly higher (closer to 300), but the same conclusion generally holds. When we decompose a big matmul into smaller matmuls, the tile sizes also matter.\footnote{When we do a large matrix multiplication, we need to break it down into smaller tiles which fit into VMEM/SMEM/TMEM, the higher-bandwidth on-chip memory. This causes us to load chunks multiple times, so it's no longer quite true that we only load $O(N^2)$ bytes. Consider an $(m, k) \cdot (k, n)$ matmul with tile sizes $bm$, $bk$, $bm$. Let $tm = m / bm$, etc. Then the total FLOPs is $2 \cdot tm \cdot tn \cdot tk \cdot bm \cdot bn \cdot bk$ and the total bytes are $2 \cdot tm \cdot tn \cdot (tk \cdot (bm \cdot bk + bk \cdot bn) + 2 \cdot bm \cdot bn)$. Ignoring the last term, we have an intensity of $bm \cdot bn / (bm + bn)$, which is similar to the above.} We'll discuss the lower-level GPU and TPU details in the next section.

\subsection*{Network communication rooflines}
\addcontentsline{toc}{subsection}{Network communication rooflines}

All the rooflines we've discussed so far have been memory-bandwidth rooflines, \emph{all within a single chip}. This shouldn't be taken as a rule. In fact, most of the rooflines we'll care about in this book involve communication between chips: usually matrix multiplications that involve matrices sharded across multiple TPUs.

To pick a somewhat contrived example, say we want to multiply two big matrices $X\sim \text{bfloat16[B, D]}$ and $Y \sim \text{bfloat16[D, F]}$ which are split evenly across 2 TPUs/GPUs (along the $D$ dimension). To do this multiplication (as we'll see in Section 3), we can multiply half of each matrix on each TPU (\texttt{A = X[:, :D // 2] @ Y[:D // 2, :]} on TPU 0 and \texttt{B = X[:, D // 2:] @ Y[D // 2:, :]} on TPU 1) and then copy the resulting ``partial sums'' to the other TPU and add them together. Say we can copy \texttt{4.5e10} bytes in each direction and perform \texttt{1.97e14} FLOPs/s on each chip. What are $T_\text{math}$ and $T_\text{comms}$?

$T_\text{math}$ is clearly half of what it was before, since each TPU is doing half the work, i.e.\footnote{We're ignoring the FLOPs required to add the two partial sums together (another DF additions), but this is basically negligible.}

$$T_\text{math} = \frac{2BDF}{2 \cdot \text{Accelerator FLOPs/s}} = \frac{BDF}{1.97e14}$$

Now what about $T_\text{comms}$? This now refers to the communication time between chips! This is just the total bytes sent divided by the network bandwidth, i.e.

$$T_\text{comms} = \frac{2BF}{\text{Network Bandwidth}} = \frac{2BF}{4.5e10}$$

Therefore we become compute-bound (now with respect to the inter-chip network) when $$\text{Intensity}(\text{matmul (2-chips)}) > \text{Intensity}(\text{TPU w.r.t. inter-chip network})$$ or equivalently when $\frac{BDF}{2BF} = \frac{D}{2} > \frac{1.97e14}{4.5e10} = 4377$ or $D > 8755$. Note that, unlike before, the critical threshold now depends on $D$ and not $B$! Try to think why that is. This is just one such example, but we highlight that this kind of roofline is critical to knowing when we can parallelize an operation across multiple TPUs.

\section*{A Few Problems to Work}
\addcontentsline{toc}{section}{A Few Problems to Work}

\textbf{Question 1 [int8 matmul]:} Say we want to do the matmul $X[B, D] \cdot_D Y[D, F] \rightarrow Z[B, F]$ in int8 precision (1 byte per parameter) instead of bfloat16.\footnote{Here and throughout we'll use the notation $A \cdot_D B$ to indicate that the multiplication is performing a contraction over the D dimension. This is an abuse of einsum notation.}

\begin{enumerate}
\item How many bytes need to be loaded from memory? How many need to be written back to memory?
\item How many total OPs are performed?
\item What is the arithmetic intensity?
\item What is a roofline estimate for $T_\text{math}$ and $T_\text{comms}$? What are reasonable upper and lower bounds for the runtime of the whole operation?
\end{enumerate}

Assume our HBM bandwidth is \texttt{8.1e11} bytes/s and our int8 peak OPs/s is \texttt{3.94e14} (about 2x bfloat16).

\textbf{Question 2 [int8 + bf16 matmul]:} In practice we often do different weight vs. activation quantization, so we might store our weights in very low precision but keep activations (and compute) in a higher precision. Say we want to quantize our weights in int8 but keep activations (and compute) in bfloat16. At what batch size do we become compute bound? Assume \texttt{1.97e14} bfloat16 FLOPs/s.

\emph{Hint: this means specifically \texttt{bfloat16[B, D] * int8[D, F] -> bfloat16[B, F]} where $B$ is the ``batch size''.}

\textbf{Question 3:} Taking the setup from Question 2, make a roofline plot of peak FLOPs/s vs. $B$ for $F = D = 4096$ and $F = D = 1024$. \emph{Use the exact number of bytes loaded, not an approximation.}

\textbf{Question 4:} What if we wanted to perform $\text{int8[B, D]} *_D \text{int8[B, D, F]} \rightarrow \text{int8[B, F]}$ where we imagine having a different matrix for each batch element. What is the arithmetic intensity of this operation?

\textbf{Problem 5 [Memory Rooflines for GPUs]:} Using the spec sheet provided by NVIDIA for the H100, calculate the batch size at which a matrix multiplication will become compute-bound. \emph{Note that the Tensor Core FLOPs numbers are twice the true value since they're only achievable with structured sparsity.}

\chapter{How to Think About TPUs}

\section*{What Is a TPU?}
\addcontentsline{toc}{section}{What Is a TPU?}

\textbf{A TPU is basically a compute core that specializes in matrix multiplication (called a TensorCore) attached to a stack of fast memory (called high-bandwidth memory or HBM).} Here's a diagram:

\begin{figure}[htb]
\centering
\includegraphics[width=\textwidth]{images/tpu-chip.png}
\caption{The basic components of a TPU chip. The TensorCore is the gray left-hand box, containing the matrix-multiply unit (MXU), vector unit (VPU), and vector memory (VMEM).}
\end{figure}

You can think of the TensorCore as basically just being a really good matrix multiplication machine, but it has a few other functions worth noting. The TensorCore has three key units:

\begin{itemize}
\item The \textbf{MXU} (Matrix Multiply Unit) is the core of the TensorCore. For most TPU generations, it performs one \texttt{bfloat16[8,128] @ bf16[128,128] -> f32[8,128]} matrix multiply\footnote{TPU v6e (Trillium) has a 256x256 MXU, while all previous generations use 128x128} every 8 cycles using a systolic array (see Appendix B for details).
  \begin{itemize}
  \item This is about \texttt{5e13} bf16 FLOPs/s per MXU at 1.5GHz on TPU v5e. Most TensorCores have 2 or 4 MXUs, so e.g. the total bf16 FLOPs/s for TPU v5e is \texttt{2e14}.
  \item TPUs also support lower precision matmuls with higher throughput (e.g. each TPU v5e chip can do \texttt{4e14} int8 OPs/s).
  \end{itemize}

\item The \textbf{VPU} (Vector Processing Unit) performs general mathematical operations like ReLU activations or pointwise addition or multiplication between vectors. Reductions (sums) are also performed here. Appendix A provides more details.

\item \textbf{VMEM} (Vector Memory) is an on-chip scratchpad located in the TensorCore, close to the compute units. It is much smaller than HBM (for example, 128 MiB on TPU v5e) but has a much higher bandwidth to the MXU. VMEM operates somewhat like an L1/L2 cache on CPUs but is much larger and programmer-controlled. Data in HBM needs to be copied into VMEM before the TensorCore can do any computation with it.
\end{itemize}

\textbf{TPUs are very, very fast at matrix multiplication}. It's mainly what they do and they do it well. TPU v5p, one of the most powerful TPUs to date, can do \texttt{2.5e14} bf16 FLOPs / second / core or \texttt{5e14} bf16 FLOPs / sec / chip. A single pod of 8960 chips can do 4 exaflops / second. That's \emph{a lot}. That's one of the most powerful supercomputers in the world. And Google has a lot of them.\footnote{TPUs, and their systolic arrays in particular, are such powerful hardware accelerators because matrix multiplication is one of the few algorithms that uses $O(n^3)$ compute for $O(n^2)$ bytes. That makes it very easy for an ordinary ALU to be bottlenecked by compute and not by memory bandwidth.}

The diagram above also includes a few other components like SMEM and the scalar unit, which are used for control flow handling and are discussed briefly in Appendix A, but aren't crucial to understand. On the other hand, HBM is important and fairly simple:

\begin{itemize}
\item \textbf{HBM} (High Bandwidth Memory) is a big chunk of fast memory that stores tensors for use by the TensorCore. HBM usually has capacity on the order of tens of gigabytes (for example, TPU v5e has 16GiB of HBM).

  \begin{itemize}
  \item When needed for a computation, tensors are streamed out of HBM through VMEM (see below) into the MXU and the result is written from VMEM back to HBM.

  \item The bandwidth between HBM and the TensorCore (through VMEM) is known as ``HBM bandwidth'' (usually around 1-2TB/sec) and limits how fast computation can be done in memory-bound workloads.
  \end{itemize}
\end{itemize}

\textbf{Generally, all TPU operations are pipelined and overlapped.} To perform a matmul $X \cdot A \to Y$, a TPU would first need to copy chunks of matrices $A$ and $X$ from HBM into VMEM, then load them into the MXU which multiplies chunks of 8x128 (for $X$) and 128x128 (for $A$), then copy the result chunk by chunk back to HBM. To do this efficiently, the matmul is pipelined so the copies to/from VMEM are overlapped with the MXU work. This allows the MXU to continue working instead of waiting on memory transfers, keeping matmuls compute-bound, not memory-bound.

A matmul would look nearly identical except it would load into the MXU instead of the VPU/Vector unit, and the loads and stores would occur in a different order, since the same weight chunk is used for multiple chunks of activations. You can see chunks of data streaming into VMEM, then into the VREGs (vector registers), then into the Vector Unit, then back into VMEM and HBM. As we're about to see, if the load from HBM to VMEM is slower than the FLOPs in the Vector Unit (or MXU), we become ``bandwidth bound'' since we're starving the VPU or MXU of work.

\begin{takeawaybox}
TPUs are very simple. They load weights from HBM into VMEM, then from VMEM into a systolic array which can perform around 200 trillion multiply-adds per second. The HBM $\leftrightarrow$ VMEM and VMEM $\leftrightarrow$ systolic array bandwidths set fundamental limits on what computations TPUs can do efficiently.
\end{takeawaybox}

\textbf{VMEM and arithmetic intensity:} VMEM is much smaller than HBM but it has a much higher bandwidth to the MXU. As we saw in Section 1, this means if an algorithm can fit all its inputs/outputs in VMEM, it's much less likely to hit communication bottlenecks. This is particularly helpful when a computation has poor arithmetic intensity: VMEM bandwidth is around 22x higher than HBM bandwidth which means an MXU operation reading from/writing to VMEM requires an arithmetic intensity of only 10-20 to achieve peak FLOPs utilization. That means if we can fit our weights into VMEM instead of HBM, our matrix multiplications can be FLOPs bound at much smaller batch sizes. And it means algorithms that fundamentally have a lower arithmetic intensity can still be efficient. VMEM is just so small this is often a challenge.\footnote{We sometimes talk about VMEM prefetching, which refers to loading weights ahead of time in VMEM so we can mask the cost of loading for our matmuls. For instance, in a normal Transformer we can sometimes load our big feed-forward weights into VMEM during attention, which can hide the cost of the weight load if we're memory bandwidth bound. This requires our weights to be small enough or sharded enough to fit a single layer into VMEM with space to spare.}

\begin{figure}[htb]
\centering
\includegraphics[width=\textwidth]{images/tpu-bandwidth.png}
\end{figure}

\textbf{A TPU chip typically (but not always) consists of two TPU cores which share memory and can be thought of as one large accelerator} with twice the FLOPs (known as a ``megacore'' configuration). This has been true since TPU v4. Older TPU chips have separate memory and are regarded as two separate accelerators (TPU v3 and older). Inference-optimized chips like the TPU v5e only have one TPU core per chip.

\begin{figure}[htb]
\centering
\includegraphics[width=\textwidth]{images/cores.png}
\end{figure}

\textbf{Chips} are arranged in \textbf{sets of 4 on a `tray'} connected to a \textbf{CPU host via PCIe network.} This is the format most readers will be familiar with, 4 chips (8 cores, though usually treated as 4 logical megacores) exposed through Colab or a single TPU-VM. For inference chips like the TPU v5e, we have 2 trays per host, instead of 1, but also only 1 core per chip, giving us 8 chips = 8 cores.\footnote{On Cloud TPU VMs, each tray is exposed as part of a separate VM, so there are once again 4 cores visible.}

\begin{figure}[htb]
\centering
\includegraphics[width=\textwidth]{images/pcie.png}
\end{figure}

\textbf{PCIe bandwidth is limited:} Like the HBM $\leftrightarrow$ VMEM link, the CPU $\leftrightarrow$ HBM PCIe connection has a specific bandwidth that limits how quickly you can load from host memory to HBM or vice-versa. PCIe bandwidth for TPU v4 is 16GB / second each way, for example, so close to 100x slower than HBM. We \emph{can} load/offload data into the host (CPU) RAM, but not very quickly.

\section*{TPU Networking}
\addcontentsline{toc}{section}{TPU Networking}

\textbf{Chips are connected to each other through the ICI network in a Pod}. In older generations (TPU v2 and TPU v3), inference chips (e.g., TPU v5e), and Trillium (TPU v6e), ICI (``inter-chip interconnects'') connects the 4 nearest neighbors (with edge links to form a 2D torus). TPU v4 and TPU v5p are connected to the nearest 6 neighbors (forming a 3D torus). Note these connections do \textbf{not} go through their hosts, they are direct links between chips.

\begin{figure}[htb]
\centering
\includegraphics[width=\textwidth]{images/ici-wraparound.png}
\end{figure}

The toroidal structure reduces the maximum distance between any two nodes from $N$ to $N / 2$, making communication much faster. TPUs also have a ``twisted torus'' configuration that wraps the torus in a Mobius-strip like topology to further reduce the average distance between nodes.

\textbf{TPU pods (connected by ICI) can get really big:} the maximum pod size (called a \textbf{superpod}) is \texttt{16x16x16} for TPU v4 and \texttt{16x20x28} for TPU v5p. These large pods are composed of reconfigurable cubes of \texttt{4x4x4} chips connected by optical wraparound links\footnote{The optical switch is simply a reconfigurable connection with the same ICI bandwidth. It just lets us connect cubes while retaining a wraparound link.} that we can reconfigure to connect very large topologies.

\begin{figure}[htb]
\centering
\includegraphics[width=\textwidth]{images/tpu-rack.png}
\end{figure}

Smaller topologies (e.g. \texttt{2x2x1}, \texttt{2x2x2}) can also be requested, albeit with no wraparounds. This is an important caveat, since it typically doubles the time of most communication. Any multiple of a full cube (e.g. \texttt{4x4x4} or \texttt{4x4x8}) will have wraparounds provided by the optical switches.\footnote{Note that a \texttt{2x2x4} won't have any wraparounds since they are provided by the optical switches which are only available on a full cube. A TPU v5e 8x16 \emph{will} have a wraparound on the longer axis, however, since it doesn't use reconfigurable optical networking.}

\begin{figure}[htb]
\centering
\includegraphics[width=\textwidth]{images/subslices.png}
\end{figure}

TPU v5e and Trillium pods consist of a single \texttt{16x16} 2D torus with wraparounds along any axis of size 16 (meaning an \texttt{8x16} has a wraparound on the long axis). TPUs v5e and v6e (Trillium) cannot expand beyond a 16x16 torus but pods can still communicate with each other over standard data-center networking (DCN), which connects TPU hosts to each other. Again, smaller topologies can be requested without wraps on dims $<16$.

\begin{figure}[htb]
\centering
\includegraphics[width=\textwidth]{images/more-subslices.png}
\end{figure}

\textbf{This nearest-neighbor connectivity is a key difference between TPUs and GPUs}. GPUs are connected with a hierarchy of switches that approximate a point-to-point connection between every GPU, rather than using local connections like a TPU. Typically, GPUs within a node (8 GPUs for H100 or as many as 72 for B200 NVL72) are directly connected, while larger topologies require O(log(N)) hops between each GPU. On the one hand, that means GPUs can send arbitrary data within a small number of hops. On the other hand, TPUs are dramatically cheaper (since NVLink switches are expensive), simpler to wire together, and can scale to much larger topologies because the number of links per device and the bandwidth per device is constant. Read more here.

\textbf{ICI is very fast relative to DCN, but is still slower than HBM bandwidth.} For instance, a TPU v5p has:

\begin{itemize}
\item \texttt{2.5e12} bytes/s (2.5 TB/s) of HBM bandwidth per chip.
\item \texttt{9e10} bytes/s (90 GB/s) of ICI bandwidth per axis, with 3 axes per chip.\footnote{The page above lists 100 GB/s of bandwidth, which is slightly different from what's listed here. TPU ICI links have slightly different bandwidths depending on the operation being performed. You can generally use the numbers in this doc without worry.}
\item \texttt{6.25e9} bytes/s (6.25 GB/s) of DCN (egress) bandwidth per TPU (via 1-2 NICs on each host).\footnote{TPU v6e has 12.5e9 bytes/s and v5e has 3.125e9 bytes/s.}
\end{itemize}

This means that when we split models across multiple chips, we need to be careful to avoid bottlenecking the MXU with slower cross-device communication.

\textbf{Multi-slice training:} A set of ICI-connected TPUs is called a \textbf{slice}. Different slices can be connected between each other using DCN, for instance to link slices on different pods. Since DCN is a much slower connection than ICI, one should try to limit how much our computation has to wait for data from DCN. DCN is host-to-host, so to transfer buffers from TPU to TPU over DCN, we first need to transfer over PCIe to the host, then egress over the network, then ingress over the target host network, then over PCIe into HBM.

\section*{Key Takeaways}
\addcontentsline{toc}{section}{Key Takeaways}

\begin{itemize}
\item TPUs are simple and can in most cases be thought of as a matrix multiply unit connected to memory (super fast), other chips over ICI (rather fast), and the rest of the datacenter over DCN (somewhat fast).

\item Communication is limited by our various network bandwidths in order of speed:
  \begin{itemize}
  \item HBM bandwidth: Between a TensorCore and its associated HBM.
  \item ICI bandwidth: Between a TPU chip and its nearest 4 or 6 neighbors.
  \item PCIe bandwidth: Between a CPU host and its associated tray(s) of chips.
  \item DCN bandwidth: Between multiple CPU hosts, typically hosts not connected by ICI.
  \end{itemize}

\item \textbf{Within a slice, TPUs are only connected to their nearest neighbors via ICI.} This means communication over ICI between distant chips in a slice needs to hop over the intervening chips first.

\item \textbf{Weight matrices need to be padded to at least size 128} (256 on TPU v6) in both dimensions to fill up the MXU (in fact, smaller axes are padded to 128).

\item \textbf{Lower precision matrix multiplication tends to be faster.} TPUs can do int8 or int4 FLOPs roughly 2x/4x faster than bfloat16 FLOPs for generations that support it. VPU operations are still performed in fp32.

\item To avoid bottlenecking the TPU compute unit, we need to \textbf{make sure the amount of communication across each channel is proportional to its speed}.
\end{itemize}

\subsection*{TPU Specs}
\addcontentsline{toc}{subsection}{TPU Specs}

Here are some specific numbers for our chips:

{\scriptsize
\begin{longtable}{lccccc}
\toprule
\textbf{Model} & \textbf{\shortstack{Pod \\ size}} & \textbf{\shortstack{Host \\ size}} & \textbf{\shortstack{HBM capacity \\ /chip}} & \textbf{\shortstack{HBM BW/chip \\ (bytes/s)}} & \textbf{\shortstack{FLOPs/s/chip \\ (bf16)}} \\
\midrule
TPU v3 & 32x32 & 4x2 & 32GB & 9.0e11 & 1.4e14 \\
TPU v4p & 16x16x16 & 2x2x1 & 32GB & 1.2e12 & 2.75e14 \\
TPU v5p & 16x20x28 & 2x2x1 & 96GB & 2.8e12 & 4.59e14 \\
TPU v5e & 16x16 & 4x2 & 16GB & 8.1e11 & 1.97e14 \\
TPU v6e & 16x16 & 4x2 & 32GB & 1.6e12 & 9.20e14 \\
\bottomrule
\end{longtable}
}

Host size refers to the topology of TPUs connected to a single host (e.g. TPU v5e has a single CPU host connected to 8 TPUs in a 4x2 topology). Here are interconnect figures:

{\scriptsize
\begin{longtable}{lcc}
\toprule
\textbf{Model} & \textbf{ICI BW/link (one-way, bytes/s)} & \textbf{ICI BW/link (bidi, bytes/s)} \\
\midrule
TPU v3 & 1e11 & 2e11 \\
TPU v4p & 4.5e10 & 9e10 \\
TPU v5p & 9e10 & 1.8e11 \\
TPU v5e & 4.5e10 & 9e10 \\
TPU v6e & 9e10 & 1.8e11 \\
\bottomrule
\end{longtable}
}

We include both one-way (unidirectional) bandwidth and bidi (bidirectional) bandwidth since unidirectional bandwidth is more true to the hardware but bidirectional bandwidth occurs more often in equations involving a full ring.\footnote{By bidi (bidirectional) bandwidth we mean the total bytes that can be sent along a single link in both directions, or equally, the total number of outgoing bytes from a single TPU along a particular axis, assuming we can use both links efficiently. This is true when we have a functioning ring, AKA when we have a wraparound connection on the particular axis. This occurs on inference chips when we have a full 16 axis, or on training chips (v*p) when we have an axis which is a multiple of 4. We prefer to use the bidirectional bandwidth because it appears frequently in calculations involving bidirectional comms.}

PCIe bandwidth is typically around \texttt{1.6e10} bytes / second per TPU (\texttt{3.2e10} for TPU v6e), while DCN bandwidth is typically around \texttt{6.25e9} bytes / second per TPU (\texttt{12.5e9} for TPU v6e and \texttt{3.125e9} for TPU v5e).

\section*{Worked Problems}
\addcontentsline{toc}{section}{Worked Problems}

These numbers are a little dry, but they let you make basic roofline estimates for model performance. Let's work a few problems to explain why this is useful. You'll see more examples in Part 3.

\textbf{Question 1 [bounding LLM latency]:} Say you want to sample from a 200B parameter model in bf16 that's split across 32 TPU v4p. How long would it take to load all the parameters from HBM into the systolic array? \emph{Hint: use the numbers above.}

\textbf{Question 2 [TPU details]:} Consider a full TPU v5e pod. How many total CPU hosts are there? How many TPU TensorCores? What is the total FLOPs/s for the whole pod? What is the total HBM? Do the same exercise for TPU v5p pod.

\textbf{Question 3 [PCIe operational intensity]:} Imagine we're forced to store a big weight matrix $A$ of type $\text{bfloat16}[D, F]$, and a batch of activations $x$ of type $\text{bfloat16}[B, D]$ in host DRAM and want to do a matrix multiplication on them. This is running on a single host, and we're using a single TPU v6e chip attached to it. You can assume $B \ll D$, and $F = 4D$ (we'll see in future chapters why these are reasonable assumptions). What is the smallest batch size $B$ we need to remain FLOPs bound over PCIe? Assume PCIe bandwidth of 1.5e10 bytes / second.

\textbf{Question 4 [general matmul latency]:} Let's say we want to multiply a weight matrix int8[16384, 4096] by an activation matrix of size int8[B, 4096] where B is some unknown batch size. Let's say we're on 1 TPU v5e to start.

\begin{enumerate}
\item How long will this multiplication take as a function of B? \emph{Hint: it may help to calculate how long it will take to load the arrays from HBM and how long the multiplication will actually take. Which is bottlenecking you?}
\item What if we wanted to run this operation out of VMEM? How long would it take as a function of B?
\end{enumerate}

\textbf{Question 5 [ICI bandwidth]:} Let's say we have a TPU v5e \texttt{4x4} slice. Let's say we want to send an array of type \texttt{bfloat16[8, 128, 8192]} from \texttt{TPU\{0,0\}} to \texttt{TPU\{3, 3\}}. Let's say the per-hop latency for TPU v5e is $1\mu s$.

\begin{enumerate}
\item How soon will the first byte arrive at its destination?
\item How long will the total transfer take?
\end{enumerate}

\textbf{Question 6 [pulling it all together, hard]:} Imagine you have a big matrix \textbf{A}: \texttt{int8[128 * 1024, 128 * 1024]} sharded evenly across a TPU v5e 4x4 slice but offloaded to host DRAM on each chip. Let's say you want to copy the entire array to TPU\{0, 0\} and multiply it by a vector \texttt{bf16[8, 128 * 1024]}. How long will this take? \emph{Hint: use the numbers above.}

\section*{Appendix}
\addcontentsline{toc}{section}{Appendix}

\subsection*{Appendix A: More on TPU internals}
\addcontentsline{toc}{subsection}{Appendix A: More on TPU internals}

Here we'll dive more deeply into the internal operations of a TPU. Unless otherwise noted, we'll provide specs for a TPU v5p.

\subsubsection*{VPU}

The VPU is the TPU's vector arithmetic core. The VPU consists of a two dimensional SIMD vector machine (the \textbf{VPU}) that performs elementwise arithmetic operations like vadd (vector addition) or vmax (elementwise max) and a set of vector registers called \textbf{VREGs} that hold data for the VPU and MXU.

\textbf{VREGs:} Each TPU v5p core has 64 32-bit VREGs (32 in TPU v4), giving us a total of about \texttt{64 * 8 * 128 * 4 = 256kB} of VREG memory per core (or 2x this for the whole chip since we have two cores). A TPU v5p can load 3 registers from VMEM each cycle, and write 1 register to VMEM each cycle.

\textbf{VPU:} The VPU is a 2D vector arithmetic unit of shape \texttt{(8, 128)} where the 128 dimension is referred to as lane axis and the dimension of 8 is referred to as the sublane axis. Each (lane, sublane) pair on v5 contains 4 standard floating-point ALUs which are independent of each other. The VPU executes most arithmetic instructions in one cycle in each of its ALUs (like vadd or vector add) with a latency of 2 cycles, so e.g. in v5 you can add 4 pairs of f32 values together from VREGs in each cycle. A typical VPU instruction might look like \texttt{\{v2 = vadd.8x128.f32 v0, v1\}} where v0 and v1 are input VREGs and v2 is an output VREG.

All lanes and sublanes execute the same program every cycle in a pure SIMD manner, but each ALU can perform a different operation. So we can e.g. process 1 vadd and 1 vsub in a single cycle, each of which operates on two full VREGs and writes the output to a third.

\textbf{Pop Quiz [Calculating VPU throughput]:} Using the above information, calculate how many vector FLOPs/s a TPU v5p can perform. A TPU v5p has a clock speed of about 1.75GHz.

\textbf{Reductions:} Generally, communication or reduction across the sublane dimension is easier than across the lane dimension. For instance, the VPU supports an intra-lane shuffle operation that can roll along the axis of size 8 in about a cycle. This can be used to perform efficient reductions along the sublane dimension (just shuffle by 4, 2, and 1 and do 3 pairs of elementwise sums).

Cross-lane reductions are much harder and involve a separate hardware unit called the XLU or ``cross lane unit'', which is slow and fairly expensive.

\textbf{Comparison to GPUs:} For those familiar with NVIDIA GPUs, each ALU in the VPU is analogous to a CUDA core, and a single VPU lane is analogous to a ``Warp Scheduler'', i.e. the set of usually 32 CUDA Cores that perform SIMD arithmetic. Reductions within the lane are pretty easy, but if we need to cross lanes, we need to transit at least VMEM/XLU/SMEM which is much slower. See the GPU section for more details.

\subsubsection*{Scalar Core}

The scalar core is the control unit of the TPU. It fetches and dispatches all instructions and executes transfers from HBM into VMEM, and can be programmed to do scalar metadata work. Because the scalar core is single-threaded, one side-effect of this is that each core of the TPU is only capable of creating one DMA request per cycle.

To put this in context, a single scalar core controls a VPU (consisting of 4096 ALUs), 4 MXUs, 2 XLUs, and multiple DMA engines. The highly skewed nature of control per unit compute is a source of hardware efficiency, but also limits the ability to do data dependent vectorization in any interesting way.
\chapter{Sharded Matrices and How to Multiply Them}
\label{chap:sharding}

% Chapter 3: Sharding
% This chapter is divided into multiple files for easier management

% Section 1: Partitioning Notation and Collective Operations
% Lines 89-184 from sharding.md

\section*{Partitioning Notation and Collective Operations}
\addcontentsline{toc}{section}{Partitioning Notation and Collective Operations}

When we train an LLM on ten thousand TPUs or GPUs, we're still doing abstractly the same computation as when we're training on one. The difference is that \textbf{our arrays don't fit in the HBM of a single TPU/GPU}, so we have to split them.\footnote{It's worth noting that we may also choose to parallelize for speed. Even if we could fit on a smaller number of chips, scaling to more simply gives us more FLOPs/s. During inference, for instance, we can sometimes fit on smaller topologies but choose to scale to larger ones in order to reduce latency. Likewise, during training we often scale to more chips to reduce the step time.} We call this ``\emph{sharding}'' or ``\emph{partitioning}'' our arrays. The art of scaling is figuring out how to shard our models so computation remains efficient.

Here's an example 2D array \textbf{A} sharded across 4 TPUs:

\begin{figure}[htb]
\centering
\includegraphics[width=\textwidth]{images/sharding-example.png}
\caption{An example array of shape \textbf{A}[I, J] gets sharded across 4 devices. Both dimensions are evenly sharded across 2 devices with a sharding \textbf{A}[I$_X$, J$_Y$]. Each TPU holds 1/4 of the total memory.}
\end{figure}

Note how the sharded array still has the same \emph{global} or \emph{logical shape} as unsharded array, say \texttt{(4, 128)}, but it also has a \emph{device local shape}, like \texttt{(2, 64)}, which gives us the actual size in bytes that each TPU is holding (in the figure above, each TPU holds ¼ of the total array). Now we'll generalize this to arbitrary arrays.

\subsection*{A unified notation for sharding}
\addcontentsline{toc}{subsection}{A unified notation for sharding}

We use a variant of \emph{named-axis notation} to describe \emph{how} the tensor is sharded in blocks across the devices: we assume the existence of a 2D or 3D grid of devices called the \textbf{device mesh} where each axis has been given \textbf{mesh axis names e.g. X, Y, and Z.} We can then specify how the matrix data is laid out across the device mesh by describing how each named dimension of the array is partitioned across the physical mesh axes. We call this assignment a \textbf{sharding}.

\textbf{Example (the diagram above)}: For the above diagram, we have:
\begin{itemize}
\item \textbf{Mesh:} the device mesh above \texttt{Mesh(devices=((0, 1), (2, 3)), axis\_names=('X', 'Y'))}, which tells us we have 4 TPUs in a 2x2 grid, with axis names $X$ and $Y$.
\item \textbf{Sharding:} $A[I_X, J_Y]$, which tells us to shard the first axis, $I$, along the mesh axis $X$, and the second axis, $J$, along the mesh axis $Y$. This sharding tells us that each shard holds $1 / (\lvert X\rvert \cdot \lvert Y\rvert)$ of the array.
\end{itemize}

Taken together, we know that the local shape of the array (the size of the shard that an individual device holds) is $(\lvert I\rvert / 2, \lvert J\rvert / 2)$, where $\lvert I\rvert$ is the size of A's first dimension and $\lvert J\rvert$ is the size of A's second dimension.

\textbf{\textcolor{blue!60!black}{Pop Quiz [2D sharding across 1 axis]:}} Consider an array \texttt{fp32[1024, 4096]} with sharding $A[I_{XY}, J]$ and mesh \texttt{\{'X': 8, 'Y': 2\}}. How much data is held by each device? How much time would it take to load this array from HBM on H100s (assuming \texttt{3.4e12} memory bandwidth per chip)?

\textbf{Visualizing these shardings:} Let's try to visualize these shardings by looking at a 2D array of data split over 4 devices:

\begin{figure}[htb]
\centering
\includegraphics[width=\textwidth]{images/sharding-colored1.png}
\end{figure}

We write the \emph{fully-replicated} form of the matrix simply as $A[I, J]$ with no sharding assignment. This means that \emph{each} device contains a full copy of the entire matrix.

\begin{figure}[htb]
\centering
\includegraphics[width=\textwidth]{images/sharding-colored2.png}
\end{figure}

We can indicate that one of these dimensions has been partitioned across a mesh axis with a subscript mesh axis. For instance $A[I_X, J]$ would mean that the \textbf{I} logical axis has been partitioned across the \textbf{X} mesh dimension, but that the \textbf{J} dimension is \emph{not} partitioned, and the blocks remain \emph{partially-replicated} across the \textbf{Y} mesh axis.

\begin{figure}[htb]
\centering
\includegraphics[width=\textwidth]{images/sharding-colored3.png}
\end{figure}

$A[I_X, J_Y]$ means that the \textbf{I} logical axis has been partitioned across the \textbf{X} mesh axis, and that the \textbf{J} dimension has been partitioned across the \textbf{Y} mesh axis.

\begin{figure}[htb]
\centering
\includegraphics[width=\textwidth]{images/sharding-colored4.png}
\end{figure}

We illustrate the other possibilities in the figure below:

\begin{figure}[htb]
\centering
\includegraphics[width=\textwidth]{images/sharding-colored5.png}
\end{figure}

Here $A[I_{XY}, J]$ means that we treat the \textbf{X} and \textbf{Y} mesh axes as a larger flattened dimension and partition the \textbf{I} named axis across all the devices. The order of the multiple mesh-axis subscripts matters, as it specifies the traversal order of the partitioning across the grid.

\begin{figure}[htb]
\centering
\includegraphics[width=\textwidth]{images/sharding-colored6.png}
\end{figure}

Lastly, note that we \emph{cannot} have multiple named axes sharded along the \emph{same} mesh dimension. e.g. $A[I_X, J_X]$ is a nonsensical, forbidden sharding. Once a mesh dimension has been used to shard one dimension of an array, it is in a sense ``spent''.

\textbf{\textcolor{green!60!black}{Pop Quiz:}} Let \textbf{A} be an array with shape \texttt{int8[128, 2048]}, sharding $A[I_{XY}, J]$, and mesh \texttt{Mesh(\{'X': 2, 'Y': 8, 'Z': 2\})} (so 32 devices total). How much memory does \textbf{A} use per device? How much total memory does \textbf{A} use across all devices?

\subsection*{How do we describe this in code?}
\addcontentsline{toc}{subsection}{How do we describe this in code?}

So far we've avoided talking about code, but now is a good chance for a sneak peek. JAX uses a named sharding syntax that very closely matches the abstract syntax we describe above. We'll talk more about this in Section 10, but here's a quick preview. You can play with this in a Google Colab and profile the result to see how JAX handles different shardings. This snippet does 3 things:

\begin{enumerate}
\item Creates a \textbf{jax.Mesh} that maps our 8 TPUs into a 4x2 grid with names `X' and `Y' assigned to the two axes.
\item Creates matrices A and B where A is sharded along both its dimensions and B is sharded along the output dimension.
\item Compiles and performs a simple matrix multiplication that returns a sharded array.
\end{enumerate}

\begin{Shaded}
\begin{Highlighting}[]
\ImportTok{import}\NormalTok{ jax}
\ImportTok{import}\NormalTok{ jax.numpy }\ImportTok{as}\NormalTok{ jnp}

\CommentTok{\# Create mesh: TPU v2{-}8 in 4x2 grid}
\ControlFlowTok{assert} \BuiltInTok{len}\NormalTok{(jax.devices()) }\OperatorTok{==} \DecValTok{8}
\NormalTok{mesh }\OperatorTok{=}\NormalTok{ jax.make\_mesh(}
\NormalTok{    axis\_shapes}\OperatorTok{=}\NormalTok{(}\DecValTok{4}\NormalTok{, }\DecValTok{2}\NormalTok{),}
\NormalTok{    axis\_names}\OperatorTok{=}\NormalTok{(}\StringTok{\textquotesingle{}X\textquotesingle{}}\NormalTok{, }\StringTok{\textquotesingle{}Y\textquotesingle{}}\NormalTok{))}

\CommentTok{\# Helper to define sharding}
\KeywordTok{def} \FunctionTok{P}\NormalTok{(}\OperatorTok{*}\NormalTok{args):}
  \ControlFlowTok{return}\NormalTok{ jax.NamedSharding(}
\NormalTok{      mesh,}
\NormalTok{      jax.sharding.PartitionSpec(}\OperatorTok{*}\NormalTok{args))}

\CommentTok{\# Shard A over both dims, B over output}
\NormalTok{A }\OperatorTok{=}\NormalTok{ jnp.zeros(}
\NormalTok{    (}\DecValTok{8}\NormalTok{, }\DecValTok{2048}\NormalTok{),}
\NormalTok{    dtype}\OperatorTok{=}\NormalTok{jnp.bfloat16,}
\NormalTok{    device}\OperatorTok{=}\FunctionTok{P}\NormalTok{(}\StringTok{\textquotesingle{}X\textquotesingle{}}\NormalTok{, }\StringTok{\textquotesingle{}Y\textquotesingle{}}\NormalTok{))}

\NormalTok{B }\OperatorTok{=}\NormalTok{ jnp.zeros(}
\NormalTok{    (}\DecValTok{2048}\NormalTok{, }\DecValTok{8192}\NormalTok{),}
\NormalTok{    dtype}\OperatorTok{=}\NormalTok{jnp.bfloat16,}
\NormalTok{    device}\OperatorTok{=}\FunctionTok{P}\NormalTok{(}\VariableTok{None}\NormalTok{, }\StringTok{\textquotesingle{}Y\textquotesingle{}}\NormalTok{))}

\CommentTok{\# Matmul on sharded arrays}
\NormalTok{y }\OperatorTok{=}\NormalTok{ jax.jit(}
    \KeywordTok{lambda}\NormalTok{ A, B: jnp.einsum(}
\NormalTok{        }\StringTok{\textquotesingle{}BD,DF{-}\textgreater{}BF\textquotesingle{}}\NormalTok{, A, B),}
\NormalTok{    out\_shardings}\OperatorTok{=}\FunctionTok{P}\NormalTok{(}\StringTok{\textquotesingle{}X\textquotesingle{}}\NormalTok{, }\StringTok{\textquotesingle{}Y\textquotesingle{}}\NormalTok{))(A, B)}
\end{Highlighting}
\end{Shaded}

The cool thing about JAX is that these arrays behave as if they're unsharded! \texttt{B.shape} will tell us the global or logical shape (2048, 8192). We have to actually look at \texttt{B.addressable\_shards} to see how it's locally sharded. We can perform operations on these arrays and JAX will attempt to figure out how to broadcast or reshape them to perform the operations. For instance, in the above example, the local shape of \textbf{A} is \texttt{[2, 1024]} and for \textbf{B} is \texttt{[2048, 4096]}. JAX/XLA will automatically add communication across these arrays as necessary to perform the final multiplication.

\section*{Computation With Sharded Arrays}
\addcontentsline{toc}{section}{Computation With Sharded Arrays}

If you have an array of data that's distributed across many devices and wish to perform mathematical operations on it, what are the overheads associated with sharding both the data and the computation?

Obviously, this depends on the computation involved.

\begin{itemize}
\item For \emph{elementwise} operations, there is \textbf{no overhead} for operating on a distributed array.
\item When we wish to perform operations across elements resident on many devices, things get complicated. Thankfully, for most machine learning nearly all computation takes place in the form of matrix multiplications, and they are relatively simple to analyze.
\end{itemize}

The rest of this section will deal with how to multiply sharded matrices. To a first approximation, this involves moving chunks of a matrix around so you can fully multiply or sum each chunk. \textbf{Each sharding will involve different communication.} For example, $A[I_X, J] \cdot B[J, K_Y] \to C[I_X, K_Y]$ can be multiplied without any communication because the \emph{contracting dimension} (J, the one we're actually summing over) is unsharded. However, if we wanted the output unsharded (i.e. $A[I_X, J] \cdot B[J, K_Y] \to C[I, K]$), we would need to copy $A$ or $C$ to every device (using an \emph{AllGather}). These two choices have different communication costs, so we need to calculate this cost and pick the lowest one.

\paragraph{You can think of this in terms of ``block matrix multiplication''.}

To understand this, it can be helpful to recall the concept of a ``block matrix'', or a nested matrix of matrices:

\begin{equation}
\small
\begin{pmatrix}
a_{00} & a_{01} & a_{02} & a_{03} \\
a_{10} & a_{11} & a_{12} & a_{13} \\
a_{20} & a_{21} & a_{22} & a_{23} \\
a_{30} & a_{31} & a_{32} & a_{33}
\end{pmatrix}
=
\left(
\begin{matrix}
\begin{bmatrix}
a_{00} & a_{01} \\
a_{10} & a_{11}
\end{bmatrix} \\
\begin{bmatrix}
a_{20} & a_{21} \\
a_{30} & a_{31}
\end{bmatrix}
\end{matrix}
\begin{matrix}
\begin{bmatrix}
a_{02} & a_{03} \\
a_{12} & a_{13}
\end{bmatrix} \\
\begin{bmatrix}
a_{22} & a_{23} \\
a_{32} & a_{33}
\end{bmatrix}
\end{matrix}
\right)
=
\begin{pmatrix}
\mathbf{A_{00}} & \mathbf{A_{01}} \\
\mathbf{A_{10}} & \mathbf{A_{11}}
\end{pmatrix}
\end{equation}

Matrix multiplication has the nice property that when the matrix multiplicands are written in terms of blocks, the product can be written in terms of block matmuls following the standard rule:

\begin{align}
\small
\begin{pmatrix}
A_{00} & A_{01} \\
A_{10} & A_{11}
\end{pmatrix}
&\cdot
\begin{pmatrix}
B_{00} & B_{01} \\
B_{10} & B_{11}
\end{pmatrix} \nonumber \\
&=
\begin{pmatrix}
A_{00}B_{00} + A_{01}B_{10} & A_{00}B_{01} + A_{01}B_{11} \\
A_{10}B_{00} + A_{11}B_{10} & A_{10}B_{01} + A_{11}B_{11}
\end{pmatrix}
\end{align}

What this means is that implementing distributed matrix multiplications reduces down to moving these sharded blocks over the network, performing \emph{local} matrix multiplications on the blocks, and summing their results. \textbf{The question then is what communication to add, and how expensive it is.}

Conveniently, we can boil down all possible shardings into roughly 4 cases we need to consider, each of which has a rule for what communication we need to add
\begin{enumerate}
\item \textbf{Case 1:} neither input is sharded along the contracting dimension. \emph{We can multiply local shards without any communication.}
\item \textbf{Case 2:} one input has a sharded contracting dimension. \emph{We typically ``AllGather'' the sharded input along the contracting dimension.}
\item \textbf{Case 3:} both inputs are sharded along the contracting dimension. \emph{We can multiply the local shards, then ``AllReduce'' the result.}
\item \textbf{Case 4:} both inputs have a non-contracting dimension sharded along the same axis. We cannot proceed without AllGathering one of the two inputs first.
\end{enumerate}

You can think of these as rules that simply need to be followed, but it's also valuable to understand why these rules hold and how expensive they are. We'll go through each one of these in detail now.

\subsection*{Case 1: neither multiplicand has a sharded contracting dimension}
\addcontentsline{toc}{subsection}{Case 1: neither multiplicand has a sharded contracting dimension}

\textbf{Lemma:} when multiplying sharded matrices, the computation is valid and the output follows the sharding of the inputs \emph{unless} the contracting dimension is sharded or both matrices are sharded along the same axis. For example, this works fine

\begin{equation*}
\mathbf{A}[I_X, J] \cdot \mathbf{B}[J, K_Y] \rightarrow \mathbf{C}[I_X, K_Y]
\end{equation*}

with no communication whatsoever, and results in a tensor sharded across both the X and Y hardware dimensions. Try to think about why this is. Basically, the computation is \emph{independent} of the sharding, since each batch entry has some local chunk of the axis being contracted that it can multiply and reduce. Any of these cases work fine and follow this rule:

\begin{align*}
\mathbf{A}[I, J] \cdot \mathbf{B}[J, K] \rightarrow &\ \mathbf{C}[I, K] \\
\mathbf{A}[I_X, J] \cdot \mathbf{B}[J, K] \rightarrow &\ \mathbf{C}[I_X, K]\\
\mathbf{A}[I, J] \cdot \mathbf{B}[J, K_Y] \rightarrow &\ \mathbf{C}[I, K_Y]\\
\mathbf{A}[I_X, J] \cdot \mathbf{B}[J, K_Y] \rightarrow &\ \mathbf{C}[I_X, K_Y]
\end{align*}

Because neither \textbf{A} nor \textbf{B} has a sharded contracting dimension \textbf{J}, we can simply perform the local block matrix multiplies of the inputs and the results will \emph{already} be sharded according to the desired output shardings. When both multiplicands have non-contracting dimensions sharded along the same axis, this is no longer true (see the invalid shardings section for details).

\subsection*{Case 2: one multiplicand has a sharded contracting dimension}
\addcontentsline{toc}{subsection}{Case 2: one multiplicand has a sharded contracting dimension}

Let's consider what to do when one input \textbf{A} is sharded along the contracting \textbf{J} dimension and \textbf{B} is fully replicated:

$$\mathbf{A}[I, J_X] \cdot \mathbf{B}[J, K] \rightarrow \mathbf{C}[I, K]$$

We cannot simply multiply the local chunks of \textbf{A} and \textbf{B} because we need to sum over the full contracting dimension of \textbf{A}, which is split across the X axis. Typically, we first ``\textbf{AllGather}'' the shards of \textbf{A} so every device has a full copy, and only then multiply against \textbf{B:}

$$\textbf{AllGather}_X[I, J_X] \rightarrow \mathbf{A}[I, J]$$

$$\mathbf{A}[I, J] \cdot \mathbf{B}[J, K] \rightarrow \mathbf{C}[I, K]$$

This way the actual multiplication can be done fully on each device.

\begin{takeawaybox}
When multiplying matrices where one of the matrices is sharded along the contracting dimension, we generally AllGather it first so the contraction is no longer sharded, then do a local matmul.
\end{takeawaybox}

Note that when \textbf{B} is not also sharded along X, we could also do the local partial matmul and then sum (or \emph{AllReduce}) the sharded partial sums, which can be faster in some cases. See Question 4 below.

\textbf{What is an AllGather?} An AllGather is the first core MPI communication primitive we will discuss. An AllGather \emph{removes the sharding} along an axis and reassembles the shards spread across devices onto \emph{each} device along that axis. Using the notation above, an AllGather removes a subscript from a set of axes, e.g.

$$\textbf{AllGather}_{XY}(A[I_{XY}, J]) \rightarrow A[I, J]$$

We don't have to remove all subscripts for a given dimension, e.g. $A[I_{XY}, J] \rightarrow A[I_Y, J]$ is also an AllGather, just over only a single axis. Also note that we may also wish to use an AllGather to remove \emph{non-contracting} dimension sharding, for instance in the matrix multiply:

$$A[I_X, J] \cdot B[J, K] \rightarrow C[I, K]$$

We could either AllGather \textbf{A} initially to remove the input sharding, or we can do the sharded matmul and then AllGather the result \textbf{C}.

\textbf{How is an AllGather actually performed?} To perform a 1-dimensional AllGather around a single TPU axis (a ring), we basically have each TPU pass its shard around a ring until every device has a copy.\footnote{A GPU AllGather can also work like this, where you create a ring out of the GPUs in a node and pass the chunks around in that (arbitrary) order.}

We can either do an AllGather in one direction or both directions (two directions is shown above). If we do one direction, each TPU sends chunks of size $\text{bytes} / N$ over $N - 1$ hops around the ring. If we do two directions, we have $\lfloor \frac{N}{2} \rfloor$ hops of size $2 \cdot \text{bytes} / N$.

\textbf{How long does this take?} Let's take the bidirectional AllGather and calculate how long it takes. Let $V$ be the number of bytes in the array, and $X$ be the number of shards on the contracting dimension. Then from the above diagram, each hop sends $V / \lvert X\rvert$ bytes in each direction, so each hop takes

$$T_{hop} = \frac{2 \cdot V}{X \cdot W_\text{ici}}$$

where $W_\text{ici}$ is the \textbf{bidirectional} ICI bandwidth.\footnote{The factor of 2 in the numerator comes from the fact that we're using the bidirectional bandwidth. We send $V / X$ in each direction, or $2V / X$ total.} We need to send a total of $\lvert X\rvert / 2$ hops to reach every TPU\footnote{technically, $\lfloor X / 2 \rfloor$}, so the total reduction takes

$$T_{total} = \frac{2 \cdot V \cdot X}{2 \cdot X \cdot W_\text{ici}}$$

$$T_{total} = \frac{V}{W_\text{ici}}$$

Note that this \textbf{doesn't depend on $X$!} That's kind of striking, because it means even though our TPUs are only locally connected, the locality of the connections doesn't matter. We're just bottlenecked by the speed of each link.

\begin{takeawaybox}
When performing an AllGather (or a ReduceScatter or AllReduce) in a throughput-bound regime, the actual communication time depends only on the size of the array and the available bandwidth, not the number of devices over which our array is sharded!
\end{takeawaybox}

\textbf{A note on ICI latency:} Each hop over an ICI link has some intrinsic overhead regardless of the data volume. This is typically around 1us. This means when our array $A$ is very small and each hop takes less than 1us, we can enter a ``latency-bound'' regime where the calculation \emph{does} depend on $X$.

Let $T_\text{min}$ be the minimum time for a single hop. Then

$$T_{hop} = \max \left[ T_{min}, \frac{2 \cdot V}{X \cdot W_\text{ici}} \right]$$

$$T_{total} = \max \left[ \frac{T_{min} \cdot X}{2}, \frac{V}{W_\text{ici}} \right]$$

since we perform $X / 2$ hops. For large reductions or gathers, we're solidly bandwidth bound. We're sending so much data that the overhead of each hop is essentially negligible. But for small arrays (e.g. when sampling from a model), this isn't negligible, and the ICI bandwidth isn't relevant. We're bound purely by latency. Another way to put this is that given a particular TPU, e.g. TPU v5e with \texttt{4.5e10} unidirectional ICI bandwidth, sending any buffer under \texttt{4.5e10 * 1e-6 = 45kB} will be latency bound.

Here is an empirical measurement of AllGather bandwidth on a TPU v5e 8x16 slice. The array is sharded across the 16 axis so it has a full bidirectional ring.

\begin{figure}[htb]
\centering
\includegraphics[width=\textwidth]{images/all-gather-bandwidth.png}
\caption{Empirical bandwidth and estimated link bandwidth for TPU v5e during an AllGather. BW in orange is the actual bytes per second AllGathered, while the blue curve shows the empirical unidirectional link bandwidth calculated according to the known cost of the collective.}
\end{figure}

Note that we not only achieve about 95\% of the peak claimed bandwidth (\texttt{4.5e10}) but also that we achieve this peak at about 10MB, which when 16-way sharded gives us about 500kB per device (\emph{aside}: this is much better than GPUs).

\textbf{What happens when we AllGather over multiple axes?} When we gather over multiple axes, we have multiple dimensions of ICI over which to perform the gather. For instance, AllGather\textsubscript{XY}([B, D\textsubscript{XY}]) operates over two hardware mesh axes. This increases the available bandwidth by a factor of $N_\text{axes}$.

When considering latency, we end up with the general rule:

$$T_{total} = \max \left[ \frac{T_{min} \cdot \sum_{i} |X_i|}{2}, \frac{V}{W_\text{ici} \cdot N_\text{axes}} \right]$$

where $\sum_i \lvert X_i \rvert / 2$ is the length of the longest path in the TPU mesh.

\textbf{\textcolor{purple}{Pop Quiz 2 [AllGather time]:}} Using the numbers from Part 2, how long does it take to perform the AllGather\textsubscript{Y}([E\textsubscript{Y}, F]) → [E, F] on a TPUv5e with a 2D mesh \texttt{\{'X': 8, 'Y': 4\}}, $E = 2048$, $F = 8192$ in bfloat16? What about with $E=256, F=256$?

\emph{For part (1)}, we can use the formula above. Since we're performing the AllGather over one axis, we have $T_{\text{comms}} = \text{34e6} / \text{9e10} = \text{377us}$. To check that we're not latency-bound, we know over an axis of size 4, we'll have at most 3 hops, so our latency bound is something like 3us, so we're not close. However, TPU v5e only has a wraparound connection when one axis has size 16, so here \emph{we actually can't do a fully bidirectional AllGather}. We have to do 3 hops for data from the edges to reach the other edge, so in theory we have more like $T_{\text{comms}} = 3 * \text{8.4e6} / \text{4.5e10} = 560\mu s$. \textbf{Here's an actual profile} from this Colab, which shows $680 \mu s$, which is reasonable since we're likely not getting 100\% of the theoretical bandwidth! \emph{For part (2)} each shard has size \texttt{64 * 256 * 2 = 32kB. 32e3 / 4.5e10 = 0.7us}, so we're latency bound. Since we have 3 hops, this will take roughly 3 * 1us = 3us. In practice, it's closer to 8us.

\begin{takeawaybox}
\textbf{Note:} when we have a 2D mesh like \texttt{\{'X': 16, 'Y': 4\}}, it is not necessary for each axis to correspond to a specific \emph{hardware} axis. This means for instance the above could describe a 4x4x4 TPU v5p cube with 2 axes on the $X$ axis. This will come into play later when we describe data parallelism over multiple axes.
\end{takeawaybox}

\subsection*{Case 3: both multiplicands have sharded contracting dimensions}
\addcontentsline{toc}{subsection}{Case 3: both multiplicands have sharded contracting dimensions}

The third fundamental case is when both multiplicands are sharded on their contracting dimensions, along the same mesh axis:

$$\textbf{A}[I, J_X] \cdot \textbf{B}[J_X, K] \rightarrow C[I, K]$$

In this case the \emph{local} sharded block matrix multiplies are at least \emph{possible} to perform, since they will share the same sets of contracting indices. But each product will only represent a \emph{partial sum} of the full desired product, and each device along the \textbf{X} dimension will be left with different \emph{partial sums} of this final desired product. This is so common that we extend our notation to explicitly mark this condition:

$$\textbf{A}[I, J_X] \cdot_\text{LOCAL} \textbf{B}[J_X, K] \rightarrow C[I, K] \{\ U_X \}$$

The notation \textbf{\{ U\textsubscript{X} \}} reads ``\textbf{unreduced} along X mesh axis'' and refers to this status of the operation being ``incomplete'' in a sense, in that it will only be finished pending a final sum. The $\cdot_\text{LOCAL}$ syntax means we perform the local sum but leave the result unreduced.

This can be seen as the following result about matrix multiplications and outer products:

$$A \cdot B = \sum_{i=1}^{P} \underbrace{A_{:,i} \otimes B_{i,:}}_{\in \mathbb{R}^{n \times m}}$$

where ⊗ is the outer product. Thus, if TPU \textbf{i} on axis \textbf{X} has the \textbf{i}th column of \textbf{A}, and the \textbf{i}th row of \textbf{B}, we can do a local matrix multiplication to obtain $A_{:,i} \otimes B_{i,:} \in \mathbb{R}_{n\times m}$. This matrix has, in each entry, the \textbf{i}th term of the sum that \textbf{A • B} has at that entry. We still need to perform that sum over \textbf{P}, which we sharded over mesh axis \textbf{X}, to obtain the full \textbf{A • B}. This works the same way if we write \textbf{A} and \textbf{B} by blocks (i.e. shards), and then sum over each resulting shard of the result.

We can perform this summation using a full \textbf{AllReduce} across the \textbf{X} axis to remedy this:

\begin{align*}
A[I, J_X] \cdot_\text{LOCAL} B[J_X, K] \rightarrow &\ C[I, K] \{ U_X \} \\
\textbf{AllReduce}_X C[I, K] \{ U_X \} \rightarrow &\ C[I, K]
\end{align*}

AllReduce removes partial sums, resulting in \emph{each} device along the axis having the same fully-summed value. AllReduce is the second of several key communications we'll discuss in this section, the first being the AllGather, and the others being ReduceScatter and AllToAll. An AllReduce takes an array with an unreduced (partially summed) axis and performs the sum by passing those shards around the unreduced axis and accumulating the result. The signature is

$$\textbf{AllReduce}_Y A[I_X, J] \{U_Y\} \rightarrow A[I_X, J]$$

This means it simply removes the $\{U_Y\}$ suffix but otherwise leaves the result unchanged.

\textbf{How expensive is an AllReduce?} One mental model for how an AllReduce is performed is that every device sends its shard to its neighbors, and sums up all the shards that it receives. Clearly, this is more expensive than an AllGather because each ``shard'' has the same shape as the full array. Generally, \textbf{an AllReduce is twice as expensive as an AllGather.} One way to see this is to note that an \textbf{AllReduce} can be expressed as a composition of two other primitives: a \textbf{ReduceScatter} and an \textbf{AllGather}. Like an AllReduce, a ReduceScatter resolves partial sums on an array but results in an output `scattered' or partitioned along a given dimension. AllGather collects all those pieces and `unpartitions/unshards/replicates' the logical axis along that physical axis.

\begin{align*}
\textbf{ReduceScatter}_{Y,J} : A[I_X,J] \{U_Y\} \rightarrow &\ A[I_X, J_Y] \\
\textbf{AllGather}_Y : A[I_X, J_Y] \rightarrow &\ A[I_X, J]
\end{align*}

\textbf{What about a ReduceScatter?} Just as the AllReduce removes a subscript ($F_Y \to F$ above), a ReduceScatter sums an unreduced/partially summed array and then scatters (shards) a different logical axis along the same mesh axis. $[F]\{U_Y\} \to [F_Y]$. 

The communication time for each hop is simply the per-shard bytes $V / Y$ divided by the bandwidth $W_\text{ici}$, as it was for an AllGather, so we have

$$T_{\text{comms per AllGather or ReduceScatter}} = \frac{V}{W_\text{ici}}$$

$$T_{\text{comms per AllReduce}} = 2 \cdot \frac{V}{W_\text{ici}}$$

where $W_\text{ici}$ is the bidirectional bandwidth, so long as we have a full ring to reduce over.

\subsection*{Case 4: both multiplicands have a non-contracting dimension sharded along the same axis}
\addcontentsline{toc}{subsection}{Case 4: both multiplicands have a non-contracting dimension sharded along the same axis}

Each mesh dimension can appear at most once when sharding a tensor. Performing the above rules can sometimes lead to a situation where this rule is violated, such as:

$$A[I_X, J] \cdot B[J, K_X] \rightarrow C[I_X, K_X]$$

This is invalid because a given shard, say \textbf{i}, along dimension \textbf{X}, would have the \textbf{(i, i)}th shard of \textbf{C}, that is, a diagonal entry. There is not enough information among all shards, then, to recover anything but the diagonal entries of the result, so we cannot allow this sharding.

The way to resolve this is to AllGather some of the dimensions. Here we have two choices:

\begin{align*}
\textbf{AllGather}_X A[I_X, J] \rightarrow &\ A[I, J] \\
A[I, J] \cdot B[J, K_X] \rightarrow &\ C[I, K_X]
\end{align*}

or

\begin{align*}
\textbf{AllGather}_X B[J, K_X] \rightarrow &\ B[J, K] \\
A[I_X, J] \cdot B[J, K] \rightarrow &\ C[I_X, K]
\end{align*}

In either case, the result will only mention \textbf{X} once in its shape. Which one we pick will be based on what sharding the following operations need.

\section*{A Deeper Dive into TPU Communication Primitives}
\addcontentsline{toc}{section}{A Deeper Dive into TPU Communication Primitives}

The previous 4 cases have introduced several ``core communication primitives'' used to perform sharded matrix multiplications:

\begin{enumerate}
    \item \textbf{AllGather:} removes a subscript from a sharding, gathering the shards.
    \item \textbf{ReduceScatter:} removes an ``un-reduced'' suffix from an array by summing shards over that axis, leaving the array sharded over a second axis.
    \item \textbf{AllReduce:} removes an ``un-reduced'' suffix, leaving the array unsharded along that axis.
\end{enumerate}

There's one more core communication primitive to mention that arises in the case of Mixture of Experts (MoE) models and other computations: the \textbf{AllToAll}.

\subsection*{Our final communication primitive: the AllToAll}
\addcontentsline{toc}{subsection}{Our final communication primitive: the AllToAll}

A final fundamental collective which does not occur naturally when considering sharded matrix multiplies, but which comes up constantly in practice, is the \textbf{AllToAll} collective, or more precisely the special case of a \emph{sharded transposition} or resharding operation. e.g.

$$\textbf{AllToAll}_{X, J} A[I_X, J] \rightarrow A[I, J_X]$$

AllToAlls are typically required to rearrange sharded layouts between different regions of a sharded computation that don't have compatible layout schemes. They arise naturally when considering sharded mixture-of-experts models. \emph{You can think of an AllToAll as moving a subscript from one axis to another}. Because an all to all doesn't need to replicate all of the data of each shard across the ring, it's actually \emph{cheaper} than an AllGather (by a factor of ¼)\footnote{For even-sized bidirectional rings, each device will send $(N/2 + (N/2-1) + \ldots + 1)$ chunks right and $((N/2-1) + \ldots + 1)$ chunks left $= 0.5 \cdot (N / 2) \cdot (N/2 + 1) + 0.5 \cdot (N / 2) \cdot (N/2 - 1) = N^2/4$. The size of each chunk (aka shard of a shard) is $\text{bytes} / N^2$ so the per-device cost is $(\text{bytes} / N^2) \cdot N^2 / 4 = \text{bytes} / 4$. This result scales across all devices as the total bandwidth scales with device number.}.

If we generalize to an ND AllToAll, the overall cost for an array of $V$ bytes on an AxBxC mesh is

$$T_\text{comms per AllToAll} = \frac{V \cdot \max(A, B, C, \ldots)}{4 \cdot N \cdot W_\text{ici}}$$

where as usual $W_\text{ici}$ is the bidirectional ICI bandwidth. For a 1D mesh, this reduces to $V / (4 \cdot W_\text{ici})$, which is 1 / 4 the cost of an AllReduce. In 2D, the cost actually scales down with the size of the smallest axis.

\emph{Aside: If you want a hand-wavy derivation of this fact, start with a 1D torus $\mathbb{Z} / N\mathbb{Z}$. If we pick a source and target node at random, they are on average N / 4 hops from each other, giving us a cost of $(V \cdot N) / (4 * N)$. Now if we consider an ND torus, each axis is basically independent. Each node has $1 / Z$ bytes and on average has to hop its data $\max(A, B, C, \ldots) / 4$ hops.}

\subsection*{More about the ReduceScatter}
\addcontentsline{toc}{subsection}{More about the ReduceScatter}

ReduceScatter is a more fundamental operation than it first appears, as it is actually the derivative of an AllGather, and vice versa. i.e. if in the forward pass we have:

$$\textbf{AllGather}_X A[I_X] \rightarrow A[I]$$

Then we ReduceScatter the reverse-mode derivatives \textbf{A'} (which will in general be different on each shard) to derive the sharded \textbf{A'}:

$$\textbf{ReduceScatter}_X A'[I] \{ U_X \} \rightarrow A'[I_X]$$

Likewise, $\text{ReduceScatter}_X(A[I] \{U_X\}) \to A[I_X]$ in the forward pass implies $\text{AllGather}_{X}(A'[I_X]) \to A'[I]$ in the backwards pass.

\paragraph{How AllGather and ReduceScatter are derivatives of eachother}

This stems from the fact that broadcasts and reductions are transposes of eachother as linear operators, and AllGather and ReduceScatter are outer products (also known as \href{https://en.wikipedia.org/wiki/Kronecker_product}{Kronecker products}) of broadcast and reduce, respectively. Concretely, if we have a vector $x \in \mathbb{R}^n$, any number of devices $p \in \mathbb{N}$, and we let $u = (1, \ldots, 1) \in \mathbb{R}^p$, we can define broadcast and reduce in the following way, which should match your intuitive understanding of them:

\begin{align*}
\text{broadcast} &: \mathbb{R}^n \rightarrow \mathbb{R}^{p n} \\
\text{broadcast} &= u \otimes \mathbf{I}_n \\
\text{reduce} &: \mathbb{R}^{p n} \rightarrow \mathbb{R}^n \\
\text{reduce} &= u^T \otimes \mathbf{I}_n
\end{align*}

Let's see how this looks in an example, where $n = 1$, $p = 2$. If $x = (7)$, we have $$\text{broadcast}(x) = \left(\begin{pmatrix} 1 \\ 1 \end{pmatrix} \otimes \begin{pmatrix} 1 \end{pmatrix}\right) x = \begin{pmatrix} 1 \\ 1 \end{pmatrix} x = \begin{pmatrix}  7\\  7  \end{pmatrix} \in \mathbb{R}^{p n}$$. This matches what we'd expect, broadcasting a vector in $\mathbb{R}^n$ to $\mathbb{R}^{pn}$. Now letting $y = (8, 9)$, we have $$\text{reduce}(y) = \left(\begin{pmatrix} 1 & 1 \end{pmatrix} \otimes \begin{pmatrix} 1\end{pmatrix}\right) y = \begin{pmatrix} 1 & 1  \end{pmatrix} \begin{pmatrix}  8 \\ 9  \end{pmatrix} = \begin{pmatrix}   17    \end{pmatrix}.$$ This again matches what we'd expect, reducing a vector in $\mathbb{R}^{p n}$ to a vector in $\mathbb{R}^{n}$. Since $(A \otimes B)^T = A^T \otimes B^T$ for any two matrices $A$ and $B$, we see that $\text{reduce} = \text{broadcast}^T$. We recover AllGather and ReduceScatter as the following outer products:

\begin{align*}
\text{AllGather} &: \mathbb{R}^{p n} \rightarrow \mathbb{R}^{p^2 n} \\
\text{AllGather} &= \text{broadcast} \otimes \mathbf{I}_p \\
\text{ReduceScatter} &= \mathbb{R}^{p^2 n} \rightarrow \mathbb{R}^{p n} \\
\text{ReduceScatter} &= \text{reduce} \otimes \mathbf{I}_p
\end{align*}

Here we think of $\mathbb{R}^{p^2 n}$ as $\mathbb{R}^{p \times p n}$, so one $\mathbb{R}^{p n}$ vector for each of our $p$ devices. We suggest playing around with small examples, say $n = 2$, $p = 3$, to see what these operators look like as matrices. Using the same transposition property, we once more obtain $\text{AllGather}^T = \text{ReduceScatter}$, and of course $\text{ReduceScatter}^T = \text{AllGather}$. This transposition will arise during backpropagation, since if we have $y = Ax$ for some linear operator $A$, such as AllGather or ReduceScatter, then during backpropagation we will have the derivative of the loss with respect to $y$, $\frac{\partial L}{\partial y}$, and we obtain $\frac{\partial L}{\partial x}$ as $\frac{\partial L}{\partial x} = A^T \frac{\partial L}{\partial y}$. This shows how the derivative of AllGather will be ReduceScatter, and viceversa.

Turning an AllReduce into an AllGather and ReduceScatter also has the convenient property that we can defer the final AllGather until some later moment. Very commonly we'd rather not pay the cost of reassembling the full matrix product replicated across the devices. Rather we'd like to preserve a sharded state even in this case of combining two multiplicands with sharded contracting dimensions:

$$A[I, J_X] \cdot B[J_X, K] \rightarrow C[I, K_X]$$

In this case, we can also perform a ReduceScatter instead of an AllReduce, and then optionally perform the AllGather at some later time, i.e.

\begin{align*}
A[I, J_X] \cdot_{LOCAL} B[J_X, K] \rightarrow &\ C[I, K] \{ U_X \} \\
\textbf{ReduceScatter}_{X,K} C[I, K] \{ U_X \} \rightarrow &\ C[I, K_X]
\end{align*}

Note that ReduceScatter \emph{introduces} a sharded dimension, and so has a natural freedom to shard along either the \textbf{I} or \textbf{K} named dimensions in this case. We generally need to choose \emph{which} named dimension to introduce a new sharding to when using a ReduceScatter (though the choice is usually forced by the larger modeling context). This is why we use the syntax \textbf{ReduceScatter\textsubscript{X,K}} to specify the axis to shard.

\section*{What Have We Learned?}
\addcontentsline{toc}{section}{What Have We Learned?}

\begin{itemize}
    \item The sharding of an array is specified by a \textbf{Mesh} that names the physical, hardware axes of our TPU mesh and a \textbf{Sharding} that assigns mesh axis names to the logical axes of the array.
    \begin{itemize}
        \item For example, \textbf{A}[I$_{\text{XY}}$, J] describes an abstract array \textbf{A} with its first dimension sharded along two mesh axes X and Y. Combined with Mesh(mesh\_shape=(4, 8), axis\_names=(`X', `Y')) or the abbreviated Mesh(\{`X': 4, `Y': 8\}), this tells us our array is sharded 32 ways along the first dimension.
    \end{itemize}

    \item \textbf{Arithmetic with sharded arrays works exactly like with unsharded arrays unless you perform a contraction along a sharded axis}. In that case, we have to introduce some communication. We consider four cases:
    \begin{enumerate}
        \item \textit{Neither array is sharded along the contracting dimension}: no communication is needed.
        \item \textit{One array is sharded along the contracting dimension} (or the contracting dimensions are sharded along different axes): we AllGather one of the inputs before performing the operation.
        \item \textit{Both arrays are identically sharded along the contracting dimension:} we multiply the shards locally then perform an AllReduce or ReduceScatter.
        \item \textit{Both arrays are sharded along the same mesh axis along a non-contracting dimension:} we AllGather one of the inputs first.
    \end{enumerate}

    \item TPUs use roughly \textbf{4 core communication primitives}:
    \begin{enumerate}
        \item AllGather: $[A_X, B] \to [A, B]$
        \item ReduceScatter: $[A, B] \{U_X\} \to [A, B_X]$
        \item AllToAll: $[A, B_X] \to [A_X, B]$
        \item AllReduce: $[A_X, B]\{U_Y\} \to [A_X, B]$ (technically not a primitive since it combines a ReduceScatter + AllGather)
    \end{enumerate}
\end{itemize}

\begin{figure}[htb]
    \centering
    \includegraphics[width=\textwidth]{images/all-collectives.png}
    \caption{All collective communication primitives used in TPU sharding.}
    \label{fig:all-collectives}
\end{figure}

\begin{itemize}
    \item The cost and latency of each of these operations \textbf{doesn't depend on the size of the axis (as long as they're bandwidth bound)}, but only on the size of the input arrays and the bandwidth of the link. For a unidirectional AllGather/ReduceScatter:
\end{itemize}

\begin{align}
T_{\text{comm per AllGather or ReduceScatter}} &= \frac{\text{Data volume}}{\text{bandwidth}} \cdot \frac{\text{Axis} - 1}{\text{Axis}} \nonumber \\
&\longrightarrow \frac{\text{Data volume}}{\text{bandwidth (bidirectional)}}
\end{align}

\begin{itemize}
    \item An AllReduce is composed of a ReduceScatter followed by an AllGather, and thus has 2x the above cost. An AllToAll only has to pass shards part-way around the ring and is thus ¼ the cost of an AllGather. Here's a summary:
\end{itemize}

{\scriptsize
\setlength{\tabcolsep}{2pt}
\begin{longtable}{p{2cm} p{3cm} p{1.8cm} p{2cm}}
\caption{Summary of collective communication operations} \\
\toprule
\textbf{Operation} & \textbf{Description} & \textbf{Syntax} & \textbf{Runtime} \\
\midrule
\endfirsthead

\multicolumn{4}{c}{\tablename\ \thetable\ -- continued from previous page} \\
\toprule
\textbf{Operation} & \textbf{Description} & \textbf{Syntax} & \textbf{Runtime} \\
\midrule
\endhead

\midrule
\multicolumn{4}{r}{Continued on next page} \\
\endfoot

\bottomrule
\endlastfoot

\textbf{AllGather} & Gathers shards of an array along an axis, removing a subscript. & $[A_X, B] \to [A, B]$ & bytes / (bidir. ICI BW * num\_axes) \\
\textbf{ReduceScatter} & Sums a partially summed array and shards it along another axis. & $[A, B] \{U_X\} \to [A_X, B]$ & Same as AllGather \\
\textbf{AllReduce} & Sums a partially summed array. Removes \{U$_x$\}. Combines AllGather and ReduceScatter. & $[A_X, B]\{U_Y\} \to [A_X, B]$ & 2 * AllGather \\
\textbf{AllToAll} & Gathers an axis and shards a different dimension along same axis. & $[A, B_X] \to [A_X, B]$ & AllGather / 4 (bidir. ring) \\
\end{longtable}
}

\section{Worked Problems}

I'm going to invent a new model based on LLaMA-2 13B for this section. Here are the details:

{\scriptsize
\setlength{\tabcolsep}{2pt}
\begin{longtable}{p{5cm}p{2.5cm}}
\hline
hyperparam & value \\
\hline
L (num\_layers) & 64 \\
D (d\_model) & 4,096 \\
F (ffw\_dimension) & 16,384 \\
N (num\_heads) & 32 \\
K (num\_kv\_heads) & 8 \\
H (qkv\_dim) & 256 \\
V (num\_embeddings) & 32,128 \\
\hline
\end{longtable}
}

\textbf{Question 1:} How many parameters does the above model have? How large are its KV caches per token in int8? \textit{You can assume we share the input and output projection matrices.}



\textbf{Question 2:} Say we want to serve this model on a TPUv5e 4x4 slice and can fully shard our KV cache over this topology. What's the largest batch size we can fit, assuming we use int8 for everything and want to support 128k sequences? What if we dropped the number of KV heads to 1?



\textbf{Question 3:} How long does it take to load all the parameters into the MXU from HBM assuming they're fully sharded on a TPU v5e 4x4 slice? Assume int8 parameters. \textit{This is a good lower bound on the per-step latency.}



\textbf{Question 4:} Let's say we want to serve this model on a TPUv5e 4x4 slice using int8 FLOPs and parameters/activations. How would we shard it for both prefill and decode? \textit{Hint: maybe answer these questions first:}

\begin{enumerate}
\item What does ICI look like on a 4x4?
\item What's the roofline bound on tensor parallelism?
\item How can we shard the KV caches?
\end{enumerate}

For this sharding, what is the rough per-step latency for generation?

\textbf{Question 5:} Let's pretend the above model is actually an MoE. An MoE model is effectively a dense model with E copies of the FFW block. Each token passes through k of the FFW blocks and these \texttt{k} are averaged to produce the output. Let's use \texttt{E=16} and \texttt{k=2} with the above settings.

\begin{enumerate}
\item How many total and activated parameters does it have? \textit{Activated means used by any given token.}
\item What batch size is needed to become FLOPs bound on TPU v5e?
\item How large are its KV caches per token?
\item How many FLOPs are involved in a forward pass with T tokens?
\end{enumerate}



\textbf{Question 6:} With MoEs, we can do ``expert sharding'', where we split our experts across one axis of our mesh. In our standard notation, our first FFW weight has shape \texttt{[E, D, F]} and we shard it as [E$_Z$, D$_X$, F$_Y$] where \texttt{X} is only used during training as our FSDP dimension. Let's say we want to do inference on a TPU v5e:

\begin{enumerate}
\item What's the HBM weight loading time for the above model on a TPU v5e 8x16 slice with Y=8, Z=16? How much free HBM is available per TPU?
\item What is the smallest slice we could fit our model on?
\end{enumerate}

\textbf{Question 7 [2D model sharding]:} Here we'll work through the math of what the \href{https://arxiv.org/pdf/2211.05102}{ESTI paper} calls 2D weight-stationary sharding. We describe this briefly in Appendix B, but try doing this problem first to see if you can work out the math. The basic idea of 2D weight stationary sharding is to shard our weights along both the $D$ and $F$ axes so that each chunk is roughly square. This reduces the comms load and allows us to scale slightly farther.

Here's the algorithm for 2D weight stationary:

\begin{enumerate}
\item In[B, D$_X$] = \textbf{AllGather}$_{YZ}$(In[B, D$_{XYZ}$])
\item Tmp[B, F$_{YZ}$] \{U.X\} = In[B, D$_X$] *$_D$ W$_{\text{in}}$[D$_X$, F$_{YZ}$]
\item Tmp[B, F$_{YZ}$] = \textbf{AllReduce}$_X$(Tmp[B, F$_{YZ}$] \{U.X\})
\item Out[B, D$_X$] \{U.YZ\} = Tmp[B, F$_{YZ}$] *$_F$ W2[F$_{YZ}$, D$_X$]
\item Out[B, D$_{XYZ}$] = \textbf{ReduceScatter}$_{YZ}$(Out[B, D$_X$] \{U.YZ\})
\end{enumerate}

Your goal is to work out $T_\text{math}$ and $T_\text{comms}$ for this algorithm and find when it will outperform traditional 3D model sharding?

Let's work out $T_\text{math}$ and $T_\text{comms}$ All our FLOPs are fully sharded so as before we have $T_\text{math} = 4BDF / (N \cdot C)$ but our comms are now

\begin{align*}
T_\text{2D comms} &= \frac{2BD}{2X \cdot W_\text{ici}} + \frac{4BF}{YZ \cdot W_\text{ici}} + \frac{2BD}{2X \cdot W_\text{ici}} \\
&= \frac{2BD}{X \cdot W_\text{ici}} + \frac{4BF}{YZ \cdot W_\text{ici}}
\end{align*}

where we note that the AllReduce is twice as expensive and we scale our comms by the number of axes over which each operation is performed. Assuming we have freedom to choose our topology and assuming $F=4D$ (as in LLaMA-2), we claim (by some basic calculus) that the optimal values for $X$, $Y$, and $Z$ are $X = \sqrt{N / 8}$, $YZ = \sqrt{8N}$ so the total communication is

\begin{align*}
T_\text{2D comms} &= \frac{2B}{W_\text{ici}} \left(\frac{D}{X} + \frac{8D}{YZ}\right) \\
&= \frac{\sqrt{128} BD}{\sqrt{N} \cdot W_\text{ici}} \approx \frac{11.3 BD}{\sqrt{N} \cdot W_\text{ici}}
\end{align*}

Firstly, copying from above, normal 1D model parallelism would have $T_\text{model parallel comms} = 4BD / (3 \cdot W_\text{ici})$, so when are the new comms smaller? We have

\begin{align*}
T_\text{model parallel comms} > T_\text{2D comms} &\iff \frac{4BD}{3 \cdot W_\text{ici}} > \frac{\sqrt{128} BD}{\sqrt{N} \cdot W_\text{ici}} \\
&\iff N > 128 \cdot \left(\frac{3}{4}\right)^2 = 81
\end{align*}

For a general $F$, we claim this condition is

$$N > 32 \cdot \left(\frac{F}{D}\right) \cdot \left(\frac{3}{4}\right)^2$$

So that tells us if we have more than 81 chips, we're better off using this new scheme. Now this is a slightly weird result because we've historically found ourselves ICI bound at around \textasciitilde 20 way tensor parallelism. But here, even if we're communication-bound, our total communication continues to decrease with the number of total chips! What this tells us is that we can continuous to increase our chips, increase our batch size, do more parameter scaling, and see reduced latency.


\chapter{All the Transformer Math You Need to Know}
\label{chap:transformers}

% Chapter 4: Transformer Math
% This chapter is divided into multiple files for easier management

\section*{Counting Dots}
\addcontentsline{toc}{section}{Counting Dots}

Let's start with vectors $x$,$y$ and matrices $A$,$B$ of the following shapes:

$$
\def \red#1{\textcolor{red}{#1}}
\def \green#1{\textcolor{green}{#1}}
\def \blue#1{\textcolor{blue}{#1}}
\def \purple#1{\textcolor{purple}{#1}}
\def \orange#1{\textcolor{orange}{#1}}
\def \gray#1{\textcolor{gray}{#1}}

\begin{array}{cc}
\textrm{array}  & \textrm{shape} \\ \hline
x               & \textrm{[P]}   \\
y               & \textrm{[P]}   \\
A               & \textrm{[N P]} \\
B               & \textrm{[P M]} \\
\hline
\end {array}
$$

\begin{itemize}
\item A dot product of $x \cdot y$ requires $P$ \emph{adds} and \emph{multiplies}, or $2P$ floating-point operations total.
\item A matrix-vector product $Ax$ does $N$ dot-products along the rows of $A$, for $2NP$ FLOPs.
\item A matrix-matrix product $AB$ does a matrix-vector product for each of the $M$ columns of $B$, for $2NPM$ FLOPs total.
\item In general, if we have two higher dimensional arrays $C$ and $D$, where some dimensions are \textcolor{red}{CONTRACTING} and some are \textcolor{blue}{BATCHING}.  (e.g. $C[\textcolor{blue}{GH}IJ\textcolor{red}{KL}], D[\textcolor{blue}{GH}MN\textcolor{red}{KL}]$) then the FLOPs cost of this contraction is two times the product of all of the $C$ and $D$ dimensions where the batch and contraction dimensions are only counted once, (e.g. $2\textcolor{blue}{GH}IJMN\textcolor{red}{KL}$). Note that a dimension is only batching if it occurs in both multiplicands. (Note also that the factor of 2 won't apply if there are no contracting dimensions and this is just an elementwise product.)
\end{itemize}

{\footnotesize
\begin{center}
\begin{tabular}{lll}
\toprule
Operation & FLOPs & Data \\
\midrule
$x \cdot y$  & $2P$   & $2P$      \\
$A x$        & $2NP$  & $NP + P$  \\
$AB$         & $2NPM$ & $NP + PM$ \\
General einsum & $2 \prod c_i \prod d_j^*$ & $\prod c_i + \prod d_j$ \\
\bottomrule
\end{tabular}
\end{center}

\noindent\textit{*where $d_j^*$ are non-batch, non-contracting dims of $d$}
}

Make note of the fact that for a matrix-matrix multiply, the \emph{compute} scales cubically $O(N^3)$ while the data transfer only scales quadratically $O(N^2)$ -- this means that as we scale up our matmul size, it becomes \emph{easier} to hit the compute-saturated limit. This is extremely unusual, and explains in large part why we use architectures dominated by matrix multiplication -- they're amenable to being scaled!

\begin{figure}[htb]
\centering
\includegraphics[width=\textwidth]{images/matmul-flops.png}
\end{figure}

\subsection*{Forward and reverse FLOPs}
\addcontentsline{toc}{subsection}{Forward and reverse FLOPs}

During training, we don't particularly care about the result of a given matrix multiply; we really care about its derivative. That means we do significantly more FLOPs during backpropagation.

If we imagine \textbf{B} is just one matrix in a larger network and \textbf{A} are our input activations with \textbf{C = A B}, the derivative of the loss \textbf{L} with respect to \textbf{B} is given by the chain rule:

$$\frac{\partial L}{\partial B} = \frac{\partial L}{\partial C}\frac{\partial C}{\partial B} = A^T \left(\frac{\partial L}{\partial C}\right)$$

which is an outer product and requires $2NPM$ FLOPs to compute (since it contracts over the $N$ dimension). Likewise, the derivative of the loss with respect to \textbf{A} is

$$\frac{\partial L}{\partial A} = \frac{\partial L}{\partial C}\frac{\partial C}{\partial A} = \left(\frac{\partial L}{\partial C}\right) B^T$$

is again $2NPM$ FLOPs since \textbf{dL/dC} is a (co-)vector of size $[N, M]$. While this quantity isn't the derivative wrt. a parameter, it's used to compute derivatives for previous layers of the network (e.g. just as dL/dC is used to compute dL/dB above).

Adding these up, we see that \textbf{during training, we have a total of 6NPM FLOPs}, compared to 2NPM during inference: 2NPM in the forward pass, 4NPM in the backward pass. Since PM is the number of parameters in the matrix, this is the simplest form of the famous $6 * \text{num parameters} * \text{num tokens}$ approximation of Transformer FLOPs during training: each token requires $6 * \text{num parameters}$ FLOPs. We'll show a more correct derivation below.

\section{Transformer Accounting}

Transformers are the future. Well, they're the present at least. Maybe a few years ago, they were one of many architectures. But today, it's worth knowing pretty much every detail of the architecture. We won't reintroduce the architecture but \href{https://jalammar.github.io/illustrated-transformer/}{this blog} and the \href{https://arxiv.org/abs/1706.03762}{original Transformer paper} may be helpful references.

Here's a basic diagram of the Transformer decoder architecture:

\begin{figure}[htb]
\centering
\includegraphics[width=\textwidth]{images/transformer-diagram.png}
\caption{This diagram shows one layer of a standard Transformer and flows from top-to-bottom. We use a single-letter convention to describe the shapes and layouts of arrays in a Transformer, again showing contracting dimensions in red, and batched dimensions in blue. In a given operation, the input shape is given on top-left and the parameter shape is given on the top-right, with the resulting shape below, e.g. BTD is the input shape for the gating einsum and DF is the weight shape.}
\end{figure}

\textbf{Note [gating einsum]}: The diagram above uses a ``\href{https://arxiv.org/abs/2002.05202}{gating einsums}''~\cite{glu} where we split the up-projection matrix into two matrices ($W_\text{In1}$ and $W_\text{In2}$ above) whose outputs are elementwise multiplied as a kind of ``gating function''. Not all LLMs use this, so you will sometimes see a single $W_\text{In}$ matrix and a total MLP parameter count of 2DF instead of 3DF. Typically in this case, D and F will be scaled up to keep the parameter count the same as the 3 matrix case. With that said, some form of gating einsum is used by LLAMA, DeepSeek, and many other models.

\textbf{Note 2 [MHA attention]}: With self-attention, T and S are the same but for cross-attention they may be different. With vanilla Multi-Head Attention (MHA), N and K are the same while for \href{https://arxiv.org/abs/1911.02150}{Multi-Query Attention} (MQA)~\cite{mqa} K=1 and for \href{https://arxiv.org/abs/2305.13245}{Grouped MQA} (GMQA)~\cite{gmqa} K merely has to divide N.

\section{Global FLOPs and Params Calculation}

For the below we're going to compute per-layer FLOPs to avoid having to stick factors of \textbf{L} everywhere.

\subsection{MLPs}

The MLPs of a Transformer typically consist of 2 input matmuls that are element-wise combined and a single output matmul:

\begin{equation*}
\begin{array}{ccc}
\textrm{operation} & \textrm{train FLOPs} & \textrm{params} \\
\hline \\
A[B,T,\textcolor{red}{D}] \cdot W_{in1}[\textcolor{red}{D}, F] & 6BTDF & DF \\[10pt]
A[B,T,\textcolor{red}{D}] \cdot W_{in2}[\textcolor{red}{D}, F] & 6BTDF & DF \\[10pt]
\sigma\left(A_{in1}\right)[B,T, F] * A_{in2}[B,T, F] & \textcolor{gray}{O(BTF)} \\[10pt]
A[B,T,\textcolor{red}{F}] \cdot W_{out}[\textcolor{red}{F}, D] & 6BTDF & DF \\[10pt]
\hline \\
& \approx 18BTDF & 3DF
\end{array}
\end{equation*}

\subsection{Attention}

For the generic grouped-query attention case with different \textbf{Q} and \textbf{KV} head numbers, let us assume equal head dimension H for \textbf{Q},\textbf{K},\textbf{V} projections, and estimate the cost of the \textbf{QKVO} matmuls:

{\small
\begin{equation*}
\begin{array}{ccc}
\textrm{operation} & \textrm{train FLOPs} & \textrm{params} \\
\hline \\
A[B,T,\textcolor{red}{D}] \cdot W_{Q}[\textcolor{red}{D}, N, H] & 6BTDNH & DNH \\[10pt]
A[B,T,\textcolor{red}{D}] \cdot W_{K}[\textcolor{red}{D}, K, H] & 6BTDKH & DKH \\[10pt]
A[B,T,\textcolor{red}{D}] \cdot W_{V}[\textcolor{red}{D}, K, H] & 6BTDKH & DKH \\[10pt]
A[B,T,\textcolor{red}{N}, \textcolor{red}{H}] \cdot W_{O}[\textcolor{red}{N}, \textcolor{red}{H}, D] & 6BTDNH & DNH \\[10pt]
\hline \\ & 12BTD(N+K)H & 2D(N+K)H
\end{array}
\end{equation*}
}

The dot-product attention operation is more subtle, effectively being a $TH \cdot HS$ matmul batched over the $B$, $K$ dimensions, a softmax, and a $TS \cdot SH$ matmul again batched over the $B$, $K$ dimensions. We highlight the batched dims in blue:

\begin{equation*}
\begin{array}{cc}
\textrm{operation} & \textrm{train FLOPs} \\
\hline \\[3pt]
Q[\textcolor{blue}{B}, T, \textcolor{blue}{K}, G, \textcolor{red}{H}] \cdot K[\textcolor{blue}{B}, S, \textcolor{blue}{K}, \textcolor{red}{H}]
& 6BTSKGH = 6BTSNH  \\[3pt]
\textrm{softmax}_S \;\; L[B, T, S, K, G] & \textcolor{gray}{O(BTSKG) = O(BTSN)} \\[3pt]
S[\textcolor{blue}{B}, T, \textcolor{red}{S}, \textcolor{blue}{K}, G] \cdot V[\textcolor{blue}{B}, \textcolor{red}{S}, \textcolor{blue}{K}, H]
& 6BTSKGH = 6BTSNH \\[3pt]
\hline \\
& \approx 12BTSNH = 12BT^2NH \\
\end{array}
\end{equation*}

\subsection{Other Operations}

There are several other operations happening in a Transformer.  Layernorms are comparatively cheap and can be ignored for first-order cost estimates. There is also the final enormous (though not per-layer) unembedding matrix multiply.

\begin{equation*}
\begin{array}{ccc}
\textsf{operation} & \textsf{train FLOPs} & \textsf{params} \\
\hline \\
\textrm{layernorm}_D \;\; A[B,T,\textcolor{red}{D}] & \textcolor{gray}{O\left(BTD\right)} & \textcolor{gray}{D} \\[10pt]
A[B,T,\textcolor{red}{D}] \cdot W_{unembed}[\textcolor{red}{D}, V] & 6BTDV & DV \\
\end{array}
\end{equation*}

\subsection{General rule of thumb for Transformer FLOPs}

If we neglect the cost of dot-product attention for shorter-context training, then the total FLOPs across all layers is

\begin{align*}
&(18BTDF + 12BTD(N+K)H)L \\
&= 6 \times BT \times (3DF + 2D(N+K)H)L \\
&= 6 \times \textrm{num tokens} \times \textrm{parameter count}
\end{align*}

Leading to a famous rule of thumb for estimating dense Transformer FLOP count, ignoring the attention FLOPs. (Unembedding is another simple matmul with $6BSDV$ FLOPs and $DV$ params, and follows the same rule of thumb.)

\subsection{Fractional cost of attention with context length}

If we do account for dot-product attention above and assume $F=4D$, $D=NH$ (as is typical) and $N=K$:

{\small
\begin{align*}
\frac{\textrm{attention FLOPs}}{\textrm{matmul FLOPs}} &= \frac{12BT^2NH}{18BTDF + 24BTDNH} = \frac{12BT^2D}{4 \times 18 BTD^2 + 24 BTD^2} \\
&= \frac{12BT^2D}{96 BTD^2} = \frac{T}{8D}
\end{align*}
}

So the takeaway is that \textbf{dot-product attention FLOPs only become dominant during training once T>8D}. For D \textasciitilde{} 8k, this would be \textasciitilde{}64K tokens. This makes some sense, since it means as the MLP size increases, the attention FLOPs become less critical. For large models, the quadratic cost of attention is not actually a huge obstacle to longer context training. However, for smaller models, even e.g. Gemma-27B, D=4608 which means attention becomes dominant around 32k sequence lengths. Flash Attention also helps alleviate the cost of long-context, which we discuss briefly in Appendix A.

\section{Miscellaneous Math}

\subsection{Sparsity and Mixture-of-Experts}

We'd be remiss not to briefly discuss Mixture of Experts (MoE) models~\cite{moe}, which replace the single dense MLP blocks in a standard Transformer with a set of independent MLPs that can be dynamically routed between. To a first approximation, \textbf{an MoE is just a normal dense model with E MLP blocks per layer}, instead of just one. Each token activates $k$ of these experts, typically $k=2$. This increases the parameter count by $O(E)$, while multiplying the total number of activated parameters per token by $k$, compared with the dense version.

\begin{figure}[htb]
    \centering
    \includegraphics[width=\textwidth]{images/moe.png}
    \caption{An example MoE layer with $n$ experts. The gating expert routes each token to $k$ of them, and the output of those $k$ MLPs get summed. Our parameter count is $n$ times the size of each expert, but only $k$ are used for each token. \href{https://deepgram.com/learn/mixture-of-experts-ml-model-guide}{Source}.}
    \label{fig:moe}
\end{figure}

Compared to a dense model, an MoE introduces new comms, primarily two AllToAlls (one before and one after the MoE block) that route tokens to the correct expert and bring them back to their home device.\footnote{Technically, this only happens if we are data or sequence sharded along the same axis as our experts.} However as we saw in the previous section, the cost of each AllToAll is only 1/4 that of a comparable AllGather along a single axis (for a bidirectional ring).

\subsection{Gradient checkpointing}

Backpropagation as an algorithm trades memory for compute. Instead of a backward pass requiring $O(n_\text{layers}^2)$ FLOPs, \textbf{it requires $O(n_\text{layers})$ memory}, saving all intermediate activations generated during the forward pass. While this is better than quadratic compute, it's incredibly expensive memory-wise: a model with $B \cdot T=4M$ (4M total tokens per batch), L=64, and D=8192 that avoids all unnecessary backward pass compute would have to save roughly $2 \cdot 20 \cdot B \cdot T \cdot D \cdot L = 84TB$ of activations in bfloat16. 20 comes from (roughly) counting every intermediate node in the Transformer diagram above, since e.g.

\begin{equation}
f(x) = \exp(g(x))
\end{equation}

\begin{equation}
\frac{df}{dx} = \exp(g(x)) \cdot \frac{dg}{dx}
\end{equation}

so to avoid recomputing we need to save $g(x)$ and $\exp(g(x))$ from the forward pass. To avoid saving this much memory, we can choose to only save some fraction of the intermediate activations. Here are a few strategies we use.

\begin{itemize}
    \item \textbf{Block remat}: only save the input to each layer. This is the most aggressive method we use and only saves 1 checkpoint per layer, meaning we'd only save 4.2TB in the example above. This forces us to repeat essentially all forward pass FLOPs in the backward pass, meaning we increase our FLOPs from $6ND$ to roughly $8ND$.
    \item \textbf{Big matmuls only:} another simple policy is to only save the outputs of large matmuls. This lets us avoid recomputing any large matmuls during the backward pass, but still makes us recompute other activation functions and parts of attention. This reduces 20 per layer to closer to 7 per layer.
\end{itemize}

This by no means comprehensive. When using JAX, these are typically controlled by \texttt{jax.remat}/\texttt{jax.checkpoint} (you can read more \href{https://jax.readthedocs.io/en/latest/_autosummary/jax.checkpoint.html}{here}).

\subsection{Key-Value (KV) caching}

As we'll see in Section 7, LLM inference has two key parts, prefill and generation.

\begin{itemize}
    \item \textbf{Prefill} processes a long prompt and saves its attention activations in a Key-Value Cache (KV Cache) for use in generation, specifically the key-value projections in the attention block.
    \item \textbf{Generation} batches several of these KV caches together and samples tokens from each of them.
\end{itemize}

Each KV cache is then effectively an array of size $[2, S, L, K, H]$ where the 2 accounts for the keys and values. This is quite large! The total size of the Key-Value cache in int8 is $2SLKH$. For a moderately-sized model with 8k context length, 64 layers, and $KH = NH = D = 8192$, this is $2 \cdot 8192 \cdot 64 \cdot 8192 = 8\text{GiB}$. You can see why we would want to use GMQA with $K \ll N$.

\section{What Should You Take Away from this Section?}

\begin{itemize}
\item The overall parameters and FLOPs of a Transformer are fairly easy to calculate, and are summarized here, assuming MHA (with batch size B, vocab size V, a sequence of length T, D=$d_{\text{model}}$, and F=$d_{\text{ff}}$):
\end{itemize}

{\scriptsize
\setlength{\tabcolsep}{3pt}
\begin{table}[h]
\centering
\begin{tabular}{p{2cm}p{2.8cm}p{3.2cm}}
\toprule
\textbf{Component} & \textbf{Params per layer} & \textbf{Training FLOPs per layer} \\
\midrule
\textbf{MLP} & 3DF & 18BTDF \\
\textbf{Attention} & 4DNH & 24BTDNH + 12BT$^2$NH \\
\textbf{Other} & D & BTD \\
\textbf{Vocab} & DV (total, not per-layer) & 12BTDV \\
\bottomrule
\end{tabular}
\end{table}
}

\begin{itemize}
\item The parameter count of the MLP block dominates the total parameter count and the MLP block also dominates the FLOPs budget as long as the sequence length $T < 8D$.
\item The total FLOPs budget during training is well approximated by $6 \cdot \text{num\_params} \cdot \text{num\_tokens}$ for reasonable context lengths.
\item During inference, our KV caches are roughly $2 \cdot S \cdot L \cdot N \cdot H$ per cache, although architectural modifications can often reduce this.
\end{itemize}

\section{A Few Problems to Work}

\textbf{Question 1:} How many parameters does a model with $D=4096$, $F=4 \cdot D$, $V=32,000$, and $L=64$ have? What fraction of these are attention parameters? How large are our KV caches per token? \textit{You can assume $N\cdot H=D$ and multi-head attention with int8 KVs.}

\textbf{Question 2:} How many total FLOPs are required to perform A[B$_X$, D$_Y$] *$_D$ W[D$_Y$, F] on \texttt{\{'X': 4, 'Y': 8, 'Z': 4\}}. How many FLOPs are performed by each TPU?

\textbf{Question 3:} How many FLOPs are involved in performing $A[I,J,K,L] * B[I,J,M,N,O] \rightarrow C[K,L,M,N,O]$?

\textbf{Question 4:} What is the arithmetic intensity of self-attention (ignoring the Q/K/V/O projections)? \textit{Give the answer as a function of the Q and KV lengths T and S.} At what context length is attention FLOPs-bound? Given the HBM bandwidth of our TPUs, plot the effective relative cost of attention to the FFW block as the context length grows.

\textbf{Question 5:} At what sequence length are self-attention FLOPs equal to the QKVO projection FLOPs?

\textbf{Question 6:} Say we only save the output of each of the 7 main matmuls in a Transformer layer during our forward pass (Q, K, V, O + the three FFW matrices). How many extra FLOPs do we need to ``rematerialize'' during the backwards pass?

\textbf{Question 7:} DeepSeek v3 says it was trained for 2.79M H800 hours on 14.8T tokens (source). Given that it has 37B activated parameters, roughly what hardware utilization did they achieve? \textit{Hint: note that they used FP8 FLOPs without structured sparsity.}

\textbf{Question 8:} Mixture of Experts (MoE) models have $E$ copies of a standard dense MLP block, and each token activates $k$ of these experts. What batch size in tokens is required to be compute-bound for an MoE with weights in int8 on TPU v5e? For DeepSeek, which has 256 (routed) experts and $k=8$, what is this number?

\section{Appendix}

\subsection{Appendix A: How does Flash Attention work?}

The traditional objection to scaling Transformers to very long context is that the attention FLOPs and memory usage scale quadratically with context length. While it's true that the attention QK product has shape $[B, S, T, N]$ where B is the batch size, S and T are the Q and K sequence dims, and N is the number of heads, this claim comes with some serious caveats:

\begin{enumerate}
\item As we noted in Section 4, even though this is quadratic, the attention FLOPs only dominated when $S > 8 \cdot D$, and especially during training the memory of a single attention matrix is small compared to all of the weights and activation checkpoints living in memory, especially when sharded.
\item We don't need to materialize the full attention matrix in order to compute attention! We can compute local sums and maxes and avoid ever materializing more than a small chunk of the array. While the total FLOPs is still quadratic, we drastically reduce memory pressure.
\end{enumerate}

This second observation was first made by \href{https://arxiv.org/abs/2112.05682}{Rabe et al. 2021} and later in the \href{https://arxiv.org/abs/2205.14135}{Flash Attention paper} (Dao et al. 2022). The basic idea is to compute the attention in chunks of K/V, where we compute the local softmax and some auxiliary statistics, then pass them onto the next chunk which combines them with its local chunk. Specifically, we compute

\begin{enumerate}
\item \textbf{M:} The running max of $q \cdot k$ over the sequence dimension
\item \textbf{O:} The running full attention softmax over the sequence dimension
\item \textbf{L:} The running denominator $\sum_i (q \cdot k_i - \text{running max})$
\end{enumerate}

With these, we can compute the new max, the new running sum, and the new output with only a constant amount of memory. To give a sketchy description of how this works, attention is roughly this operation:

$$\text{Attn}(Q, K, V) = \sum_i \frac{\exp(Q \cdot K_i - \max_j Q \cdot K_j) V_i}{\sum_l \exp(Q \cdot K_l - \max_j Q \cdot K_j)}$$

The max is subtracted for numerical stability and can be added without affecting the outcome since $\sum_i \exp(a_i + b) = \exp(b) \sum \exp(a)$. Looking just at the denominator above,  if we imagine having two contiguous chunks of key vectors, $K^1$ and $K^2$ and we compute the local softmax sums $L^1$ and $L^2$ for each

$$L^1 = \sum_i \exp(Q \cdot K_i^1 - \max_j Q \cdot K_j^1)$$

$$L^2 = \sum_i \exp(Q \cdot K_i^2 - \max_j Q \cdot K_j^2)$$

Then we can combine these into the full softmax sum for these two chunks together by using

\begin{align*}
L^\text{combined} &= \exp(M^1 - \max(M^1, M^2)) \cdot L^1 \\
&\quad + \exp(M^2 - \max(M^1, M^2)) \cdot L^2
\end{align*}

where

$$M^1 = \max_j Q \cdot K_j^1 \text{ and } M^2 = \max_j Q \cdot K_j^2$$

This can be done for the full softmax as well, giving us a way of accumulating arbitrarily large softmax sums. Here's the full algorithm from the Flash Attention paper.

\begin{figure}[htb]
\centering
\includegraphics[width=\textwidth]{images/flash-algo.png}
\caption{Flash Attention algorithm}
\end{figure}

From a hardware standpoint, this lets us fit our chunk of Q into VMEM (what the algorithm above calls on-chip SRAM) so we only have to load the KV chunks on each iteration, reducing the arithmetic intensity. We can also keep the running statistics in VMEM.

One last subtle point worth emphasizing is an attention softmax property that's used to make the Flash VJP (reverse mode derivative) calculation practical for training.  If we define an intermediate softmax array as:

$$S_{ij} = \frac{e^{\tau q_i \cdot k_j}}{\sum_k e^{\tau q_i \cdot k_j}}$$

In attention, we obtain \textit{dS} from reverse-mode \textit{dO} and \textit{V} arrays:

$$dS_{ij} = dO_{id} \cdot_d V_{jd} = \sum_d dO_{id} V_{jd}$$

During the backpropagation of this gradient to Q and K

$$d(q_i \cdot k_j) = (dS_{ij} - S_{ij} \cdot_j dS_{ij}) S_{ij}$$

We exploit an identity that allows us to exchange a contraction along the large key \textbf{length} dimension with a local contraction along the feature \textbf{depth} dimension.

\begin{align*}
S_{ij} \cdot_j dS_{ij} &= \sum_j \frac{e^{\tau q_i \cdot k_j}}{\sum_k e^{\tau q_i \cdot k_k}} \sum_d dO_{id} V_{jd} \\
&= \sum_d dO_{id} \sum_j \frac{e^{\tau q_i \cdot k_j}}{\sum_k e^{\tau q_i \cdot k_k}} V_{jd} \\
&= \sum_d dO_{id} O_{id} \\
&= dO_{id} \cdot_d O_{id}
\end{align*}

This replacement is crucial for being able to implement a sequence-block \textit{local} calculation for the VJP, and enables further clever sharding schemes like ring attention.


\chapter{How to Parallelize a Transformer for Training}
\label{chap:training}

\textit{Our goal in this section is to apply results from the previous section to a very practical problem: training the LLaMA 3 family (herd) of models. Unlike the previous sections we want you to do a lot of this work yourself. For this reason, we've hidden the answers to each section so you can try to answer it first. Try grabbing a pen and doing by hand!}

\section{What does LLaMA 3 look like?}

The LLaMA-3 model family~\cite{llama3} includes 3 main models: LLaMA 3 8B, 70B, and 405B. We'll mostly focus on 70B, and leave 8B and 405B for you to explore in the problem section at the end. Here's the architecture for LLaMA 3-70B, taken from the LLaMA \href{https://huggingface.co/meta-llama/Meta-Llama-3-70B/blob/main/config.json}{HuggingFace page}.

{\scriptsize
\begin{longtable}{p{5cm} p{3cm}}
\toprule
\textbf{hyperparam} & \textbf{value} \\
\midrule
\endfirsthead
\toprule
\textbf{hyperparam} & \textbf{value} \\
\midrule
\endhead
\midrule
\multicolumn{2}{r}{\textit{Continued on next page}} \\
\endfoot
\bottomrule
\endlastfoot
$n_\text{layers}$ (L) & 80 \\
$d_\text{model}$ (D) & 8,192 \\
$d_{ff}$ (F) & 28,672 \\
$n_\text{heads}$ (N) & 64 \\
$n_\text{kv\_heads}$ (K) & 8 \\
$d_\text{qkv}$ (H) & 128 \\
$n_\text{embeddings}$ (V) & 128,256 \\
\end{longtable}
}

To highlight how easy this is to find, here's the config itself, along with a mapping:

\begin{figure}[htb]
\centering
\includegraphics[width=\textwidth]{images/llama-json.png}
\caption{LLaMA 3 configuration file showing the architecture parameters and their mapping to the hyperparameters listed above~.}
\label{fig:llama-json}
\end{figure}

\textit{It's useful to make a big table with these numbers for many different open-source LLMs, so you can quickly compare the design decisions they've made.}

\subsection{Data Parallelism}

\textbf{Syntax:} $\text{In}[B_X, D] \cdot_D W_\text{in}[D, F] \cdot_F W_\text{out}[F, D] \rightarrow \text{Out}[B_X, D]$

When your model fits on a single chip with even a tiny batch size (>240 tokens, so as to be compute-bound), \textbf{you should always use simple data parallelism.} Pure data parallelism splits our activations across any number of TPUs so long as the number of TPUs is smaller than our batch size. The forward pass involves no communication, but at the end of every step, \textbf{each TPU performs an AllReduce on its local gradients to synchronize them before updating the parameters.}

\begin{figure}[htb]
\centering
\includegraphics[width=\textwidth]{images/data-parallelism.png}
\caption{Pure data parallelism (forward pass). Our activations (left) are fully sharded along the batch dimension and our weights are fully replicated, so each TPU has an identical copy of the weights. This means the total memory of our weights is increased by a factor of N, but no communication is required on the forward-pass.}
\end{figure}

\begin{algorithmbox}
\textbf{Pure Data Parallelism Algorithm:}

\textbf{Forward pass:} need to compute Loss[$B_X$]

\begin{enumerate}
    \item Tmp[$B_X$, F] = In[$B_X$, D] $*_D$ $W_{\text{in}}$[D, F]
    \item Out[$B_X$, D] = Tmp[$B_X$, F] $*_F$ $W_{\text{out}}$[F, D]
    \item Loss[$B_X$] = ...
\end{enumerate}

\textbf{Backward pass:} need to compute $\text{dW}_{\text{out}}$[F, D], $\text{dW}_{\text{in}}$[D, F]

\begin{enumerate}
    \item dOut[$B_X$, D] = ...
    \item $\text{dW}_{\text{out}}$[F, D] \{$U_X$\} = Tmp[$B_X$, F] $*_B$ dOut[$B_X$, D]
    \item $\text{dW}_{\text{out}}$[F, D] = \textbf{AllReduce}($\text{dW}_{\text{out}}$[F, D] \{$U_X$\}) (\textit{not on critical path, can be done async})
    \item dTmp[$B_X$, F] = dOut[$B_X$, D] $*_D$ $W_{\text{out}}$[F, D]
    \item $\text{dW}_{\text{in}}$[D, F] \{$U_X$\} = In[$B_X$, D] $*_B$ dTmp[$B_X$, F]
    \item $\text{dW}_{\text{in}}$[D, F] = \textbf{AllReduce}($\text{dW}_{\text{in}}$[D, F] \{$U_X$\}) (\textit{not on critical path, can be done async})
    \item dIn[$B_X$, D] = dTmp[$B_X$, F] $*_F$ $W_{\text{in}}$[D, F] (\textit{needed for previous layers})
\end{enumerate}
\end{algorithmbox}

We ignore the details of the loss function and abbreviate $\text{Tmp} = W_\text{in} \cdot \text{In}$.~ Note that, although our final loss is the average \textbf{AllReduce}(Loss[$B_X$]), we only need to compute the AllReduce on the backward pass when averaging weight gradients.

Note that the forward pass has no communication --- \textbf{it's all in the backward pass}! The backward pass also has the great property that the AllReduces aren't in the ``critical path'', meaning that each AllReduce can be performed whenever it's convenient and doesn't block you from performing subsequent operations. The overall communication cost \textit{can still bottleneck us} if it exceeds our total compute cost, but it is much more forgiving from an implementation standpoint. We'll see that model/tensor parallelism doesn't have this property.

\textbf{Why do this?} Pure data parallelism reduces activation memory pressure by splitting our activations over the batch dimension, allowing us to almost arbitrarily increase batch size as long as we have more chips to split the batch dimension over. Especially during training when our activations often dominate our memory usage, this is very helpful.

\textbf{Why not do this?} Pure data parallelism does nothing to reduce memory pressure from model parameters or optimizer states, which means pure data parallelism is rarely useful for interesting models at scale where our parameters + optimizer state don't fit in a single TPU. To give a sense of scale, if we train with parameters in bf16 and optimizer state in fp32 with Adam\footnote{Adam stores parameters, first order and second order accumulators. Since the params are in bfloat16 and optimizer state is in float32, this gives us \texttt{2 + 8 = 10} bytes per parameters.}, the largest model we can fit has $\text{TPU memory} / 10$ parameters, so e.g. on a TPUv5p chip with 96GB of HBM and pure data parallelism this is about 9B parameters.

\begin{takeawaybox}
The largest model we can train with Adam and pure data parallelism has $\text{num\_params} = \text{HBM per device} / 10$.~ For TPU v5p this is roughly 9B parameters.\footnote{Note that this doesn't include gradient checkpoints, so this wouldn't actually be useful. This is an absolute lower bound with a batch of 1 token.}
\end{takeawaybox}

\textit{To make this useful for real models during training, we'll need to at least partly shard the model parameters or optimizer.}

\textbf{When do we become bottlenecked by communication?} As we can see above, we have two AllReduces per layer, each of size $2DF$ (for bf16 weights). When does data parallelism make us communication bound?

As in the table above, let $C$ = per-chip FLOPs, $W_{\text{ici}}$ = \textbf{bidirectional} network bandwidth, and $X$ = number of shards across which the batch is partitioned\footnote{We assume this partitioning is done over an ICI mesh, so the relevant network bandwidth is $W_\text{ici}$}.~ Let's calculate the time required to perform the relevant matmuls, $T_\text{math}$,~ and the required communication time $T_\text{comms}$.~ Since this parallelism scheme requires no communication in the forward pass, we only need to calculate these quantities for the backwards pass.

\textit{Communication time:} From a previous section we know that the time required to perform an AllReduce in a 1D mesh depends only on the total bytes of the array being AllReduced and the ICI bandwidth $W_\text{ici}$;~ specifically the AllReduce time is $2 \cdot \text{total bytes} / W_\text{ici}$.~ Since we need to AllReduce for both $W_\text{in}$ and $W_\text{out}$,~ we have 2 AllReduces per layer. Each AllReduce is for a weight matrix, i.e. an array of $DF$ parameters, or $2DF$ bytes. Putting this all together, the total time for the AllReduce in a single layer is

\begin{align}
T_\text{comms} &= \frac{2 \cdot 2 \cdot 2 \cdot D \cdot F}{W_\text{ici}}.
\end{align}

\textit{Matmul time:} Each layer comprises two matmuls in the forward pass, or four matmuls in the backwards pass, each of which requires $2(B/X)DF$ FLOPs. Thus, for a single layer in the backward pass, we have

\begin{align}
T_\text{math} &= \frac{2 \cdot 2 \cdot 2 \cdot B \cdot D \cdot F}{X \cdot C}
\end{align}

Since we overlap, the total time per layer is the max of these two quantities:

\begin{align*}
T &\approx \max(\frac{8 \cdot B \cdot D \cdot F}{X \cdot C}, \frac{8 \cdot D \cdot F}{W_\text{ici}}) \\
T &\approx 8 \cdot D \cdot F \cdot \max(\frac{B}{X \cdot C}, \frac{1}{W_\text{ici}})
\end{align*}

We become compute-bound when $T_\text{math}/T_\text{comms} > 1$,~ or when

\begin{align}
\frac{B}{X} > \frac{C}{W_\text{ici}}.
\end{align}

The upshot is that, to remain compute-bound with data parallelism, we need the per-device batch size $B / X$ to exceed the ICI operational intensity, $C / W_\text{ici}$.~ This is ultimately a consequence of the fact that the computation time scales with the per-device batch size, while the bandwidth-bound time is independent of this quantity (since we are transferring model weights). Note the resemblance of the $B > C/W_\text{ici}$ condition to the single-device compute-bound rule $B > 240$;~ in that case as well, the rule came from the fact that computation time scaled with batch size while data-transfer size was (in the $B \ll F, D$ regime) independent of batch size.

Let's put in some real numbers to get a sense of scale. For TPUv5p, \texttt{C=4.6e14} and \texttt{W=2 * 9e10} for 1D data parallelism over ICI, so \textbf{our batch size per chip must be at least 2,550 to avoid being communication-bound}. Since we can do data parallelism over multiple axes, if we dedicate all three axes of a TPUv5p pod to pure data parallelism, we 3x our bandwidth $W_\text{ici}$ and can scale down to only BS=850 per TPU or 7.6M tokens per batch per pod (of 8960 chips)! \textbf{This tells us that it's fairly hard to become bottlenecked by pure data parallelism!}

\begin{takeawaybox}
\textbf{Note [context parallelism]:} Throughout this section, $B$ always refers to the total batch size \textbf{in tokens}. Clearly, however, our batch is made up of many different sequences, so how does this work? As far as the MLP is concerned, \textbf{tokens are tokens}! It doesn't matter if they belong to the same sequence or two different sequences. So we are more or less free to do data parallelism over both the batch and sequence dimension: we call this context parallelism or sequence parallelism, but you can think of it as simply being another kind of data parallelism. Attention is trickier than the MLP since we do some cross-sequence computation, but this can be handled by gathering KVs or Qs during attention and carefully overlapping FLOPs and comms (typically using something called ``ring attention''). Throughout this section, we will just ignore our sequence dimension entirely and assume some amount of batch or sequence parallelism.
\end{takeawaybox}

\textbf{Note on multiple mesh axes:} We should quickly note how multiple axes affects the available bandwidth. When we use multiple mesh axes for a given parallelism strategy, we get more bandwidth.

\begin{itemize}
    \item \textbf{Definition:} $M_X$ ($M_Y$, $M_Z$, etc.) is the number of hardware mesh axes that a given parallelism strategy spans.
    \item \textbf{Effect (bandwidth-bound):} Using $M$ axes provides ($\approx M$ times) aggregate link bandwidth, so collective time scales $\propto 1/M_X$.~
\end{itemize}

\subsection{Fully-Sharded Data Parallelism (FSDP)}

\textbf{Syntax:} $\text{In}[B_X, D] \cdot_D W_\text{in}[D_X, F] \cdot_F W_\text{out}[F, D_X] \rightarrow \text{Out}[B_X, D]$

Fully-sharded data parallelism (often called FSDP or ZeRO-sharding~\cite{zero}) splits the model optimizer states and weights across the data parallel shards and efficiently gathers and scatters them as needed. \textbf{Compared to pure data parallelism, FSDP drastically reduces per-device memory usage and saves on backward pass FLOPs, with very minimal overhead.}

\begin{figure}[htb]
\centering
\includegraphics[width=\textwidth]{images/fsdp.png}
\caption{FSDP shards the contracting dimension of $W_{\text{in}}$ and the output dimension of $W_{\text{out}}$ along the data dimension. This reduces memory but (from Section~3) requires us to gather the weights for $W$ before we perform the matmul. Note that the activations (left) \textit{are not sharded along the contracting dimension}, which is what forces us to gather. \textbf{Note that our weight optimizer state is likewise sharded along the contracting dimension.}}
\end{figure}

You'll remember (from Section 3) that an AllReduce can be decomposed into an AllGather and a ReduceScatter. This means that, instead of doing the full gradient AllReduce for standard data parallelism, we can shard the weights and optimizer states across chips, AllGather them at each layer during the forward pass and ReduceScatter across the weights during the backward pass at no extra cost.

\begin{algorithmbox}
\textbf{Fully-Sharded Data Parallelism (FSDP):}

\textbf{Forward pass:} need to compute Loss[$B_X$]

\begin{enumerate}
    \item $W_{\text{in}}$[D, F] = \textbf{AllGather}($W_{\text{in}}$[$D_X$, F]) (\textit{not on critical path, can do it during previous layer})
    \item Tmp[$B_X$, F] = In[$B_X$, D] $*_D$ $W_{\text{in}}$[D, F] (\textit{can throw away $W_{\text{in}}$[D, F] now})
    \item $W_{\text{out}}$[F, D] = \textbf{AllGather}($W_{\text{out}}$[F, $D_X$]) (\textit{not on critical path, can do it during previous layer})
    \item Out[$B_X$, D] = Tmp[$B_X$, F] $*_F$ $W_{\text{out}}$[F, D]
    \item Loss[$B_X$] = ...
\end{enumerate}

\textbf{Backward pass:} need to compute $\text{dW}_{\text{out}}$[F, $D_X$], $\text{dW}_{\text{in}}$[$D_X$, F]

\begin{enumerate}
    \item dOut[$B_X$, D] = ...
    \item $\text{dW}_{\text{out}}$[F, D] \{$U_X$\} = Tmp[$B_X$, F] $*_B$ dOut[$B_X$, D]
    \item $\text{dW}_{\text{out}}$[F, $D_X$] = \textbf{ReduceScatter}($\text{dW}_{\text{out}}$[F, D] \{$U_X$\}) (\textit{not on critical path, can be done async})
    \item $W_{\text{out}}$[F, D] = \textbf{AllGather}($W_{\text{out}}$[F, $D_X$]) (\textit{can be done ahead of time})
    \item dTmp[$B_X$, F] = dOut[$B_X$, D] $*_D$ $W_{\text{out}}$[F, D] (\textit{can throw away $W_{\text{out}}$[F, D] here})
    \item $\text{dW}_{\text{in}}$[D, F] \{$U_X$\} = dTmp[$B_X$, F] $*_B$ In[$B_X$, D]
    \item $\text{dW}_{\text{in}}$[$D_X$, F] = \textbf{ReduceScatter}($\text{dW}_{\text{in}}$[D, F] \{$U_X$\}) (\textit{not on critical path, can be done async})
    \item $W_{\text{in}}$[D, F] = \textbf{AllGather}($W_{\text{in}}$[$D_X$, F]) (\textit{can be done ahead of time})
    \item dIn[$B_X$, D] = dTmp[$B_X$, F] $*_F$ $W_{\text{in}}$[D, F] (\textit{needed for previous layers}) (\textit{can throw away $W_{\text{in}}$[D, F] here})
\end{enumerate}
\end{algorithmbox}

This is also called ``ZeRO Sharding'', from ``ZeRo Overhead sharding'' since we don't perform any unnecessary compute or store any unnecessary state. ZeRO-\{1,2,3\} are used to refer to sharding the optimizer states, gradients, and weights in this way, respectively. Since all have the same communication cost\footnote{Technically, FSDP adds communication in the forward pass that pure DP doesn't have, but this is in the same proportion as the backward pass so it should have no effect on the comms roofline. The key here is that ZeRO-3 turns a backward-pass AllReduce into an AllGather and a ReduceScatter, which have the same total comms volume.}, we can basically always do ZeRO-3 sharding, which shards the parameters, gradients, and optimizer states across a set of devices.

\textbf{Why would we do this?} Standard data parallelism involves a lot of duplicated work. Each TPU AllReduces the full gradient, then updates the full optimizer state (identical work on all TPUs), then updates the parameters (again, fully duplicated). For ZeRO sharding (sharding the gradients/optimizer state), instead of an AllReduce, you can ReduceScatter the gradients, update only your shard of the optimizer state, update a shard of the parameters, then AllGather the parameters as needed for your forward pass.

\textbf{When do we become bottlenecked by communication?} Our relative FLOPs and comms costs are exactly the same as pure data parallelism, since each AllReduce in the backward pass has become an AllGather + ReduceScatter. Recall that an AllReduce is implemented as an AllGather and a ReduceScatter, each with half the cost. Here we model the forward pass since it has the same FLOPs-to-comms ratio as the backward pass:

$$\begin{aligned}
T_\text{math} &= \frac{2 \cdot 2 \cdot B \cdot D \cdot F}{X \cdot C} \\
T_\text{comms} &= \frac{2 \cdot 2 \cdot D \cdot F}{W_\text{ici}} \\
T &\approx \max\left(\frac{4 \cdot B \cdot D \cdot F}{X \cdot C}, \frac{4 \cdot D \cdot F}{W_\text{ici}}\right) \\
T &\approx 4 \cdot D \cdot F \cdot \max\left(\frac{B}{X \cdot C}, \frac{1}{W_\text{ici}}\right)
\end{aligned}$$

Therefore, as with pure data-parallelism, we are compute bound when $B / X > C / W_\text{ici}$, i.e. when the per-device batch size $B/X$ exceeds the ``ICI operational intensity'' $C/W_\text{ici}$ (\texttt{4.59e14 / 1.8e11 = 2550} for v5p). This is great for us, because it means if our per-device batch size is big enough to be compute-bound for pure data-parallelism, we can --- without worrying about leaving the compute-bound regime --- simply upgrade to FSDP, saving ourselves a massive amount of parameter and optimizer state memory!~ Though we did have to add communication to the forward pass, this cost is immaterial since it just overlaps with forward-pass FLOPs.

\begin{takeawaybox}
Both FSDP and pure Data Parallelism become bandwidth-bound on TPUv5 when the per-device batch size is less than $2550 / M_X$, where $M_X$ is the number of mesh axes.
\end{takeawaybox}

For example, DeepSeek-V2 (one of the only recent strong model to release information about its training batch size) used a batch size of \textasciitilde40M tokens. \textbf{This would allow us to scale to roughly 47,000 chips, or around 5 TPUv5 pods, before we hit a bandwidth limit.}

For LLaMA-3 70B, which was trained for approximately \texttt{6.3e24 (15e12 * 70e9 * 6)} FLOPs, we could split a batch of 16M tokens over roughly \texttt{16e6 / (2550 / 3) = 18,823} chips (roughly 2 pods of 8960 chips), each with \texttt{4.59e14} FLOPs running at 50\% peak FLOPs utilization (often called MFU), and \textbf{train it in approximately 17 days}. Not bad! But let's explore how we can do better.

\begin{takeawaybox}
\textbf{Note on critical batch size:} Somewhat unintuitively, we become more communication bottlenecked as our total batch size decreases (with fixed chip number). Data parallelism and FSDP let us scale to arbitrarily many chips so long as we can keep increasing our batch size! However, in practice, as our batch size increases, we tend to see diminishing returns in training since our gradients become almost noise-free. We also sometimes see training instability. Thus, the game of finding an optimal sharding scheme in the ``unlimited compute regime'' often starts from a fixed batch size, determined by scaling laws, and a known (large) number of chips, and then aims to find a partitioning that allows us to fit that small batch size on so many chips.
\end{takeawaybox}

\subsection{Tensor Parallelism}

\textbf{Syntax:} $\text{In}[B, D_Y] \cdot_D W_\text{in}[D, F_Y] \cdot_F W_\text{out}[F_Y, D] \rightarrow \text{Out}[B, D_Y]$ (we use $Y$ to eventually combine with FSDP)

In a fully-sharded data-parallel AllReduce we move the weights across chips. We can also shard the feedforward dimension of the model and move the activations during the layer~--- this is called ``1D model parallelism'' or Megatron sharding~\cite{megatron}. This can unlock a smaller efficient batch size per pod. The figure below shows an example of a single matrix sharded in this way:

\begin{figure}[htb]
    \centering
    \includegraphics[width=\textwidth]{images/model-parallelism.png}
    \caption{An example of basic tensor parallelism. Since we're only sharding our activations over Y (unlike in FSDP where we shard over X), we replicate our activations over X. Using our standard syntax, this is $\textbf{A}[B, D_Y] \times \textbf{B}[D, F_Y] \rightarrow \textbf{C}[B, F_Y]$. Because we're only sharding over one of the contracting dimensions, we typically AllGather the activations $\textbf{A}$ before the matmul.}
    \label{fig:model-parallelism}
\end{figure}

As noted, $\textbf{In}[B, D_Y] \times_D W_{\text{in}}[D, F_Y] \times_F W_{\text{out}}[F_Y, D] \rightarrow \textbf{Out}[B, D_Y]$ means we have to gather our activations before the first matmul. This is cheaper than ZeRO sharding when the activations are smaller than the weights. This is typically true only with some amount of ZeRO sharding added (which reduces the size of the gather). This is one of the reasons we tend to mix ZeRO sharding and tensor parallelism.

\begin{tcolorbox}[algorithmbox, title=Algorithm: Tensor Parallelism]

\textbf{Forward pass:} need to compute Loss[B]

\begin{enumerate}
    \item In[B, D] = \textbf{AllGather}(In[B, D\textsubscript{Y}]) \textit{(on critical path)}
    \item Tmp[B, F\textsubscript{Y}] = In[B, D] *\textsubscript{D} W\textsubscript{in}[D, F\textsubscript{Y}] \textit{(not sharded along contracting, so no comms)}
    \item Out[B, D] \{U\textsubscript{Y}\} = Tmp[B, F\textsubscript{Y}] *\textsubscript{F} W\textsubscript{out}[F\textsubscript{Y}, D]
    \item Out[B, D\textsubscript{Y}] = \textbf{ReduceScatter}(Out[B, D] \{U\textsubscript{Y}\}) \textit{(on critical path)}
    \item Loss[B] = ...
\end{enumerate}

\textbf{Backward pass:} need to compute dW\textsubscript{out}[F\textsubscript{Y}, D], dW\textsubscript{in}[D, F\textsubscript{Y}]

\begin{enumerate}
    \item dOut[B, D\textsubscript{Y}] = ...
    \item dOut[B, D] = \textbf{AllGather}(dOut[B, D\textsubscript{Y}]) \textit{(on critical path)}
    \item dW\textsubscript{out}[F\textsubscript{Y}, D] = Tmp[B, F\textsubscript{Y}] *\textsubscript{B} dOut[B, D]
    \item dTmp[B, F\textsubscript{Y}] = dOut[B, D] *\textsubscript{D} W\textsubscript{out}[F\textsubscript{Y}, D] \textit{(can throw away dOut[B, D] here)}
    \item In[B, D] = \textbf{AllGather}(In[B, D\textsubscript{Y}]) \textit{(this can be skipped by sharing with (1) from the forward pass)}
    \item dW\textsubscript{in}[D, F\textsubscript{Y}] = dTmp[B, F\textsubscript{Y}] *\textsubscript{B} In[B, D]
    \item dIn[B, D] \{U.Y\} = dTmp[B, F\textsubscript{Y}] *\textsubscript{F} W\textsubscript{in}[D, F\textsubscript{Y}] \textit{(needed for previous layers)}
    \item dIn[B, D\textsubscript{Y}] = \textbf{ReduceScatter}(dIn[B, D] \{U.Y\}) \textit{(on critical path)}
\end{enumerate}

\end{tcolorbox}

One nice thing about tensor parallelism is that it interacts nicely with the two matrices in our Transformer forward pass. Naively, we would do an AllReduce after each of the two matrices. But here we first do $\textbf{In}[B, D_Y] \times W_{\text{in}}[D, F_Y] \rightarrow \textbf{Tmp}[B, F_Y]$ and then $\textbf{Tmp}[B, F_Y] \times W_{\text{out}}[F_Y, D] \rightarrow \textbf{Out}[B, D_Y]$. This means we AllGather $\textbf{In}$ at the beginning, and ReduceScatter $\textbf{Out}$ at the end, rather than doing an AllReduce.

\textbf{How costly is this?} Let's only model the forward pass~- the backwards pass is just the transpose of each operation here. In 1D tensor parallelism we AllGather the activations before the first matmul, and ReduceScatter them after the second, sending two bytes at a time (bf16). Let's figure out when we're bottlenecked by communication.

\begin{align}
T_\text{math} & = \frac{4 \cdot B \cdot D \cdot F}{Y \cdot C} \\
T_\text{comms} & =
\frac{2 \cdot 2 \cdot (B \cdot D)}{W_\text{ici}}\\
\textnormal{T} & \approx \max \left(\frac{4 \cdot B \cdot D \cdot F}{Y \cdot C}, \frac{2 \cdot 2 \cdot (B \cdot D)}{W_\text{ici}}\right)
\end{align}

Noting that we want compute cost to be greater than comms cost, we get:

\begin{align}
\frac{4 \cdot B \cdot D \cdot F}{Y \cdot C} > \frac{2 \cdot 2 \cdot (B \cdot D)}{W_\text{ici}}
\end{align}

\begin{align}
\frac{F}{Y \cdot C} > \frac{1}{W_\text{ici}}
\end{align}

\begin{align}
F > Y \cdot \frac{C}{W_\text{ici}}
\end{align}

Thus for instance, for TPUv5p, $C / W_{ici} = 2550$ in bf16, so we can only do tensor parallelism up to $Y < F / 2550$. When we have multiple ICI axes, our $T_\text{comms}$ is reduced by a factor of $M_Y$, so we get $Y < M_Y \cdot F / 2550$.

\begin{tcolorbox}[takeawaybox]
Tensor Parallelism becomes communication bound when $Y > M_Y \cdot F / 2550$. For most models this is between 8 and 16-way tensor parallelism.
\end{tcolorbox}

\textbf{Note that this doesn't depend on the precision of the computation}, since e.g.\ for int8, on TPUv5p, $C_\text{int8} / W_{ici}$ is $5100$ instead of $2550$ but the comms volume is also halved, so the two factors of two cancel.

\textbf{Let's think about some examples:}

\begin{itemize}
    \item On TPUv5p with LLaMA 3-70B with $D = 8192,$ $F \approx 30,000$, we can comfortably do 8-way tensor parallelism, but will be communication bound on 16 way tensor parallelism. The required F for model 8 way model sharding is 20k.

    \item For Gemma 7B, $F \approx 50k$, so we become communication bound with 19-way tensor parallelism. That means we could likely do 16-way and still see good performance.
\end{itemize}

\subsection{Combining FSDP and Tensor Parallelism}

\textbf{Syntax:} $\text{In}[B_X, D_Y] \cdot_D W_\text{in}[D_X, F_Y] \cdot_F W_\text{out}[F_Y, D_X] \rightarrow \text{Out}[B_X, D_Y]$

The nice thing about FSDP and tensor parallelism is that they can be combined. By sharding \textbf{W\textsubscript{in}} and \textbf{W\textsubscript{out}} along both axes we both save memory and compute. Because we shard $B$ along $X$, we reduce the size of the model-parallel AllGathers, and because we shard $F$ along $Y$, we reduce the communication overhead of FSDP. This means a combination of the two can get us to an even lower effective batch size than we saw above.

\begin{figure}[htb]
    \centering
    \includegraphics[width=\textwidth]{images/mixed-fsdp-model-parallelism.png}
    \caption{A diagram combining FSDP and tensor parallelism. Unlike the other cases, there is no duplication of model parameters.}
    \label{fig:mixed-fsdp-model-parallelism}
\end{figure}

{\footnotesize
\begin{algorithmbox}

\textbf{Algorithm: Combining FSDP and Tensor Parallelism}

\textbf{Forward pass:} need to compute Loss[B]

\begin{enumerate}
    \item In[B\textsubscript{X}, D] = \textbf{AllGather}\textsubscript{Y}(In[B\textsubscript{X}, D\textsubscript{Y}]) \textit{(on critical path)}
    \item W\textsubscript{in}[D, F\textsubscript{Y}] = \textbf{AllGather}\textsubscript{X}(W\textsubscript{in}[D\textsubscript{X}, F\textsubscript{Y}]) \textit{(can be done ahead of time)}
    \item Tmp[B\textsubscript{X}, F\textsubscript{Y}] = In[B\textsubscript{X}, D] *\textsubscript{D} W\textsubscript{in}[D, F\textsubscript{Y}]
    \item W\textsubscript{out}[F\textsubscript{Y}, D] = \textbf{AllGather}\textsubscript{X}(W\textsubscript{out}[F\textsubscript{Y}, D\textsubscript{X}]) \textit{(can be done ahead of time)}
    \item Out[B\textsubscript{X}, D] \{U.Y\} = Tmp[B\textsubscript{X}, F\textsubscript{Y}] *\textsubscript{F} W\textsubscript{out}[F\textsubscript{Y}, D]
    \item Out[B\textsubscript{X}, D\textsubscript{Y}] = \textbf{ReduceScatter}\textsubscript{Y}(Out[B\textsubscript{X}, D] \{U.Y\}) \textit{(on critical path)}
    \item Loss[B\textsubscript{X}] = ...
\end{enumerate}

\textbf{Backward pass:} need to compute dW\textsubscript{out}[F\textsubscript{Y}, D\textsubscript{X}], dW\textsubscript{in}[D\textsubscript{X}, F\textsubscript{Y}]

\begin{enumerate}
    \item dOut[B\textsubscript{X}, D\textsubscript{Y}] = ...
    \item dOut[B\textsubscript{X}, D] = \textbf{AllGather}\textsubscript{Y}(dOut[B\textsubscript{X}, D\textsubscript{Y}]) \textit{(on critical path)}
    \item dW\textsubscript{out}[F\textsubscript{Y}, D] \{U.X\} = Tmp[B\textsubscript{X}, F\textsubscript{Y}] *\textsubscript{B} dOut[B\textsubscript{X}, D]
    \item dW\textsubscript{out}[F\textsubscript{Y}, D\textsubscript{X}] = \textbf{ReduceScatter}\textsubscript{X}(dW\textsubscript{out}[F\textsubscript{Y}, D] \{U.X\})
    \item W\textsubscript{out}[F\textsubscript{Y}, D] = \textbf{AllGather}\textsubscript{X}(W\textsubscript{out}[F\textsubscript{Y}, D\textsubscript{X}]) \textit{(can be done ahead of time)}
    \item dTmp[B\textsubscript{X}, F\textsubscript{Y}] = dOut[B\textsubscript{X}, D] *\textsubscript{D} W\textsubscript{out}[F\textsubscript{Y}, D] \textit{(can throw away dOut[B, D] here)}
    \item In[B\textsubscript{X}, D] = \textbf{AllGather}\textsubscript{Y}(In[B\textsubscript{X}, D\textsubscript{Y}]) \textit{(not on critical path + this can be shared with (2) from the previous layer)}
    \item dW\textsubscript{in}[D, F\textsubscript{Y}] \{U.X\} = dTmp[B\textsubscript{X}, F\textsubscript{Y}] *\textsubscript{B} In[B\textsubscript{X}, D]
    \item dW\textsubscript{in}[D\textsubscript{X}, F\textsubscript{Y}] = \textbf{ReduceScatter}\textsubscript{X}(dW\textsubscript{in}[D, F\textsubscript{Y}] \{U.X\})
    \item W\textsubscript{in}[D, F\textsubscript{Y}] = \textbf{AllGather}\textsubscript{X}(W\textsubscript{in}[D\textsubscript{X}, F\textsubscript{Y}]) \textit{(can be done ahead of time)}
    \item dIn[B\textsubscript{X}, D] \{U.Y\} = dTmp[B\textsubscript{X}, F\textsubscript{Y}] *\textsubscript{F} W\textsubscript{in}[D, F\textsubscript{Y}] \textit{(needed for previous layers)}
    \item dIn[B\textsubscript{X}, D\textsubscript{Y}] = \textbf{ReduceScatter}\textsubscript{Y}(dIn[B\textsubscript{X}, D] \{U.Y\}) \textit{(on critical path)}
\end{enumerate}

\end{algorithmbox}
}

\textbf{What's the right combination of FSDP and TP?} A simple but key maxim is that FSDP moves weights and tensor parallelism moves activations. That means as our batch size shrinks (especially as we do more data parallelism), tensor parallelism becomes cheaper because our activations per-shard are smaller.

\begin{itemize}
    \item Tensor parallelism performs $\mathbf{AllGather}_Y([B_X, D_Y])$ which shrinks as $X$ grows.
    \item FSDP performs $\mathbf{AllGather}_X([D_X, F_Y])$ which shrinks as $Y$ grows.
\end{itemize}

Thus by combining both we can push our minimum batch size per replica down even more. We can calculate the optimal amount of FSDP and TP in the same way as above:

Let $X$ be the number of chips dedicated to FSDP and $Y$ be the number of chips dedicated to tensor parallelism. Let $N$ be the total number of chips in our slice with $N=XY$. Let $M_X$ and $M_Y$ be the number of mesh axes over which we do FSDP and TP respectively (these should roughly sum to~3). We'll purely model the forward pass since it has the most communication per FLOP. Then adding up the comms in the algorithm above, we~have

\begin{align*}
T_\text{FSDP comms}(B, X, Y) &= \frac{2\cdot 2\cdot D \cdot F}{Y \cdot W_\text{ici} \cdot M_X} \\
T_\text{TP comms}(B, X, Y) &= \frac{2 \cdot 2 \cdot B \cdot D}{X \cdot W_\text{ici} \cdot M_Y}
\end{align*}

And likewise our total FLOPs time~is

$$T_\text{math} = \frac{2\cdot 2 \cdot B \cdot D \cdot F}{N \cdot C}.$$

To simplify the analysis, we make two assumptions: first, we allow $X$ and $Y$ to take on non-integer values (as long as they are positive and satisfy $XY=N$); second, we assume that we can fully overlap comms on the $X$ and $Y$ axis with each other. Under the second assumption, the total comms time~is

$$T_\text{comms} = \max\left(T_\text{FSDP comms}, T_\text{TP comms}\right)$$

Before we ask under what conditions we'll be compute-bound, let's find the optimal values for $X$ and $Y$ to minimize our total communication. Since our FLOPs is independent of $X$ and $Y$, the optimal settings are those that simply minimize comms. To do this, let's write $T_\text{comms}$ above in terms of $X$ and $N$ (which is held fixed, as it's the number of chips in our system) rather than $X$ and $Y$:

\begin{align*}
T_\text{comms} (X) &= \frac{4D}{W_\text{ici}} \max\left(\frac{F \cdot X}{N \cdot M_X}, \frac{B}{X \cdot M_Y}\right)
\end{align*}

Because $T_\text{FSDP comms}$ is monotonically increasing in $X$, and $T_\text{TP comms}$ is monotonically decreasing in $X$, the maximum must be minimized when $T_\text{FSDP comms} = T_\text{TP comms}$, which occurs~when

\begin{align*}
\frac{FX_{opt}}{M_X} = \frac{BN}{X_{opt} M_Y} \rightarrow \\
X_{opt} = \sqrt{\frac{B}{F} \frac{M_X}{M_Y} N}
\end{align*}

This is super useful! This tells us, for a given $B$, $F$, and $N$, what amount of FSDP is optimal. Let's get a sense of scale. Plugging in realistic values, namely $N = 64$ (corresponding to a~4x4x4 array of chips), $B=48,000$, $F=32768$, gives roughly $X\approx 13.9$. So we would choose $X$ to be~16 and $Y$ to be~4, close to our calculated optimum.

\begin{takeawaybox}
In general, during training, the optimal amount of FSDP is $X_{opt} = \sqrt{\frac{B}{F} \frac{M_X}{M_Y} N}$.
\end{takeawaybox}

Now let's return to the question we've been asking of all our parallelism strategies: \textbf{under what conditions will we be compute-bound?} Since we can overlap FLOPs and comms, we are compute-bound~when

$$\max\left(T_\text{FSDP comms}, T_\text{TP comms}\right) < T_\text{math}$$

By letting $\alpha \equiv C / W_\text{ici}$, the ICI arithmetic intensity, we can~simplify:

$$\max\left(\frac{F}{Y \cdot M_X}, \frac{B}{X \cdot M_Y}\right) < \frac{B \cdot F}{N \cdot \alpha}$$

Since we calculated $X_{opt}$ to make the LHS maximum equal, we can just plug it into either side (noting that $Y_{opt} = N/X_{opt}$), i.e.

\begin{align*}
&\frac{F}{N \cdot W_\text{ici} \cdot M_X} \sqrt{\frac{B}{F} \frac{M_X}{M_Y} N} \\
&\quad < \frac{B \cdot F}{N \cdot C}
\end{align*}

Further simplifying, we find~that

$$ \sqrt{\frac{B\cdot F}{M_X \cdot M_Y \cdot N}} < \frac{B \cdot F}{N \cdot \alpha},$$

where the left-hand-side is proportional to the communication time and the right-hand-side is proportional to the computation time. Note that while the computation time scales linearly with the batch size (as it does regardless of parallelism), the communication time scales as the square root of the batch size. The ratio of the computation to communication time thus also scales as the square of the batch~size:

$$ \frac{T_\text{math}}{T_\text{comms}} = \frac{\sqrt{BF}\sqrt{M_X M_Y}}{\alpha \sqrt{N}}. $$

To ensure that this ratio is greater than one so we are compute bound, we~require

$$ \frac{B}{N} > \frac{\alpha^2}{M_X M_Y F}$$

To get approximate numbers, again plug in $F=32,768$, $\alpha=2550$, and $M_X M_Y=2$ (as it must be for a~3D~mesh). This gives roughly $B/N > 99$. This roughly wins us a factor of eight compared to the purely data parallel (or FSDP) case, where assuming a~3D~mesh we calculate that $B/N$ must exceed about~850 to be compute bound.

\begin{takeawaybox}
Combining tensor parallelism with FSDP allows us to drop to a per-device batch size of $2550^2 / 2F$. This lets us handle a batch of as little as~100 per device, which is roughly a factor of eight smaller than we could achieve with just FSDP.
\end{takeawaybox}

Below we plot the ratio of FLOPs to comms time for mixed FSDP~+~TP, comparing it both to only tensor parallelism (TP) and only data parallelism (FSDP), on a representative 4x4x4 chip array. While pure FSDP parallelism dominates for very large batch sizes, in the regime where batch size over number of chips is between roughly~100 and~850, a mixed FSDP~+~TP strategy is required in order to be compute-bound.

\begin{figure}[htb]
    \centering
    \includegraphics[width=\textwidth]{images/mixed-fsdp-comms-2.png}
    \caption{Ratio of FLOPs to comms time for optimal mixed FSDP/TP on a TPUv5p 4x4x4 slice with F=30k. As expected, tensor parallelism has a fixed ratio with batch size; ideal mixed FSDP~+~TP scales with $\sqrt{B}$, and FSDP scales with~$B$. However, in intermediate batch size regimes, only FSDP~+~TP achieves a ratio greater than unity.}
    \label{fig:mixed-fsdp-comms-2}
\end{figure}

Here's another example of TPU v5p 16x16x16 showing the FLOPs and comms time as a function of batch size for different sharding schemes.

\begin{figure}[htb]
    \centering
    \includegraphics[width=\textwidth]{images/math-comms-time.png}
    \caption{Time taken for communication with different parallelism schemes. The black dashed line is the time taken by the matrix multiplication FLOPs, so any curve above this line is comms-bound. We note that all strategies become bandwidth-bound below batch size 6e5, which is in line with our expected $4096 \times 2550^2 / (2 \times 8192 \times 4) = 4e5$.}
    \label{fig:math-comms-time}
\end{figure}

The black curve is the amount of time spent on model FLOPs, meaning any batch size where this is lower than all comms costs is strictly comms bound. You'll notice the black curve intersects the green curve at about~\texttt{4e5}, as predicted.

You'll notice this generally agrees with the above (minimum around FSDP=256, TP=16), plus or minus some wiggle factor for some slight differences in the number of axes for each.

\subsection{Pipelining}

You'll probably notice we've avoided talking about pipelining at all in the previous sections. Pipelining is a dominant strategy for GPU parallelism that is somewhat less essential on TPUs. Briefly, pipelined training involves splitting the layers of a model across multiple devices and passing the activations between pipeline stages during the forward and backward pass. The algorithm is something like:

\begin{enumerate}
    \item Initialize your data on TPU~0 with your weights sharded across the layer dimension ($W_\text{in}[L_Z, D_X, F_Y]$ for pipelining with FSDP and tensor parallelism).
    \item Perform the first layer on TPU~0, then copy the resulting activations to TPU~1, and repeat until you get to the last TPU.
    \item Compute the loss function and its derivative $\partial L / \partial x_L$.
    \item For the last pipeline stage, compute the derivatives $\partial L / \partial W_L$ and $\partial L / \partial x_{L-1}$, then copy $\partial L / \partial x_{L-1}$ to the previous pipeline stage and repeat until you reach TPU~0.
\end{enumerate}

Here is some (working) Python pseudo-code. This pseudocode should run on a Cloud TPU VM. While it's not very efficient or realistic, it gives you a sense how data is being propagated across devices.

\begin{Shaded}
\begin{Highlighting}[]
\NormalTok{batch\_size }\OperatorTok{=} \DecValTok{32}
\NormalTok{d\_model }\OperatorTok{=} \DecValTok{128}
\NormalTok{d\_ff }\OperatorTok{=} \DecValTok{4} \OperatorTok{*}\NormalTok{ d\_model}

\NormalTok{num\_layers }\OperatorTok{=} \BuiltInTok{len}\NormalTok{(jax.devices())}

\NormalTok{key }\OperatorTok{=}\NormalTok{ jax.random.PRNGKey(}\DecValTok{0}\NormalTok{)}

\CommentTok{\# Pretend each layer is just a single matmul.}
\NormalTok{x }\OperatorTok{=}\NormalTok{ jax.random.normal(key, (batch\_size, d\_model))}
\NormalTok{weights }\OperatorTok{=}\NormalTok{ jax.random.normal(key, (num\_layers, d\_model, d\_model))}

\KeywordTok{def} \FunctionTok{layer\_fn}\NormalTok{(x, weight):}
  \ControlFlowTok{return}\NormalTok{ x }\OperatorTok{@}\NormalTok{ weight}

\CommentTok{\# Assume we have num\_layers == num\_pipeline\_stages}
\NormalTok{intermediates }\OperatorTok{=}\NormalTok{ [x]}
\ControlFlowTok{for}\NormalTok{ i }\KeywordTok{in} \BuiltInTok{range}\NormalTok{(num\_layers):}
\NormalTok{  x }\OperatorTok{=} \FunctionTok{layer\_fn}\NormalTok{(x, weights[i])}
\NormalTok{  intermediates.append(x)}

  \ControlFlowTok{if}\NormalTok{ i }\OperatorTok{!=}\NormalTok{ num\_layers }\OperatorTok{{-}} \DecValTok{1}\NormalTok{:}
\NormalTok{    x }\OperatorTok{=}\NormalTok{ jax.device\_put(x, jax.devices()[i}\OperatorTok{+}\DecValTok{1}\NormalTok{])}

\KeywordTok{def} \FunctionTok{loss\_fn}\NormalTok{(batch):}
  \ControlFlowTok{return}\NormalTok{ jnp.mean(batch }\OperatorTok{**} \DecValTok{2}\NormalTok{)  }\CommentTok{\# make up some fake loss function}

\NormalTok{loss, dx }\OperatorTok{=}\NormalTok{ jax.value\_and\_grad(loss\_fn)(x)}

\ControlFlowTok{for}\NormalTok{ i }\KeywordTok{in} \BuiltInTok{range}\NormalTok{(}\DecValTok{0}\NormalTok{, num\_layers, }\OperatorTok{{-}}\DecValTok{1}\NormalTok{):}
\NormalTok{  \_, f\_vjp }\OperatorTok{=}\NormalTok{ jax.vjp(layer\_fn, intermediates[i }\OperatorTok{+} \DecValTok{1}\NormalTok{], weights[i])}
\NormalTok{  dx, dw }\OperatorTok{=}\NormalTok{ f\_vjp(dx)  }\CommentTok{\# compute the jvp dx @ J(L)(x[i], W[i])}
\NormalTok{  weights[i] }\OperatorTok{=}\NormalTok{ weights[i] }\OperatorTok{{-}} \FloatTok{0.01} \OperatorTok{*}\NormalTok{ dw  }\CommentTok{\# update our weights}

  \ControlFlowTok{if}\NormalTok{ i }\OperatorTok{!=} \DecValTok{0}\NormalTok{:}
\NormalTok{    dx }\OperatorTok{=}\NormalTok{ jax.device\_put(dx, jax.devices()[i}\OperatorTok{{-}}\DecValTok{1}\NormalTok{])}
\end{Highlighting}
\end{Shaded}

\textbf{Why is this a good idea?} Pipelining is great for many reasons: it has a low communication cost between pipeline stages, meaning you can train very large models even with low bandwidth interconnects. This is often very useful on GPUs since they are not densely connected by ICI in the way TPUs are.

\textbf{Why is this difficult/annoying?} You might have noticed in the pseudocode above that TPU~0 is almost always idle! It's only doing work on the very first and last step of the pipeline. The period of idleness is called a pipeline bubble and is very annoying to deal with. Typically we try to mitigate this first with microbatching, which sends multiple small batches through the pipeline, keeping TPU~0 utilized for at least a larger fraction of the total step time.

A second approach is to carefully overlap the forward matmul $W_i @ x_i$, the backward $dx$ matmul $W_i @ \partial L / \partial x_{i+1}$, and the $dW$ matmul $\partial L / \partial x_{i+1} @ x_i$. Since each of these requires some FLOPs, we can overlap them to fully hide the bubble. Here's a plot from the recent DeepSeek v3~paper~\cite{DeepSeek3} showing their ``bubble-free'' pipeline schedule:

\begin{figure}[htb]
    \centering
    \includegraphics[width=\textwidth]{images/deepseek-pipeline.png}
    \caption{The DeepSeek v3 pipeline schedule (from their recent paper). Orange is the forward matmul, green is the dL/dx matmul, and blue is the dL/dW matmul. By prioritizing the backwards dL/dx multiplications, we can avoid ``stranding'' FLOPs.}
    \label{fig:deepseek-pipeline}
\end{figure}

Because it is less critical for TPUs (which have larger interconnected pods), we won't delve into this as deeply, but it's a good exercise to understand the key pipelining bottlenecks.

\subsection{Scaling Across Pods}

The largest possible TPU slice is a TPU v5p SuperPod with 8960 chips (and 2240 hosts). When we want to scale beyond this size, we need to cross the Data-Center Networking (DCN) boundary. Each TPU host comes equipped with one or several NICs (Network Interface Cards) that connect the host to other TPU v5p pods over Ethernet. As noted in the TPU Section, each host has about 200Gbps (25GB/s) of full-duplex DCN bandwidth, which is about 6.25GB/s full-duplex (egress) bandwidth per TPU.

Typically, when scaling beyond a single pod, we do some form of model parallelism or FSDP within the ICI domain, and then pure data parallelism across multiple pods. Let $N$ be the number of TPUs we want to scale to and $M$ be the number of TPUs per ICI-connected slice. To do an AllReduce over DCN, we can do a ring-reduction over the set of pods, giving us (in the backward pass):

$$T_\text{math} = \frac{2 \cdot 2 \cdot 2 \cdot BDF}{N \cdot C}$$

$$T_\text{comms} = \frac{2 \cdot 2 \cdot 2 \cdot DF}{M \cdot W_\text{dcn}}$$

The comms bandwidth scales with $M$, since unlike ICI the total bandwidth grows as we grow our ICI domain and acquire more NICs. Simplifying, we find that $T_\text{math} > T_\text{comms}$ when

$$\frac{B}{\text{slice}} > \frac{C}{W_\text{dcn}}$$

For TPU v5p, the $\frac{C}{W_\text{dcn}}$ is about \texttt{4.46e14 / 6.25e9 = 71,360}. This tells us that to efficiently scale over DCN, there is a minimum batch size per ICI domain needed to egress each node.

\textbf{How much of a problem is this?} To take a specific example, say we want to train LLaMA-3 70B on TPU v5p with a BS of 2M tokens. LLaMA-3 70B has $F\approx 30,000$. From the above sections, we know the following:

\begin{itemize}
    \item We can do Tensor Parallelism up to above $Y = M_Y \cdot F / 2550 \approxeq 11 \cdot M_Y$.
    \item We can do FSDP so long as $B / N > 2550 / M_X$. That means if we want to train with BS=2M and 3~axes of data parallelism, we'd at most be able to use $\approx 2400$ chips, roughly a quarter of a TPU v5p pod.
    \item When we combine FSDP~+~Tensor Parallelism, become bandwidth-bound when we have $B / N < 2550^2 / 2 * 30,000 = 108$, so this lets us scale to roughly 18k chips! However, the maximum size of a TPU v5p pod is 8k chips, so beyond that we have to use DCN.
\end{itemize}

The TLDR is that we have a nice recipe for training with BS=1M, using roughly X~(FSDP)~=~1024 and Y~(TP)~=~8, but with BS=2M we need to use DCN. As noted above, we have a DCN arithmetic intensity of 71,360, so we just need to make sure our batch size per ICI domain is greater than this. This is trivial for us, since with 2~pods we'd have a per-pod BS of 1M, and a per GPU batch size of 111, which is great (maybe cutting it a bit close, but theoretially sound).

\begin{takeawaybox}
Scaling across multiple TPU pods is fairly straightforward using pure data parallelism so long as our per-pod token batch size is at least 71k tokens.
\end{takeawaybox}

\section{Takeaways from LLM Training on TPUs}

\begin{itemize}
\item Increasing parallelism or reducing batch size both tend to make us more communication-bound because they reduce the amount of compute performed per chip.

\item Up to a reasonable context length~($\sim$32k) we can get away with modeling a Transformer as a stack of MLP blocks and define each of several parallelism schemes by how they shard the two/three main matmuls per layer.

\item During training there are 4~main parallelism schemes we consider, each of which has its own bandwidth and compute requirements~(data parallelism, FSDP, tensor parallelism).
\end{itemize}

{\scriptsize
\setlength{\tabcolsep}{3pt}
\begin{longtable}{p{2.5cm}p{5.5cm}}
\toprule
\textbf{Strategy} & \textbf{Description} \\
\midrule
\textbf{Data Parallelism} & Activations are batch sharded, everything else is fully-replicated, we all-reduce gradients during the backward pass. \\
\textbf{FSDP} & Activations, weights, and optimizer are batch sharded, weights are gathered just before use, gradients are reduce-scattered. \\
\textbf{Tensor Parallelism (aka Megatron, Model)} & Activations are sharded along~$d_{\text{model}}$, weights are sharded along~$d_{ff}$, activations are gathered before~$W_{\text{in}}$, the result reduce-scattered after~$W_{\text{out}}$. \\
\textbf{Mixed FSDP + Tensor Parallelism} & Both of the above, where FSDP gathers the model sharded weights. \\
\bottomrule
\end{longtable}
}

And here are the ``formulas'' for each method:

{\scriptsize
\setlength{\tabcolsep}{2pt}
\begin{longtable}{p{2cm}p{7.5cm}}
\toprule
\textbf{Strategy} & \textbf{Formula} \\
\midrule
DP & $\text{In}[B_X, D] \cdot_D W_{\text{in}}[D, F] \cdot_F W_{\text{out}}[F, D] \rightarrow \text{Out}[B_X, D]$ \\
FSDP & $\text{In}[B_X, D] \cdot_D W_{\text{in}}[D_X, F] \cdot_F W_{\text{out}}[F, D_X] \rightarrow \text{Out}[B_X, D]$ \\
TP & $\text{In}[B, D_Y] \cdot_D W_{\text{in}}[D, F_Y] \cdot_F W_{\text{out}}[F_Y, D] \rightarrow \text{Out}[B, D_Y]$ \\
TP + FSDP & $\text{In}[B_X, D_Y] \cdot_D W_{\text{in}}[D_X, F_Y] \cdot_F W_{\text{out}}[F_Y, D_X] \rightarrow \text{Out}[B_X, D_Y]$ \\
\bottomrule
\end{longtable}
}

\begin{itemize}
\item Each of these strategies has a limit at which it becomes network/communication bound, based on their per-device compute and comms. Here's compute and comms per-layer, assuming~$X$ is FSDP and~$Y$ is tensor parallelism.
\end{itemize}

{\scriptsize
\setlength{\tabcolsep}{3pt}
\begin{longtable}{p{3cm}p{2.5cm}p{2.5cm}}
\toprule
\textbf{Strategy} & \textbf{Compute per layer} & \textbf{Comms per layer} \\
& \textbf{(ignoring gating einsum)} & \textbf{(bytes, forward + backward pass)} \\
\midrule
DP & $4BDF/X + 8BDF/X$ & $0 + 8DF$ \\
FSDP & $4BDF/X + 8BDF/X$ & $4DF + 8DF$ \\
TP & $4BDF/Y + 8BDF/Y$ & $4BD + 4BD$ \\
FSDP + TP & $4BDF/(XY) + 8BDF/(XY)$ & $(4BD/X + 4DF/Y) + (8BD/X + 8DF/Y)$ \\
\bottomrule
\end{longtable}
}

\begin{itemize}
\item Pure data parallelism is rarely useful because the model and its optimizer state use bytes~= 10x parameter count. This means we can rarely fit more than a few billion parameters in memory.

\item Data parallelism and FSDP become comms bound when the~$\text{batch size per shard} < C / W$, the arithmetic intensity of the network. For ICI this is~2,550 and for DCN this is~75,000. This can be increased with more parallel axes.

\item Tensor parallelism becomes comms bound when~$|Y| > F / 2550$. \textbf{This is around 8--16~way for most models.} This is independent of the batch size.

\item Mixed FSDP~+ tensor parallelism allows us to drop the batch size to as low as~$2550^2 / 2F \approx 100$. This is remarkably low.

\item Data parallelism across pods requires a minimum batch size per pod of roughly~75,000 before becoming DCN-bound.

\item Basically, if your batch sizes are big or your model is small, things are simple. You can either do data parallelism or FSDP~+ data parallelism across DCN. The middle section is where things get interesting.
\end{itemize}

\section{Some Problems to Work}

Let's use LLaMA-2~13B as a basic model for this section. Here are the model details:

{\scriptsize
\setlength{\tabcolsep}{3pt}
\begin{longtable}{p{2cm}p{6cm}}
\toprule
\textbf{hyperparam} & \textbf{value} \\
\midrule
L & 40 \\
D & 5,120 \\
F & 13824 \\
N & 40 \\
K & 40 \\
H & 128 \\
V & 32,000 \\
\bottomrule
\end{longtable}
}

LLaMA-2 has separate embedding and output matrices and a gated MLP block.

\textbf{Question~1:} How many parameters does LLaMA-2~13B have~(I know that's silly but do the math)? \textit{Note that, as in Transformer Math, LLaMA-3 has 3~big FFW matrices, two up-projection and one down-projection. We ignored the two ``gating'' einsum matrices in this section, but they behave the same as~$W_{\text{in}}$ in this section.}

\textbf{Question~2:} Let's assume we're training with BS=16M tokens and using Adam. Ignoring parallelism for a moment, how much total memory is used by the model's parameters, optimizer state, and activations? \textit{Assume we store the parameters in bf16 and the optimizer state in fp32 and checkpoint activations three times per layer~(after the three big matmuls).}

\textbf{Question~3:} Assume we want to train with 32k~sequence length and a total batch size of 3M~tokens on a TPUv5p 16x16x16~slice. Assume we want to use bfloat16 weights and a float32 optimizer, as above.

\begin{enumerate}
\item Can we use pure data parallelism? Why or why not?
\item Can we use pure FSDP? Why or why not? With pure FSDP, how much memory will be used per device~(assume we do gradient checkpointing only after the 3~big FFW matrices).
\item Can we use mixed FSDP~+ tensor parallelism? Why or why not? If so, what should~$X$ and~$Y$ be? How much memory will be stored per device? Using only roofline FLOPs estimates and ignoring attention, how long will each training step take at~40\% MFU?
\end{enumerate}

\section{Appendix}

\subsection{Appendix A: Deriving the backward pass comms}

Above, we simplified the Transformer layer forward pass as $\text{Out}[B, D] = \text{In}[B, D] \times_D W_{\text{in}}[D, F] \times_F W_{\text{out}}[F, D]$. How do we derive the comms necessary for the backwards pass?

This follows fairly naturally from the rule in the previous section for a single matmul $\textbf{Y} = \textbf{X} \times \textbf{A}$:

\begin{align*}
\frac{dL}{dA} &= \frac{dL}{dY}\frac{dY}{dA} = X^T \left(\frac{dL}{dY}\right) \\
\frac{dL}{dX} &= \frac{dL}{dY}\frac{dY}{dX} = \left(\frac{dL}{dY}\right) A^T
\end{align*}

Using this, we get the following formulas~(letting $\text{Tmp}[B, F]$ stand for $\text{In}[B, D] \times W_{\text{in}}[D, F]$):

\begin{enumerate}
\item $\text{dW}_{\text{out}}[F, D] = \text{Tmp}[B, F] *_B \text{dOut}[B, D]$
\item $\text{dTmp}[B, F] = \text{dOut}[B, D] *_D W_{\text{out}}[F, D]$
\item $\text{dW}_{\text{in}} = \text{dTmp}[B, F] *_B \text{Tmp}[B, F]$
\item $\text{dIn}[B, D] = \text{dTmp}[B, F] *_F W_{\text{in}}[D, F]$
\end{enumerate}

Note that these formulas are mathematical statements, with no mention of sharding. The job of the backwards pass is to compute these four quantities. So to figure out the comms necessary, we just take the shardings of all the quantities which are to be matmulled in the four equations above~(Tmp, dOut, $W_{\text{out}}$, $W_{\text{in}}$), which are specified by our parallelization scheme, and use the rules of sharded matmuls to figure out what comms we have to do. Note that dOut is sharded in the same way as Out.

\chapter{Training LLaMA 3 on TPUs}
\label{chap:llama3}

% \textit{Our goal in this section is to apply results from the previous section to a very practical problem: training the LLaMA 3 family (herd) of models. Unlike the previous sections we want you to do a lot of this work yourself. For this reason, we've hidden the answers to each section so you can try to answer it first. Try grabbing a pen and doing by hand!}

\section{What does LLaMA 3 look like?}

The LLaMA-3 model family~\cite{llama3} includes 3 main models: LLaMA 3 8B, 70B, and 405B. We'll mostly focus on 70B, and leave 8B and 405B for you to explore in the problem section at the end. Here's the architecture for LLaMA 3-70B, taken from the LLaMA \href{https://huggingface.co/meta-llama/Meta-Llama-3-70B/blob/main/config.json}{HuggingFace page}.

{\scriptsize
\begin{longtable}{p{5cm} p{3cm}}
\toprule
\textbf{hyperparam} & \textbf{value} \\
\midrule
\endfirsthead
\toprule
\textbf{hyperparam} & \textbf{value} \\
\midrule
\endhead
\midrule
\multicolumn{2}{r}{\textit{Continued on next page}} \\
\endfoot
\bottomrule
\endlastfoot
$n_\text{layers}$ (L) & 80 \\
$d_\text{model}$ (D) & 8,192 \\
$d_{ff}$ (F) & 28,672 \\
$n_\text{heads}$ (N) & 64 \\
$n_\text{kv\_heads}$ (K) & 8 \\
$d_\text{qkv}$ (H) & 128 \\
$n_\text{embeddings}$ (V) & 128,256 \\
\end{longtable}
}

To highlight how easy this is to find, here's the config itself, along with a mapping:

\begin{figure}[htb]
\centering
\includegraphics[width=\textwidth]{images/llama-json.png}
\caption{LLaMA 3 configuration file showing the architecture parameters and their mapping to the hyperparameters listed above~.}
\label{fig:llama-json}
\end{figure}

\textit{It's useful to make a big table with these numbers for many different open-source LLMs, so you can quickly compare the design decisions they've made.}

% \section{Counting parameters and FLOPs}

\textbf{Question:} From this table, can we calculate the LLaMA 3-70B parameter count? Let's apply the content of Section~\ref{chap:transformers} and see if we can get 70B!

{\scriptsize
\setlength{\tabcolsep}{1.5pt}
\begin{longtable}{p{1.7cm}p{5cm}p{2.3cm}}
\toprule
\textbf{param} & \textbf{formula} & \textbf{count} \\
\midrule
FFW params & d\_model * d\_ff * 3 (gelu + out-proj) * n\_layers & 8,192 * 8,192 * 3.5 * 3 * 80 = \textbf{56.3e9} \\
\addlinespace
Vocab params & 2 (input \& output emb.) * n\_embeddings * d\_model & 2 * 128,256 * 8,192 = \textbf{2.1e9} \\
\addlinespace
Attention params & n\_layers * [ 2 (q emb. \& concat. out proj.) * d\_model * n\_heads * d\_qkv + 2 (k \& v) * d\_model * n\_kv\_heads * d\_qkv] & 80 * (2 * 8,192 * 64 * 128 + 2 * 8,192 * 8 * 128) = \textbf{12e9} \\
\addlinespace
 & & 56.3e9 + 2.1e9 + 12e9 = \textbf{70.4e9} \\
\bottomrule
\end{longtable}
}

That's great! We get the number we expect. You'll notice as expected that the FFW parameters totally dominate the overall parameter count, although attention is non-trivial.

\begin{takeawaybox}
The 3~big weight matrices in the MLP block are so much larger than all the other arrays in the Transformer that we can typically almost ignore all other parameters when reasoning about model memory or FLOPs. For LLaMA 3-70B, they represent 56B of 70B parameters.
\end{takeawaybox}

% Let's look at FLOPs now! \textit{Remember the general rules for training from Section~\ref{chap:transformers}.}

\textbf{Question:} How many FLOPs does LLaMA-3 perform per token per training step? \textit{This helps us determine how expensive the whole training process will be.}

\textbf{Question:} LLaMA 3 was trained for about 15 trillion tokens. How many FLOPs is that total?

\textbf{Question:} Let's say we wanted to train on a full TPU v5p pod with 16x20x28 = 8960 chips. How long would this take to train at 40\% MFU in bfloat16, assuming we are compute-bound?

% \textbf{Question:} LLaMA 3-70B was pretrained with a batch size of about 4M tokens. How many TPUs do we need at minimum to train with this batch size? \textit{You can assume bfloat16 parameters and float32 optimizer state, and that you checkpoint gradients 4 times per layer.}

\textbf{Question:} Under the same assumptions as the question above, if we use 8960 TPU v5p chips, how much memory will we use per-chip?

\begin{takeawaybox}
It is technically possible to train even very large models on very small topologies, with the caveat that they will likely take a long time. Being able to calculate the total FLOPs of a training run allows us to ballpark its training time by assuming a modest MFU and a known topology.
\end{takeawaybox}

% \section{How to shard LLaMA 3-70B for training}

Let's stick to our setting from above and say we want to train LLaMA 3-70B with 4M token batch size (1024 sequences of length 4096 per batch) on a TPU v5p pod of 8960 chips. Let's discuss what the best sharding strategy is for this model.

\textbf{Question:} Under the assumptions above, can we train our model with FSDP alone? To start, let's say we can't do any sequence/context parallelism. \textit{This should be the first idea you have, since it's simple and will introduce no extra communication if it works.}
\textbf{Question:} Let's relax the requirement of not doing any sequence sharding. If we allow ourselves to do FSDP over both the batch \textit{and} sequence axes, can we train LLaMA 3-70B with only FSDP on 8960 chips?
\textbf{Question:} Now let's look at mixed tensor parallelism and FSDP. Does there exist some combination that lets us remain compute-bound? What amount of FSDP and tensor parallelism should we do if so?
\begin{takeawaybox}
We can train LLaMA-3 with a 4M token batch size on a full TPU v5p pod with a mixture of data parallelism (1024-way), sequence parallelism (2-way), and tensor parallelism (4-way) without being communication-bound. We will be comms-bound if we try to do pure FSDP or FSDP + sequence parallelism. The equations we've cooked up in the previous section are very practical.
\end{takeawaybox}

% \section{A Few Problems to Work}

\textbf{Question 1:} How many parameters does a model with $D=4096$, $F=4 \cdot D$, $V=32,000$, and $L=64$ have? What fraction of these are attention parameters? How large are our KV caches per token? \textit{You can assume $N\cdot H=D$ and multi-head attention with int8 KVs.}

\textbf{Question 2:} How many total FLOPs are required to perform A[B$_X$, D$_Y$] *$_D$ W[D$_Y$, F] on \texttt{\{'X': 4, 'Y': 8, 'Z': 4\}}. How many FLOPs are performed by each TPU?

\textbf{Question 3:} How many FLOPs are involved in performing $A[I,J,K,L] * B[I,J,M,N,O] \rightarrow C[K,L,M,N,O]$?

\textbf{Question 4:} What is the arithmetic intensity of self-attention (ignoring the Q/K/V/O projections)? \textit{Give the answer as a function of the Q and KV lengths T and S.} At what context length is attention FLOPs-bound? Given the HBM bandwidth of our TPUs, plot the effective relative cost of attention to the FFW block as the context length grows.

\textbf{Question 5:} At what sequence length are self-attention FLOPs equal to the QKVO projection FLOPs?

\textbf{Question 6:} Say we only save the output of each of the 7 main matmuls in a Transformer layer during our forward pass (Q, K, V, O + the three FFW matrices). How many extra FLOPs do we need to ``rematerialize'' during the backwards pass?

\textbf{Question 7:} DeepSeek v3 says it was trained for 2.79M H800 hours on 14.8T tokens (source). Given that it has 37B activated parameters, roughly what hardware utilization did they achieve? \textit{Hint: note that they used FP8 FLOPs without structured sparsity.}

\textbf{Question 8:} Mixture of Experts (MoE) models have $E$ copies of a standard dense MLP block, and each token activates $k$ of these experts. What batch size in tokens is required to be compute-bound for an MoE with weights in int8 on TPU v5e? For DeepSeek, which has 256 (routed) experts and $k=8$, what is this number?


\chapter{All About Transformer Inference}
\label{chap:inference}

\section{The Basics of Transformer Inference}

So you've trained a Transformer, and you want to use it to generate some new sequences. \textit{At the end of the day, benchmark scores going up and loss curves going down are only proxies for whether something interesting is going to happen once the rubber hits the road!}\footnote{Historically, you can do a surprising amount of research on Transformers without ever touching inference---LLM loss, multiple choice benchmarks can be run efficiently without a proper KV cache or generation loop implementation. This meant, especially in research codebases, there's often a lot of low hanging fruits in the inference codepath.}

Sampling is conceptually simple. We put a sequence in and our favorite Transformer will spit out $\log p(\text{next token}_i \mid \text{previous tokens})$~, i.e. log-probabilities for all possible next tokens. We can sample from this distribution and obtain a new token. Append this token and repeat this process and we obtain a sequence of tokens which is a continuation of the prompt.

\begin{figure}[htb]
\centering
\includegraphics[width=\textwidth]{images/naive-inference.png}
\caption{naive sampling from a Transformer. The blue logits give us a distribution over the next token that we can sample from. Note that each step re-processes the entire prefix, leading to a $\Theta(n^2)$ runtime for the algorithm.}
\end{figure}

We have just described the naive implementation of Transformer sampling, and while it works, \textbf{we never do it in practice} because we are re-processing the entire sequence every time we generate a token. This algorithm is $O(n^2)$ on the FFW and $O(n^3)$ on the attention mechanism to generate $n$ tokens!

\textbf{How do we avoid this?} Instead of doing the full forward pass every time, it turns out we can save some intermediate activations from each forward pass that let us avoid re-processing previous tokens. Specifically, since a given token only attends to previous tokens during dot-product attention, we can simply write each token's key and value projections into a new data structure called a \textbf{KV cache}. Once we've saved these key/value projections for past tokens, future tokens can simply compute their $q_i \cdot k_j$ products without performing any new FLOPs on the earlier tokens. Amazing!

With this in mind, inference has two key parts:

\begin{itemize}
\item \textcolor{red}{\textbf{Prefill}}: Given a long prompt, we process all the tokens in the prompt at the same time and save the resulting activations (specifically, the key-value projections) in a \textbf{``KV cache''}. We also save the logits for the last token.
\item \textcolor{blue}{\textbf{Generation}}: Given a KV cache and the previous logits, we incrementally sample one token from the logits, feed that token back into the Transformer, and produce a new set of logits for the next step. We also append the KV activations for that new token to the KV cache. We repeat this until we hit a special \texttt{<EOS>} token or reach some maximum length limit.
\end{itemize}

Here's a diagram of sampling with a KV cache:

\begin{figure}[htb]
\centering
\includegraphics[width=\textwidth]{images/cached-inference.png}
\caption{diagram of efficient Transformer sampling with a KV cache.}
\end{figure}

By sampling with a KV cache, we've reduced our time complexity to generate $n$ tokens to $O(n)$ on the FFW and $O(n^2)$ on the attention, since we never reprocess a previous token. However, many forward passes are still needed to generate a sequence---that's what's happening when you query Gemini or ChatGPT and the result streams back to you. Every token is (usually) a separate (but partially cached) Transformer call to a massive model.

We will soon see that \textcolor{red}{\textbf{prefill}} and \textcolor{blue}{\textbf{generation}} are very different beasts---Transformer inference is two tasks in disguise! Compared to training, the KV cache is also a novel and significant source of complexity.

\subsection{What do we actually want to optimize?}

Before we proceed further, it's worth highlighting one aspect of inference that's totally new: latency. While during training we only care about throughput (total tokens processed per second \textbf{per chip}), during inference we have to worry about how fast we're producing tokens (both the \textbf{Time To First Token (TTFT)} and the \textbf{per-token latency}). For example:

\begin{itemize}
\item \textbf{Offline batch inference} for evals and data generation only cares about bulk cost of inference and is blind to the latency of individual samples.
\item \textbf{Chat interfaces/streaming tasks} need to run cheaply at scale while having low TTFT and generating tokens fast enough to exceed human reading speed.
\item \textbf{Edge inference} (e.g. \texttt{llama.cpp} on your laptop) only needs to service one user at a time at the lowest possible latency, potentially with heavy hardware constraints.
\end{itemize}

Maximizing hardware utilization is still critical and helps with cost and TTFT, but unlike training, it does not \textit{necessarily} translate to better experience for individual users in all contexts. Many optimizations at the accelerator, systems and model architectural level make tradeoffs between latency, throughput, context length and even model quality.

\subsection{A more granular view of the Transformer}

So far we've mostly treated a Transformer as a stack of feedforward blocks. While this is often reasonable from a FLOPs and memory standpoint, it's not sufficient to properly model inference.\footnote{One thing you'll notice throughout this section is that inference is much less forgiving than training. We typically have far fewer FLOPs, less opportunity for batching, and a much greater sensitivity to latency. KV caches dramatically complicate inference as well.} As we saw in [Part 4](../transformers), the major components of a Transformer forward pass are:

\begin{enumerate}
\item \textbf{A bunch of linear operations}, including the MLP ($W_{in}$, $W_{out}$) and the attention QKV projections and output projections ($W_Q$, $W_K$, $W_V$, and $W_O$). These all involve reading parameters and a batch of activations from HBM, doing some FLOPs, and writing the result back to HBM.
\item \textbf{Dot-product attention}. We need to read a batch of key-value projections and a batch of query activations from HBM, do a few inner products and some softmax operations, and write the attention result back to HBM.
\item \textbf{Everything else}, including applying layer norms, activation functions, tokens sampling, updating KV caches, and positional embeddings. These do take some FLOPs, but are dominated by, or fused into, the above.
\end{enumerate}

For the next couple of sections, we're going to look at each of these in the context of prefill and generation and ask what is likely to bottleneck our performance. Within a single accelerator, are we compute-bound or memory-bound? We want to emphasize how different the answers will be for prefill versus generation.

\subsection{Linear operations: what bottlenecks us?}

All our linear operations are conceptually the same, whether they live in the MLP block or attention. Their arithmetic intensity depends on the batch size. We did this math in [Section 1](../roofline) but it's worth repeating. Let's look at a single matrix multiply of a $\text{bf16[B, D]}$ batch by a $\text{bf16[D, F]}$ matrix. This could be the big MLP block ($W_\text{in}$ or $W_\text{out}$) or one of the smaller attention projections ($W_Q$, $W_K$, $W_V$, $W_O$). To do this matmul, we need to load both of these arrays from HBM into the MXU, do the multiplicaton, then write the result back to HBM. As before, we have:

\begin{align*}
T_\text{math} &= \frac{\text{Computation FLOPs}}{\text{Accelerator FLOPs/s}} = \frac{2BDF}{\text{Accelerator FLOPs/s}} \\
T_\text{comms} &= \frac{\text{Communication Bytes}}{\text{Bandwidth Bytes/s}} = \frac{2BD + 2FD + 2BF}{\text{Bandwidth Bytes/s}}
\end{align*}

A TPU or GPU can overlap these by loading as it does the compute, so to be compute-bound, we need $T_\text{math} \geq T_\text{comms}$~, or:

{\small
\begin{equation*}
\frac{2BDF}{2BD + 2DF + 2BF} \geq \frac{\text{Accelerator FLOPs/s}}{\text{Bandwidth Bytes/s}} \underset{\text{TPU v5e}}{=} \frac{1.97E+14}{8.20E+11} = 240
\end{equation*}
}

where the RHS is the arithmetic intensity of our hardware. Now let's assume $D$ and $F$ are very large compared to $B$ (usually our batches are at most 500 and $D$ and $F > 10k$), we can simplify the denominator by using the fact that $2BD + 2DF + 2BF \approxeq 2DF$ which gives us

\begin{align*}
\frac{2BDF}{2BD + 2DF + 2BF} &\approxeq \frac{2BDF}{2DF} \geq \frac{\text{Accelerator FLOPs/s}}{\text{Bandwidth Bytes/s}} \\
&\underset{\text{TPU v5e}}{=} \frac{1.97E+14}{8.20E+11} \implies B \geq 240 = B_{\text{crit}}
\end{align*}

If we quantize our weights or use lower precision FLOPs for the matrix multiplication, this critical batch size can change. For instance, if we quantize our weights to int8 or fp8, $B_\text{crit}$ decreases by 2x. If we do our FLOPs in int8 or fp8, $B_\text{crit}$ increases by 2x. Thus if we let $\beta = \text{bits per param} / \text{bits per activation}$ and $\alpha_\text{hbm} = C / W_\text{hbm}$~, our critical batch size is actually $B_\text{crit} = \beta \alpha_\text{hbm}$~.

\begin{takeawaybox}
Transformer matmuls are compute-bound \textit{iff} the per-replica \textbf{token} batch size is greater than $B_\text{crit} = C / W_\text{hbm} \cdot (\text{bits per param} / \text{bits per activation}) = \beta \cdot \alpha_\text{hbm}$ For bf16 activations on TPU v5e, this is 240 tokens. For an H100, it is about 280 tokens.
\end{takeawaybox}

During training, we'll have a high intensity during all our matrix multiplications because we reuse the same weights over a very large batch. \textbf{That high arithmetic intensity carries over to prefill, since user prompts are typically hundreds if not thousands of tokens long.} As we saw before, the hardware arithmetic intensity of a TPUv5e is 240, so if a sequence longer than 240 tokens is fed into a dense model running on this hardware at bf16, we would expect to be compute-bound and all is well. Prompts shorter than this can technically be batched together to achieve higher utilization, but this is typically not necessary.

\begin{takeawaybox}
During prefill, all matrix multiplications are basically always compute-bound. Therefore, simply maximizing hardware utilization or MFU (Model FLOPs Utilization) is enough to maximize throughput-per-chip (cost) and latency (in the form of TTFT). Unless prompts are extremely short, batching at a per-prompt level only adds latency for a small improvements in prefill throughput.
\end{takeawaybox}

However, during generation, for each request, we can only do our forward passes one token at a time since there's a sequential dependency between steps! Thus we can only (easily) achieve good utilization by batching multiple requests together, parallelizing over the batch dimension. We'll talk about this more later, but actually batching many concurrent requests together without affecting latency is hard. For that reason, \textbf{it is much harder to saturate the hardware FLOPs with generation.}

\begin{takeawaybox}
During generation, the total token batch size must be greater than $B_{\text{crit}}$ to be compute-bound on the linear/feed-forward operations (240 for bf16 params on TPU v5e). Because generation happens serially, token-by-token, this requires us to batch multiple requests together, which is hard!
\end{takeawaybox}

\textit{It's worth noting just how large this is!} Generate batch size of 240 means 240 concurrent requests generating at once, and 240 separate KV caches for dense models. That means this is difficult to achieve in practice, except in some bulk inference settings. In contrast, pushing more than 240 tokens through during a prefill is pretty routine, though some care is necessary as sparsity increases.

\textbf{Note that this exact number will differ on the kind of quantization and hardware.} Accelerators often can supply more arithmetic in lower precision. For example, if we have int8 parameters but do our computation in bf16, the critical batch size drops to 120. With int8 activations and int8 params, it jumps back up to 240 since the TPUv5e can supply 400 TOPs/s of int8 x int8.

\subsection{What about attention?}

Things get more complicated when we look at the dot-product attention operation, especially since we have to account for KV caches. Let's look at just one attention head with pure multi-headed attention. In a single Flash Attention fusion, we\footnote{We're simplifying a fair bit here by ignoring the non-matmul FLOPs in applying the softmax, masks etc. They should be overlapped with computation or HBM reads, but it can be non-trivial to do on certain TPU generations. Whese details don't change the main message, which is that KV caches are usually memory bound.}:

\begin{enumerate}
\item Read the $Q$ activations of shape $\text{bf16[B, T, D]}$ from HBM.
\item Read the $KV$ cache, which is a pair of $\text{bf16[B, S, D]}$ tensors from HBM.
\item Perform $2BSTD$ FLOPs in the $QK$ matmul. With Flash Attention, we don't need to write the $\text{bf16[B, S, T]}$ attention matrix back into HBM.
\item Perform $2BSTD$ in the attention $AV$ matmul.
\item Write the resulting $\text{bf16[B, T, D]}$ tensor back into HBM.
\end{enumerate}

Putting it all together, we get:

\begin{align*}
\text{Multiheaded Attention} & \\
\text{Arithmetic Intensity} &= \frac{4BSTD}{4BSD + 4BTD} = \frac{ST}{S+T}
\end{align*}

For prefill, $S=T$ since we're doing self-attention, so this simplifies to $T^2 / 2T = T / 2$ This is great because it means \textbf{the arithmetic intensity of attention during prefill is $\Theta(T)$}~. That means it's quite easy to be compute-bound for attention. As long as our sequence length is fairly large, we'll be fine!

But since generation has a trivial sequence dim, and the $B$ and $D$ dims cancel, we can make the approximation:

$$S \gg T = 1 \implies \frac{ST}{S+T} \approx 1$$

This is bad, since it means we cannot do anything to improve the arithmetic intensity of attention during generation. We're doing a tiny amount of FLOPs while loading a massive KV cache. \textbf{So we're basically always memory bandwidth-bound during attention!}

\begin{takeawaybox}
During prefill, attention is usually compute bound for any reasonable sequence length (roughly $> 480$ tokens) while during generation our arithmetic intensity is low and constant, so we are always memory bandwidth-bound.
\end{takeawaybox}

\textit{Why is this, conceptually?} Mainly, we're compute-bound in linear portions of the model because the parameters (the memory bandwidth-heavy components) are reused for many batch items. However, every batch item has its own KV cache, so a bigger batch size means more KV caches. We will almost \textit{always} be memory bound here unless the architecture is adjusted aggressively.

This also means you will get diminishing returns on throughput from increasing batch size once params memory becomes comparable to KV cache memory. The degree to which the diminishing returns hurt you depends on the ratio of parameter to KV cache bytes for a single sequence, i.e. roughly the ratio $2DF / SHK$ Since $HK\approx D$~, this roughly depends on the ratio of $F$ to $S$~, the sequence length. This also depends on architectural modifications that make the KV cache smaller (we'll say more in a moment).

\subsection{Theoretical estimates for LLM latency and throughput}

From this math, we can get pretty good bounds on the step time we should aim for when optimizing. \textbf{(Note: if there is one thing we want to the reader to take away from this entire chapter, it's the following).} For small batch sizes during generation (which is common), we can lower-bound our per-step latency by assuming we're memory bandwidth bound in both the attention and MLP blocks:

{\footnotesize
\begin{equation*}
\text{Theoretical Min Step Time} = \frac{\text{Batch Size} \times \text{KV Cache Size} + \text{Parameter Size}}{\text{Total Memory Bandwidth}}
\end{equation*}
}

Similarly, for throughput:

{\footnotesize
\begin{equation*}
\text{Theoretical Max Tokens/s} = \frac{\text{Batch Size} \times \text{Total Memory Bandwidth}}{\text{Batch Size} \times \text{KV Cache Size} + \text{Parameter Size}}
\end{equation*}
}

Eventually, as our batch size grows, FLOPs begin to dominate parameter loading, so in practice we have the more general equation:

{\footnotesize
\begin{align*}
&\text{Theoretical Step Time (General)} = \underbrace{\frac{\text{Batch Size} \times \text{KV Cache Size}}{\text{Total Memory Bandwidth}}}_{\text{Attention (always bandwidth-bound)}} \\
& + \underbrace{\max\left(\frac{2 \times \text{Batch Size} \times \text{Parameter Count}}{\text{Total FLOPs/s}}, \frac{\text{Parameter Size}}{\text{Total Memory Bandwidth}}\right)}_{\text{MLP (can be compute-bound)}}
\end{align*}
}

where the attention component (left) is never compute-bound, and thus doesn't need a FLOPs roofline. These are fairly useful for back-of-the-envelope calculations, e.g.

\textbf{\textcolor[rgb]{0.34,0.81,0.34}{Pop Quiz:}} Assume we want to take a generate step with a batch size of 4 tokens from a 30B parameter dense model on TPU v5e 4x4 slice in int8 with bf16 FLOPs, 8192 context and 100 kB / token KV caches. What is a reasonable lower bound on the latency of this operation? What if we wanted to sample a batch of 256 tokens?

As you can see, there's a clear tradeoff between throughput and latency here. Small batches are fast but don't utilize the hardware well. Big batches are slow but efficient. Here's the latency-throughput Pareto frontier calculated for some older PaLM models (from the \href{https://arxiv.org/pdf/2211.05102}{ESTI paper}~\cite{esti}):

\begin{figure}[htb]
\centering
\includegraphics[width=\textwidth]{images/latency-cost.png}
\caption{Pareto frontier of cost (read: throughput) versus latency for several PaLM models. Note how chip count (C) and batch size (B) moves you along the Pareto frontier, with the exception of the green dot (C:32 B:16 for PaLM 540B) where the available memory prevented the setup from supporting a good batch size and caused throughput to suffer. Note how throughput generally tends to flatten around after the batch size 240. int8 weights offers a better latency-throughput pareto optimal, but not a better max throughput.}
\end{figure}

Not only do we trade off latency and throughput with batch size as knob, we may also prefer a larger topology to a smaller one so we can fit larger batches if we find ourselves limited by HBM. The [next section](../applied-inference) explores this in more detail.

\begin{takeawaybox}
If you care about generation throughput, use the largest per-chip batch size possible. Any per-chip batch size above the TPU arithmetic intensity ($B_\text{crit}$~, usually 120 or 240) will maximize throughput. You may need to increase your topology to achieve this. Smaller batch sizes will allow you to improve latency at the cost of throughput.
\end{takeawaybox}

This is all quite theoretical. In practice we often don't quite see a sharp roofline for a few reasons:

\begin{itemize}
\item Our assumption that HBM reads will be perfectly overlapped with FLOPs is not realistic, since our compiler (XLA) is fallible.
\item For sharded models, XLA also often fails to efficiently overlap the ICI communication of our model-sharded matrix multiples with the FLOPs themselves, so we often start taking a latency hit on linears over $\text{BS}=32$~.
\item Batch sizes larger than the theoretical roofline will still see some improvement in throughput because of imperfect overlapping, but this is a good heuristic.
\end{itemize}

\subsection{What about memory?}

We've spent some time looking at bandwidth and FLOPs, but not at memory. The memory picture looks a lot different at inference time, thanks to our new data structure, the KV cache. For this section, let's pick a real model (LLaMA 2-13B) to demonstrate how different things look:

\begin{table}[htb]
\centering
{\scriptsize
\setlength{\tabcolsep}{2pt}
\begin{longtable}{p{5cm}p{2cm}}
\toprule
\textbf{hyperparam} & \textbf{value} \\
\midrule
L (num\_layers) & 40 \\
\addlinespace
D (d\_model) & 5,120 \\
\addlinespace
F (ffw\_dimension) & 13,824 \\
\addlinespace
N (num\_heads) & 40 \\
\addlinespace
K (num\_kv\_heads) & 40 \\
\addlinespace
H (qkv\_dim) & 128 \\
\addlinespace
V (num\_embeddings) & 32,000 \\
\bottomrule
\end{longtable}
}
\end{table}

What's using memory during inference? Well, obviously, our parameters. Counting those, we have:

\begin{table}[hbt]
\centering
{\scriptsize
\setlength{\tabcolsep}{1.5pt}
\begin{longtable}{p{2cm}p{4.5cm}p{2.5cm}}
\toprule
\textbf{param} & \textbf{formula} & \textbf{size (bytes)} \\
\midrule
FFW params & d\_model\textsuperscript{2} x ffw\_mult. x 3 (gelu + out-proj) x n\_layers & 5,120 x 5,120 x 2.7 x 3 x 40 = \textbf{8.5e9} \\
\addlinespace
Vocab params & 2 (in \& out emb.) x n\_emb. x d\_model & 2 x 32,000 x 5,120 = \textbf{0.3e9} \\
\addlinespace
Attn params & [2 (q \& out) x d\_model x n\_heads x d\_qkv + 2 (k \& v) x d\_model x n\_kv\_heads x d\_qkv] x n\_layers & (2 x 5,120 x 40 x 128 + 2 x 5,120 x 40 x 128) x 40 = \textbf{4.2e9} \\
\bottomrule
\end{longtable}
}
\end{table}

Adding these parameters up, we get 8.5e9 + 4.2e9 + 0.3e9 = \textbf{13e9 total parameters}, just as expected. As we saw in the previous sections, during training we might store our parameters in bfloat16 with an optimizer state in float32. That may use around 100GB of memory. That pales in comparison to our gradient checkpoints, which can use several TBs.

\textbf{How is inference different?} During inference, we store one copy of our parameters, let's say in bfloat16. That uses 26GB---and in practice we can often do much better than this with quantization. There's no optimizer state or gradients to keep track of. Because we don't checkpoint (keep activations around for the backwards pass), our activation footprint is negligible for both prefill\footnote{Particularly thanks to Flash Attention, which avoids materializing our attention matrix} and generate. If we prefill 8k tokens, a single activation only uses around \texttt{8,192 x 5,120 x 2 bytes = 80MB} of memory. Longer prefills can be broken down into many smaller forward passes, so it's not a problem for longer contexts either. Generation use even fewer tokens than that, so activations are negligible.

\textbf{The main difference is the KV cache}. These are the keys and value projections for all past tokens, bounded in size only by the maximum allowed sequence length. The total size for $T$ tokens is

$$\text{KV cache size} = 2 \cdot \text{bytes per float} \cdot H \cdot K \cdot L \cdot T$$

where $H$ is the dimension of each head, $K$ is the number of KV heads, $L$ is the number of layers, and the 2 comes from storing both the keys and values.

\textbf{This can get big very quickly}, even with modest batch size and context lengths. For LLaMA-13B, a KV cache for a single 8192 sequence at bf16 is

$$8192\ (T) \times 40\ (K) \times 128\ (H) \times 40\ (L) \times 2\ (\text{bytes}) \times 2 = 6.7 \text{GB}$$

\textbf{Just 4 of these exceed the memory usage of our parameters!} To be clear, LLaMA 2 was not optimized for KV cache size at longer contexts (it isn't always this bad, since usually $K$ is much smaller, as in LLaMA-3), but this is still illustrative. We cannot neglect these in memory or latency estimates.

\subsection{Modeling throughput and latency for LLaMA 2-13B}

Let's see what happens if we try to perform generation perfectly efficiently at different batch sizes on 8xTPU v5es, up to the critical batch size (240) derived earlier for maximum theoretical throughput.

\begin{table}[htb]
\centering
{\scriptsize
\setlength{\tabcolsep}{1.5pt}
\begin{longtable}{p{3cm}p{0.9cm}p{0.9cm}p{0.9cm}p{0.9cm}p{0.9cm}p{1cm}}
\toprule
\textbf{Batch Size} & \textbf{1} & \textbf{8} & \textbf{16} & \textbf{32} & \textbf{64} & \textbf{240} \\
\midrule
KV Cache Memory (GiB) & 6.7 & 53.6 & 107.2 & 214.4 & 428.8 & 1608 \\
\addlinespace
Total Memory (GiB) & 32.7 & 79.6 & 133.2 & 240.4 & 454.8 & 1634 \\
\addlinespace
Step Time (ms) & 4.98 & 12.13 & 20.30 & 36.65 & 69.33 & 249.09 \\
\addlinespace
Throughput (tok/s) & 200.61 & 659.30 & 787.99 & 873.21 & 923.13 & 963.53 \\
\bottomrule
\end{longtable}
}
\end{table}

8x TPU v5es gives us 128GiB of HBM, 6.5TiB/s of HBM bandwidth (0.82TiB/s each) and 1600TF/s of compute.

For this model, increasing the batch size does give us better throughput, but we suffer rapidly diminishing returns. We OOM beyond batch size 16, and need an order of magnitude more memory to go near 240. A bigger topology can improve the latency, but we've hit a wall on the per chip throughput.

Let's say we keep the total number of params the same, but magically make the KV cache 5x smaller (say, with 1:5 [GMQA](\#tricks-for-improving-generation-throughput-and-latency), which means we have 8 KV heads shared over the 40 Q heads---see next section for more details).

\begin{table}[htb]
\centering
{\scriptsize
\setlength{\tabcolsep}{1.5pt}
\begin{longtable}{p{3cm}p{0.9cm}p{0.9cm}p{0.9cm}p{0.9cm}p{0.9cm}p{1cm}}
\toprule
\textbf{Batch Size} & \textbf{1} & \textbf{8} & \textbf{16} & \textbf{32} & \textbf{64} & \textbf{240} \\
\midrule
KV Cache Memory (GiB) & 1.34 & 10.72 & 21.44 & 42.88 & 85.76 & 321.6 \\
\addlinespace
Total Memory (GiB) & 27.34 & 36.72 & 47.44 & 68.88 & 111.76 & 347.6 \\
\addlinespace
Step Time (ms) & 4.17 & 5.60 & 7.23 & 10.50 & 17.04 & 52.99 \\
\addlinespace
Throughput (tok/s) & 239.94 & 1,429.19 & 2,212.48 & 3,047.62 & 3,756.62 & 4,529.34 \\
\bottomrule
\end{longtable}
}
\end{table}

With a smaller KV cache, we still have diminishing returns, but the theoretical throughput per chip continues to scale up to batch size 240. We can fit a much bigger batch of 64, and latency is also consistently better at all batch sizes. The latency, maximum throughput, and maximum batch size all improve dramatically! In fact, later LLaMA generations used this exact optimization---LLaMA-3 8B has 32 query heads and 8 KV heads (\href{https://huggingface.co/MaziyarPanahi/Llama-3-13B-Instruct-v0.1/blob/dfdeb40bdb2c149dfa399ea2be0d56eb120f0831/config.json}{source}).
\newpage
\begin{takeawaybox}
In addition to params, the size of KV cache has a lot of bearing over the ultimate inference performance of the model. We want to keep it under control with a combination of architectural decisions and runtime optimizations.
\end{takeawaybox}

\section{Tricks for Improving Generation Throughput and Latency}

Since the original \href{https://arxiv.org/abs/1706.03762}{Attention is All You Need paper}, many techniques have been developed to make the model more efficient, often targeting the KV cache specifically. Generally speaking, a smaller KV cache makes it easier to increase batch size and context length of the generation step without hurting latency, and makes life easier for the systems surrounding the Transformer (like request caching). Ignoring effects on quality, we may see:

\textbf{Grouped multi-query attention (aka GMQA, GQA):} We can reduce the number of KV heads, and share them with many Q heads in the attention mechanism. In the extreme case, it is possible to share a single KV head across all Q heads.  This reduces the KV cache by a factor of the Q:KV ratio over pure MHA, and it has been observed that the performance of models is relatively insensitive to this change.

\begin{figure}[htb]
\centering
\includegraphics[width=\textwidth]{images/gmqa.png}
\end{figure}

This also effectively increases the arithmetic intensity of the attention computation (see Question 4 in Section 4).

\textbf{Mixing in some local attention layers:} Local attention caps the context to a small to moderately sized max length. At training time and prefill time, this involves masking the attention matrix to a diagonal strip instead of a triangle. This effectively caps the size of the max length of the KV cache for the local layers. By mixing in some local layers into the model with some global layers, the KV cache is greatly reduced in size at contexts longer than the local window.

\textbf{Sharing KVs across layers:} The model can learn to share the same KV caches across layers in some pattern. Whilst this does reduce the KV cache size, and provide benefits in increasing batch size, caching, offline storage etc. shared KV caches may need to be read from HBM multiple times, \textit{so it does not necessarily improve the step time.}

\begin{figure}[htb]
\centering
\includegraphics[width=\textwidth]{images/kv-sharing.png}
\caption{\textbf{Left:} Multiple layers of pure global attention. \textbf{Right:} An example of some global/local interleaving pattern with sharing with adjacent layers.}
\end{figure}

\textbf{Quantization:} Inference is usually less sensitive to the precision of parameters and KVs. By quantizing the parameters and KV cache (e.g. to int8, int4, \texttt{fp8} etc.), we can save on memory bandwidth on both, decrease the batch size required to reach the compute roofline and save memory to run at bigger batch sizes. Quantization has the added advantage that even if the model was not trained with quantization it can often be applied post training.

\textbf{Using ragged HBM reads and Paged Attention:} We allocated 8k of context for each KV cache in the calculations above but it is often not necessary to read the entire KV cache from memory---requests have a wide range of length distributions and don't use the max context of the model, so we can often implement kernels (e.g. Flash Attention variants) that only read the non-padding part of the KV cache.

Paged Attention~\cite{paged} is a refinement upon this that stores KV caches in OS-style page tables and mostly avoids padding the KV caches altogether. This adds a lot of complexity but means every batch only uses as much memory as it needs. This is a runtime optimization, so again it is indifferent to architecture.

\begin{figure}[htb]
\centering
\includegraphics[width=\textwidth]{images/paged-attention.png}
\caption{During generation, a single token (forth) attends to multiple KV cache blocks/pages. By paging the KV cache, we avoid loading or storing more memory than we need to.}
\end{figure}

\begin{takeawaybox}
\textbf{Big Picture:} All told, these KV cache optimizations can reduce KV cache sizes by over an order of magnitude compared to a standard MHA Transformer. This can lead to an order-of-magnitude improvement in the overall cost of the Transformer.
\end{takeawaybox}

\section{Distributing Inference Over Multiple Accelerators}

So far we've handwaved how we're scaling beyond a single chip. Following Section 5, let's explore the different strategies available to us and their tradeoffs. As always, we will look at prefill and generation separately.

\subsection{Prefill}

From a roofline standpoint, \textbf{prefill is almost identical to training} and almost all the same techniques and tradeoffs apply---model (Megatron) parallelism, sequence sharding (for sufficiently long context), pipelining, even FSDP are all viable! You just have to keep the KVs kicking around so you can do generation later. As in training, increasing the number of chips gives us access to more FLOPs/s (for potentially lower TTFT), but adds communication overhead (potentially reducing throughput per chip).

\textbf{The general rule for sharding prefill:} here's a general set of rules for prefill. We'll assume we're doing prefill on a single sequence only (no batch dimension):

\begin{enumerate}
\item \textit{Model sharding:} We typically do some amount of model parallelism first, up to the point we become ICI-bound. As we saw in Section 5, this is around $F / 2200$~ for 1 axis (usually around 4-8 way sharding).
\item \textit{Sequence parallelism:} Beyond this, we do sequence parallelism (like data parallelism but sharding across the sequence dimension). While sequence parallelism introduces some extra communication in attention, it is typically fairly small at longer contexts. As with training, we can overlap the communication and computation (using collective matmuls for Megatron and ring attention respectively).
\end{enumerate}


\begin{takeawaybox}
During prefill, almost any sharding that can work during training can work fine. Do model parallelism up to the ICI bound, then do sequence parallelism.
\end{takeawaybox}


\subsection{Generation}

Generation is a more complicated beast than prefill. For one thing, it is harder to get a large batch size because we need to batch many requests together. Latency targets are lower. Together, these mean we are typically more memory-bound and more sensitive to communication overhead, which restrict our sharding strategies:

\begin{enumerate}
\item \textbf{FSDP is impossible:} since we are memory-bound in loading our parameters and KV caches from HBM to the MXU, we do not want to move them via ICI which is orders of magnitudes slower than HBM. \textit{We want to move activations rather than weights.} This means methods similar to FSDP are usually completely unviable for generation.\footnote{Accidentally leaving it on after training is an easy and common way to have order of magnitude regressions}

\item \textbf{There is no reason to do data parallelism:} pure data parallelism is unhelpful because it replicates our parameters and doesn't help us load parameters faster. You're better off spinning up multiple copies of the model instead.\footnote{By this we mean, spin up multiple servers with copies of the model at a smaller batch size. Data parallelism at the model level is strictly worse.}

\item \textbf{No sequence = no sequence sharding.} Good luck sequence sharding.
\end{enumerate}

\textit{This mostly leaves us with variants of model sharding for dense model generation}. As with prefill, the simplest thing we we can do is simple model parallelism (with activations fully replicated, weights fully sharded over hidden dimension for the MLP) up to 4-8 ways when we become ICI bound. However, since we are often memory bandwidth bound, we can actually go beyond this limit to improve latency!

\textbf{Note on ICI bounds for generation:} during training we want to be compute-bound, so our rooflines look at when our ICI comms take longer than our FLOPs. However, during generation, if we're memory bandwidth bound by parameter loading, we can increase model sharding beyond this point and improve latency at a minimal throughput cost (in terms of tokens/sec/chip). More model sharding gives us more HBM to load our weights over, and our FLOPs don't matter.\footnote{In the sense that FLOPs time isn't bottlenecking us, so the thing we need to worry about is ICI time exceeding parameter loading time.} Let's look at how much model parallelism we can do before it becomes the bottleneck.

\begin{align*}
T_\text{HBM comms} &= \frac{2DF}{Y \cdot W_\text{hbm}} \\
T_\text{ICI comms} &= \frac{2BD}{W_\text{ici}}
\end{align*}

\begin{align*}
T_\text{ICI comms} > T_\text{HBM comms} &\rightarrow \frac{W_\text{hbm}}{W_\text{ici}} > \frac{F}{Y \cdot B} \\
&\rightarrow Y > F / (B \cdot \beta)
\end{align*}

where $\beta = W_\text{hbm} / W_\text{ici}$~ This number is usually around 8 for TPU v5e and TPU v6e. That means e.g. if $F$~ is 16,384 and $B$~ is 32, we can in theory do model parallelism up to \texttt{16384 / (32 * 8) = 64}~ ways without a meaningful hit in throughput. This assume we can fully shard our KV caches 64-ways which is difficult: we discuss this below.

For the attention layer, we also model shard attention $W_Q$~ and $W_O$~ over heads Megatron style. The KV weights are quite small, and replicating them is often cheaper than sharding beyond $K$-way sharding.


\begin{takeawaybox}
Our only options during generation are variants of model parallelism. We aim to move activations instead of KV caches or parameters, which are larger. When our batch size is large, we do model parallelism up to the FLOPs-ICI bound ($F / \alpha$). When our batch size is smaller, we can improve latency by model sharding more (at a modest throughput cost). When we want to model shard more ways than we have KV heads, we can shard our KVs along the batch dimension as well.
\end{takeawaybox}


\subsection{Sharding the KV cache}

\textbf{We also have an additional data structure that needs to be sharded---the KV cache.} Again, we almost always prefer to avoid replicating the cache, since it is the primary source of attention latency. To do this, we first Megatron-shard the KVs along the head dimension. This is limited to $K$-way sharding, so for models with a small number of heads, we shard the head dimension as much as possible and then shard along the batch dimension, i.e. $\text{KV}[2, B_Z, S, K_Y, H]$~ This means the KV cache is completely distributed.

\begin{figure}[htb]
\centering
\includegraphics[width=\textwidth]{images/esta-figure.png}
\caption{comparison of the attention mechanism with (a) Multi head attention with pure model sharding and (b) Multiquery attention with batch sharding of the KV cache. Notice how we need two extra AllToAlls to shift the activations from model sharding to batch sharding, so they can act on the KV caches.}
\end{figure}

The cost of this is two AllToAlls every attention layer---one to shift the Q activations to the batch sharding so we can compute attention with batch sharding, and one to shift the batch sharded attention output back to pure model sharded.


\textbf{Here's the full algorithm!}


Here we'll write out the full attention algorithm with model parallelism over both $Y$~ and $Z$~ I apologize for using $K$~ for both the key tensor and the KV head dimension. Let $M=N/K$.

\begin{enumerate}
\item X[B, D] = ... (existing activations, unsharded from previous layer)
\item K[B\textsubscript{Z}, S, K\textsubscript{Y}, H], V[B\textsubscript{Z}, S, K, H] = ... (existing KV cache, batch sharded)
\item Q[B, N\textsubscript{YZ}, H] = X[B, D] * W\textsubscript{Q}[D, N\textsubscript{YZ}, H]
\item Q[B\textsubscript{Z}, N\textsubscript{Y}, H] = \textbf{AllToAll}\textsubscript{Z->B}(Q[B, N\textsubscript{YZ}, H])
\item Q[B\textsubscript{Z}, K\textsubscript{Y}, M, H] = \textbf{Reshape}(Q[B\textsubscript{Z}, N\textsubscript{Y}, H])
\item O[B\textsubscript{Z}, S, K\textsubscript{Y}, M] = Q[B\textsubscript{Z}, K\textsubscript{Y}, M, H] *\textsubscript{H} K[B\textsubscript{Z}, S, K\textsubscript{Y}, H]
\item O[B\textsubscript{Z}, S, K, M] = \textbf{Softmax}\textsubscript{S}(O[B\textsubscript{Z}, S, K\textsubscript{Y}])
\item O[B\textsubscript{Z}, K\textsubscript{Y}, M, H] = O[B\textsubscript{Z}, S, K, M] *\textsubscript{S} V[B\textsubscript{Z}, S, K\textsubscript{Y}, H]
\item O[B, K\textsubscript{Y}, M\textsubscript{Z}, H] = \textbf{AllToAll}\textsubscript{Z->M}(O[B\textsubscript{Z}, K\textsubscript{Y}, M, H])
\item O[B, N\textsubscript{YZ}, H] = \textbf{Reshape}(O[B, K\textsubscript{Y}, M\textsubscript{Z}, H])
\item X[B, D] \{U\textsubscript{YZ}\} = W\textsubscript{O}[N\textsubscript{YZ}, H, D] *\textsubscript{N,H} O[B, N\textsubscript{YZ}, H]
\item X[B, D] = \textbf{AllReduce}(X[B, D] \{ U\textsubscript{YZ}\})
\end{enumerate}

This is pretty complicated but you can see generally how it works. The new comms are modestly expensive since they operate on our small activations, while in return we save a huge amount of memory bandwidth loading the KVs (which are stationary).


\begin{itemize}
\item \textbf{Sequence sharding:} If the batch size is too small, or the context is long, we can sequence shard the KV cache. Again, we pay a collective cost in accumulating the attention across shards here. First we need to AllGather the Q activations, and then accumulate the KVs in a similar fashion to Flash Attention.
\end{itemize}

\section{Designing an Effective Inference Engine}

So far we've looked at how to optimize and shard the individual prefill and generate operations efficiently in isolation. To actually use them effectively, we need to design an inference engine which can feed these two operations at a point of our choosing on the latency/throughput Pareto frontier.

The simplest method is simply to run a batch of prefill, then a batch of generations:

\begin{figure}[htb]
\centering
\includegraphics[width=\textwidth]{images/batched-prefill.png}
\caption{In the simplest setup, requests are aggregated, and the server alternates between running a batch of prefills and calling the generate function until completion for all sequences.}
\end{figure}

This is easy to implement and is the first inference setup in most codebases, but it has multiple drawbacks:

\begin{enumerate}
\item \textbf{Latency is terrible.} We couple the prefill and generate batch size. Time to first token (TTFT) is terrible at big prefill batch sizes---you need to finish all prefills before any users can see any tokens. Generate throughput is terrible at small batch sizes.
\item \textbf{We block shorter generations on longer ones.} Many sequences will finish before others, leaving empty batch slots during generation, hurting generate throughput further. The problem exacerbates as batch size and generation length increases.
\item \textbf{Prefills are padded.} Prefills are padded to the longest sequence and we waste a lot of compute. There are solutions for this, but historically XLA made it quite difficult to skip these FLOPs. Again this becomes worse the bigger the batch size and prefill sequence length.
\item \textbf{We're forced to share a sharding between prefill and generation.} Both prefill and generate live on the same slice, which means we use the same topology and shardings (unless you keep two copies of the weights) for both and is generally unhelpful for performance e.g. generate wants a lot more model sharding.
\end{enumerate}

Therefore this method is only recommended for edge applications (which usually only cares about serving a single user and using hardware with less FLOPs/byte) and rapid iteration early in the lifecycle of a Transformer codebase (due to its simplicity).

A slightly better approach involves performing prefill at batch size 1 (where it is compute-bound but has reasonable latency) but batch multiple requests together during generation:

\begin{figure}[htb]
\centering
\includegraphics[width=\textwidth]{images/interleaving.png}
\end{figure}

This will avoid wasted TTFT from batched prefill while keeping generation throughput high. We call this an \textbf{interleaved} configuration, since we ``interleave'' prefill and generation steps. This is very powerful for bulk generation applications like evaluations where throughput is the main goal. The orchestrator can be configured to prioritise prefill the moment any generation slots open up, ensuring high utilisation even for very large generation batch sizes. We can also avoid padding our prefill to the maximum length, since it isn't batched with another request.

The main disadvantage is that when the server is performing a prefill, the generation of all other requests pauses since all the compute resources will be consumed by the prefill. User A whose response is busy decoding will be blocked by user B whose prefill is occurring. This means even though TTFT has improved, the token generation will be jittery and slow on average, which is not a good user experience for many applications---other user's prefills are on the critical path of the overall latency of a request.

To get around this, we separate decode and prefill. While Transformer inference can be done on one server, it is often better from a latency standpoint to execute the two different tasks on two sets of TPUs/GPUs. Prefill servers generate KV caches that get sent across the network to the generate servers, which batch multiple caches together and generate tokens for each of them. We call this \textbf{``disaggregated''} serving.

\begin{figure}[htb]
\centering
\includegraphics[width=\textwidth]{images/disaggregation.png}
\end{figure}

This provides a few advantages:

\begin{enumerate}
\item \textbf{Low latency at scale}: A user's request never blocks on another user's, except if there is insufficient prefill capacity. The request should be immediately prefilled, then sent to the generation server, then immediately slotted into the generation buffer. If we expect many concurrent requests to come in, we can scale the number of prefill servers independently from the number of generate servers so users are not left in the prefill queue for an extended period of time.

\item \textbf{Specialization:} Quite often, the latency-optimal parameter sharding strategy/hardware topology for prefill and generate is quite different (for instance, more model parallelism is useful for generate but not prefill). Constraining the two operations to use the same sharding hurts the performance of both, and having two sets of weights uses memory. Also, by moving prefill onto its own server, it doesn't need to hold any KV caches except the one it's currently processing. That means we have a lot more memory free for history caching (see the next section) or optimizing prefill latency.
\end{enumerate}

One downside is that the KV cache now needs to be shifted across the network. This is typically acceptable but again provides a motivation for reducing KV cache size.

\begin{takeawaybox}
For latency-sensitive, high-throughput serving, we typically have to separate prefill and generation into separate servers, with prefill operating at batch 1 and generation batching many concurrent requests together.
\end{takeawaybox}

\subsection{Continuous batching}

Problem (2) above motivates the concept of \textbf{continuous batching}. We optimize and compile:

\begin{itemize}
\item A number of prefill functions with variable context lengths and inserts it into some KV buffer, some maximum batch size and context length/number of pages.
\item A generate function which takes in the KV cache, and performs the generation step for all currently active requests.
\end{itemize}

We then combine these functions with an orchestrator which queues the incoming requests, calls prefill and generate depending on the available generate slots, handles history caching (see next section) and streams the tokens out.

% Note: continuous-batching.gif is an animated GIF that cannot be included in PDF
% (XeLaTeX does not support GIF format)

\subsection{Prefix caching}

Since prefill is expensive and compute-bound (giving us less headroom), one of the best ways to reduce its cost is to do less of it. Because LLMs are autoregressive, the queries [``I'', ``like'', ``dogs''] and [``I'', ``like'', ``cats''] produce KV caches that are identical in the first two tokens. What this means is that, in principle, if we compute the ``I like dogs'' cache first and then the ``I like cats'' cache, we only need to do 1 / 3 of the compute. We can save most of the work by reusing the cache. This is particularly powerful in a few specific cases:

\begin{enumerate}
\item \textbf{Chatbots}: most chatbot conversations involve a back-and-forth dialog that strictly appends to itself. This means if we can save the KV caches from each dialog turn, we can skip computation for all but the newest tokens.
\item \textbf{Few-shot prompting}: if we have any kind of few-shot prompt, this can be saved and reused for free. System instructions often have this form as well.
\end{enumerate}

The only reason this is hard to do is memory constraints. As we've seen, KV caches are big (often many GB), and for caching to be useful we need to keep them around until a follow-up query arrives. Typically, any unused HBM on the prefill servers can be used for a local caching system. Furthermore, accelerators usually have a lot of memory on their CPU hosts (e.g. a 8xTPUv5e server has 128GiB of HBM, but around 450GiB of Host DRAM). This memory is much slower than HBM---too slow to do generation steps usually---but is fast enough for a cache read. In practice:

\begin{itemize}
\item Because the KV cache is local to the set of TPUs that handled the initial request, we need some form of affinity routing to ensure follow-up queries arrive at the same replica. This can cause issues with load balancing.
\item A smaller KV cache is helpful (again)---it enables us to save more KV caches in the same amount of space, and reduce read times.
\item The KV cache and their lookups can be stored quite naturally in a tree or trie. Evictions can happen on an LRU basis.
\end{itemize}

\begin{figure}[htb]
\centering
\includegraphics[width=\textwidth]{images/prefix-caching-trie.png}
\caption{KV prefix cache implemented as an LRU trie. We can avoid duplicating KV memory by sharing prefixes.}
\end{figure}

\subsection{Let's look at an implementation: JetStream}

Google has open-sourced a library that implements this logic called \href{https://github.com/google/JetStream}{JetStream}. The server has a set of ``prefill engines'' and ``generate engines'', usually on different TPU slices, which are orchestrated by a single controller. Prefill happens in the ``\href{https://github.com/AI-Hypercomputer/JetStream/blob/c0f83127c16d7861cacc560303a28404c6cbb24c/jetstream/core/orchestrator.py\#L499}{prefill thread}'', while generation happens in the ``\href{https://github.com/AI-Hypercomputer/JetStream/blob/c0f83127c16d7861cacc560303a28404c6cbb24c/jetstream/core/orchestrator.py\#L629}{generate thread}''. We also have a ``\href{https://github.com/AI-Hypercomputer/JetStream/blob/c0f83127c16d7861cacc560303a28404c6cbb24c/jetstream/core/orchestrator.py\#L592}{transfer thread}'' that orchestrates copying the KV caches from the prefill to generate slices.

The Engine interface (implemented \href{https://github.com/google/JetStream/blob/445f1aa8e857d0a09d72618e365daf80723bdf4c/jetstream/engine/engine\_api.py\#L138}{here}) is a generic interface that any LLM must provide. The key methods are:

\begin{itemize}
\item \textbf{prefill:} takes a set of input tokens and generates a KV cache.
\item \textbf{insert:} takes a KV cache and inserts it into the batch of KV caches that generate is generating from.
\item \textbf{generate:} takes a set of batched KV caches and generates one token per batch entry, appending a single token's KV cache to the decode state for each token.
\end{itemize}

We also have a PyTorch version of JetStream available \href{https://github.com/google/jetstream-pytorch}{here}.

\section{Worked Problems}

I'm going to invent a new model based on LLaMA-2 13B for this section. Here are the details:

{\scriptsize
\setlength{\tabcolsep}{2pt}
\begin{longtable}{p{5cm}p{2.5cm}}
\hline
hyperparam & value \\
\hline
L (num\_layers) & 64 \\
D (d\_model) & 4,096 \\
F (ffw\_dimension) & 16,384 \\
N (num\_heads) & 32 \\
K (num\_kv\_heads) & 8 \\
H (qkv\_dim) & 256 \\
V (num\_embeddings) & 32,128 \\
\hline
\end{longtable}
}

\textbf{Question 1:} How many parameters does the above model have? How large are its KV caches per token in int8? \textit{You can assume we share the input and output projection matrices.}



\textbf{Question 2:} Say we want to serve this model on a TPUv5e 4x4 slice and can fully shard our KV cache over this topology. What's the largest batch size we can fit, assuming we use int8 for everything and want to support 128k sequences? What if we dropped the number of KV heads to 1?



\textbf{Question 3:} How long does it take to load all the parameters into the MXU from HBM assuming they're fully sharded on a TPU v5e 4x4 slice? Assume int8 parameters. \textit{This is a good lower bound on the per-step latency.}



\textbf{Question 4:} Let's say we want to serve this model on a TPUv5e 4x4 slice using int8 FLOPs and parameters/activations. How would we shard it for both prefill and decode? \textit{Hint: maybe answer these questions first:}

\begin{enumerate}
\item What does ICI look like on a 4x4?
\item What's the roofline bound on tensor parallelism?
\item How can we shard the KV caches?
\end{enumerate}

For this sharding, what is the rough per-step latency for generation?

\textbf{Question 5:} Let's pretend the above model is actually an MoE. An MoE model is effectively a dense model with E copies of the FFW block. Each token passes through k of the FFW blocks and these \texttt{k} are averaged to produce the output. Let's use \texttt{E=16} and \texttt{k=2} with the above settings.

\begin{enumerate}
\item How many total and activated parameters does it have? \textit{Activated means used by any given token.}
\item What batch size is needed to become FLOPs bound on TPU v5e?
\item How large are its KV caches per token?
\item How many FLOPs are involved in a forward pass with T tokens?
\end{enumerate}



\textbf{Question 6:} With MoEs, we can do ``expert sharding'', where we split our experts across one axis of our mesh. In our standard notation, our first FFW weight has shape \texttt{[E, D, F]} and we shard it as [E$_Z$, D$_X$, F$_Y$] where \texttt{X} is only used during training as our FSDP dimension. Let's say we want to do inference on a TPU v5e:

\begin{enumerate}
\item What's the HBM weight loading time for the above model on a TPU v5e 8x16 slice with Y=8, Z=16? How much free HBM is available per TPU?
\item What is the smallest slice we could fit our model on?
\end{enumerate}

\textbf{Question 7 [2D model sharding]:} Here we'll work through the math of what the \href{https://arxiv.org/pdf/2211.05102}{ESTI paper} calls 2D weight-stationary sharding. We describe this briefly in Appendix B, but try doing this problem first to see if you can work out the math. The basic idea of 2D weight stationary sharding is to shard our weights along both the $D$ and $F$ axes so that each chunk is roughly square. This reduces the comms load and allows us to scale slightly farther.

Here's the algorithm for 2D weight stationary:

\begin{enumerate}
\item In[B, D$_X$] = \textbf{AllGather}$_{YZ}$(In[B, D$_{XYZ}$])
\item Tmp[B, F$_{YZ}$] \{U.X\} = In[B, D$_X$] *$_D$ W$_{\text{in}}$[D$_X$, F$_{YZ}$]
\item Tmp[B, F$_{YZ}$] = \textbf{AllReduce}$_X$(Tmp[B, F$_{YZ}$] \{U.X\})
\item Out[B, D$_X$] \{U.YZ\} = Tmp[B, F$_{YZ}$] *$_F$ W2[F$_{YZ}$, D$_X$]
\item Out[B, D$_{XYZ}$] = \textbf{ReduceScatter}$_{YZ}$(Out[B, D$_X$] \{U.YZ\})
\end{enumerate}

Your goal is to work out $T_\text{math}$ and $T_\text{comms}$ for this algorithm and find when it will outperform traditional 3D model sharding?

Let's work out $T_\text{math}$ and $T_\text{comms}$ All our FLOPs are fully sharded so as before we have $T_\text{math} = 4BDF / (N \cdot C)$ but our comms are now

\begin{align*}
T_\text{2D comms} &= \frac{2BD}{2X \cdot W_\text{ici}} + \frac{4BF}{YZ \cdot W_\text{ici}} + \frac{2BD}{2X \cdot W_\text{ici}} \\
&= \frac{2BD}{X \cdot W_\text{ici}} + \frac{4BF}{YZ \cdot W_\text{ici}}
\end{align*}

where we note that the AllReduce is twice as expensive and we scale our comms by the number of axes over which each operation is performed. Assuming we have freedom to choose our topology and assuming $F=4D$ (as in LLaMA-2), we claim (by some basic calculus) that the optimal values for $X$, $Y$, and $Z$ are $X = \sqrt{N / 8}$, $YZ = \sqrt{8N}$ so the total communication is

\begin{align*}
T_\text{2D comms} &= \frac{2B}{W_\text{ici}} \left(\frac{D}{X} + \frac{8D}{YZ}\right) \\
&= \frac{\sqrt{128} BD}{\sqrt{N} \cdot W_\text{ici}} \approx \frac{11.3 BD}{\sqrt{N} \cdot W_\text{ici}}
\end{align*}

Firstly, copying from above, normal 1D model parallelism would have $T_\text{model parallel comms} = 4BD / (3 \cdot W_\text{ici})$, so when are the new comms smaller? We have

\begin{align*}
T_\text{model parallel comms} > T_\text{2D comms} &\iff \frac{4BD}{3 \cdot W_\text{ici}} > \frac{\sqrt{128} BD}{\sqrt{N} \cdot W_\text{ici}} \\
&\iff N > 128 \cdot \left(\frac{3}{4}\right)^2 = 81
\end{align*}

For a general $F$, we claim this condition is

$$N > 32 \cdot \left(\frac{F}{D}\right) \cdot \left(\frac{3}{4}\right)^2$$

So that tells us if we have more than 81 chips, we're better off using this new scheme. Now this is a slightly weird result because we've historically found ourselves ICI bound at around \textasciitilde 20 way tensor parallelism. But here, even if we're communication-bound, our total communication continues to decrease with the number of total chips! What this tells us is that we can continuous to increase our chips, increase our batch size, do more parameter scaling, and see reduced latency.

\section{Appendix}

\subsection{Appendix A: How real is the batch size > 240 rule?}

The simple rule we provided above, that our batch size must be greater than 240 tokens to be compute-bound, is roughly true but ignores some ability of the TPU to prefetch the weights while other operations are not using all available HBM, like when doing inter-device communication.

Here's an empirical plot of layer time (in microseconds) for a small Transformer with $d_{\text{model}}$ 8192, $d_{\text{ff}}$ 32768, and only 2 matmuls per layer. This comes from \href{https://colab.sandbox.google.com/drive/1_6krERgtolH7hbUIo7ewAMLlbA4fqEF8?usp=sharing}{this Colab notebook}. You'll see that step time increases very slowly up until around batch 240, and then increases linearly.

\begin{figure}[htb]
\centering
\includegraphics[width=\textwidth]{images/batch-scaling-latency.png}
\end{figure}

Here's the actual throughput in tokens / us. This makes the argument fairly clearly. Since our layer is about 600M parameters sharded 4 ways here, we'd expect a latency of roughly 365us at minimum.

\begin{figure}[htb]
\centering
\includegraphics[width=\textwidth]{images/batch-scaling-throughput.png}
\end{figure}

So at least in this model, we do in fact see throughput increase until about BS240 per data parallel shard.

\subsection{Appendix B: 2D Weight Stationary sharding}

As the topology grows, if we have access to higher dimensional meshes (like that of TPUs) it is possible to refine this further with ``\textbf{2D Weight Sharding}''. By introducing a second sharding axis. We call this ``\textbf{2D Weight Stationary}'', and was described in more detail in the \href{https://arxiv.org/abs/2211.05102}{Efficiently Scaling Transformer Inference paper}.

Because we're only sharding the hidden $F$ dimension in Megatron, it can become significantly smaller than $E$ (the $d_{\text{model}}$ dimension) once the number of chips grows large with 1D sharding. This means at larger batch sizes, it can be more economical to perform a portion of the collectives over the hidden dimension after the first layer of the MLP is applied.

\begin{figure}[htb]
\centering
\includegraphics[width=\textwidth]{images/2d-weight-stationary.png}
\end{figure}

This figure shows:

\begin{enumerate}
\item 1D weight-stationary sharding, a.k.a. Pure Megatron sharding, where activations are fully replicated after AllGather, and weights are fully sharded over the hidden F dimension.
\item 2D weight stationary sharding, where weights are sharded over both the hidden F and reduction E dimension, and activations are sharded over the E dimension. We perform an AllGather on the (yz) axis before the first layer, then ReduceScatter on the (x) axis.
\end{enumerate}

For the attention layer, Megatron style sharding is also relatively simple for smaller numbers of chips. However, Megatron happens over the $n_{\text{heads}}$ dimension, which puts a limit on the amount of sharding that is possible. Modifying the 2D sharding with for (instead of sharding the hidden, we shard the $n_{\text{heads}}$ dimension), we gain the ability to scale further.

\subsection{Appendix C: Latency bound communications}

As a recap, in Section 3 we derived the amount of time it takes to perform an AllGather into a tensor of size B on each TPU, over X chips on a 1D ring links of full duplex bandwidth of WICI and latency Tmin.

\begin{equation*}
T_{\text{total}} = \max\left(\frac{T_{\text{min}} \cdot |X|}{2}, \frac{B}{W_{\text{ICI}}}\right)
\end{equation*}

For large B, the wall clock stays relatively constant because as you add more chips to the system, you simultaneously scale the amount of data movement necessary to perform the operation and the total bandwidth available.

% Note: all-gather.gif is an animated GIF that cannot be included in PDF
% (XeLaTeX does not support GIF format)

Because of the relatively low amounts of data being moved during latency optimized inference, collectives on activations are often bound by the latency term (especially for small batch sizes). One can visualise the latency quite easily, by counting the number of hops we need to complete before it is completed.

On TPUs, if the tensor size-dependent part of communication is less than 1 microsecond per hop (a hop is communication between two adjacent devices) we can be bottlenecked by the fixed overhead of actually dispatching the collective. With \texttt{4.5e10} unidirectional ICI bandwidth, ICI communication becomes latency bound when: $(\text{bytes} / n_{\text{shards}}) / 4.5e10 < 1e-6$ For 8-way Megatron sharding, this is when \texttt{buffer\_size < 360kB}. \textbf{This actually is not that small during inference:} with \texttt{BS=16} and \texttt{D=8192} in int8, our activations will use \texttt{16*8192=131kB}, so we're already latency bound.

\begin{takeawaybox}
Our comms become latency bound when $\text{total bytes} < W_{\text{ICI}} \times 1e-6$~. For instance, with model parallelism over $Y$, we become bound in int8 when $Y > BD / 45,000$~.
\end{takeawaybox}

There's a parallel to be drawn here with the compute roofline---we are incurring the fixed cost of some small operations (latency for comms, memory bandwidth for matmuls).

\subsection{Appendix D: Speculative Sampling}

When we \textit{really} care about end to end latency, there is one extra trick we can employ called speculative sampling~\cite{spec1}~\cite{spec2}. As a recap, we usually generate tokens from a large Transformer one by one:

\begin{figure}[htb]
\centering
\includegraphics[width=\textwidth]{images/spec-sampling1.png}
\end{figure}

With speculative sampling, we use a smaller, cheaper model to generate tokens and then check the result with the big model. This is easiest to understand with \textit{greedy decoding}:

\begin{figure}[htb]
\centering
\includegraphics[width=\textwidth]{images/spec-sampling2.png}
\end{figure}

\begin{enumerate}
\item We sample greedily from some smaller, cheaper model. Ideally we use a model trained to match the larger model, e.g. by distillation, but it could be as simple as simply using n-grams or token matching a small corpus of text.
\item After we've generated K tokens, we use the big model to compute the next-token logits for all the tokens we've generated so far.
\item Since we're decoding greedily, we can just check if the token generated by the smaller model has the highest probability of all possible tokens. If one of the tokens is wrong, we take the longest correct prefix and replace the first wrong token with the correct token, then go back to (1). If all the tokens are correct, we can use the last correct logit to sample an extra token before going back to (1).
\end{enumerate}

\textbf{Why is this a latency win?} This scheme still requires us to do the FLOPs-equivalent of one forward pass through the big model for every token, but because we can batch a bunch of tokens together, we can do all these FLOPs in one forward pass and take advantage of the fact that we're \textit{not} \textit{compute-bound} to score more tokens for free.

Every accepted token becomes more expensive in terms of FLOPs on average (since some will be rejected, and we have to call a draft model), but we wring more FLOPs out of the hardware, and the small model is cheap, so we win overall. We also share KV cache loads across multiple steps, so \textbf{speculative decoding can also be a throughput win for long context.} Since everything has been checked by the big model, we don't change the sampling distribution at all (though the exact trajectory will differ for non-greedy).

Traditionally, speculative decoding relies on the existence of a smaller model with a similar sampling distribution to the target model, e.g. LLaMA-2 2B for LLaMA-2 70B, which often doesn't exist. Even when this is available, the smaller drafter can still be too expensive if the acceptance rate is low. Instead, it can be helpful to embed a drafter within the main model, for instance by adding a dedicated drafter head to one of the later layers of the base model~\cite{eagle}~\cite{medusa}~\cite{DeepSeek3}. Because this head shares most of its parameters with the main model, it's faster to run and matches the sampling distribution more closely.

For normal autoregressive sampling the token/s is the same as the step time. We are still beholden to the theoretical minimum step time according to the Arithmetic Intensity section here (in fact, Speculative Sampling step times are usually quite a bit slower than normal autoregressive sampling, but because we get more than 1 token out per step on average we can get much better tokens/s).

\begin{figure}[htb]
\centering
\includegraphics[width=\textwidth]{images/spec-sampling3.png}
\caption{This figure shows the per-step latency and speculation success rate for Chinchilla (a 70B model from DeepMind) with a 4B parameter drafter (small model). For XSum (a natural language dataset), the ideal amount of speculation is about 3-4 tokens ahead, while HumanEval (a coding dataset) is more predictable and sees wins from more aggressive speculation.}
\end{figure}

\textbf{How does this work for non-greedy decoding?} This is a bit more complicated, but essentially boils down to a Metropolis-Hastings inspired algorithm where have $P_{\text{draft model}}(\text{chosen token})$ and $P_{\text{target model}}(\text{chosen token})$ derived from the logits, and reject the chosen token probabilistically if the ratio of these probabilities is smaller than some threshold.

These \href{https://arxiv.org/abs/2211.17192}{two} \href{https://arxiv.org/abs/2302.01318}{papers} derived this concurrently and have good examples of how this works in practice.

\begin{takeawaybox}
Speculative sampling is yet another powerful lever for trading throughput for better per token latency. However, in the scenario where batch size is limited (e.g. small hardware footprint or large KV caches), it becomes a win-win.
\end{takeawaybox}



% Bibliography
\bibliography{main}

\end{document}
